%% Please change the file name by replacing N with the apporpriate number
%% corresponding to the current homework and XX with your initials.
%% https://www.math.uci.edu/~gpatrick/jsOnline/hw1.html

\documentclass[11pt,letterpaper]{report}
\usepackage{amssymb,amsfonts,color,graphicx,amsmath,enumerate}
\usepackage{tikz} %This package offers the ability to draw pictures
\usepackage{amsthm}

\newcommand{\naturals}{\mathbb{N}}
\newcommand{\integers}{\mathbb{Z}}
\newcommand{\complex}{\mathbb{C}}
\newcommand{\reals}{\mathbb{R}}
\newcommand{\exreals}{\overline{\mathbb{R}}}
\newcommand{\mcal}[1]{\mathcal{#1}}
\newcommand{\mable}{measurable}
\newcommand{\quats}{\mathbb{H}}
\newcommand{\rationals}{\mathbb{Q}}
\newcommand{\norm}{\trianglelefteq}
\newcommand{\Aut}{\text{Aut}}
\newcommand{\disk}{\mathbb{D}}
\newcommand{\halfplane}{\mathbb{H}}
\newcommand{\Lp}[2]{\left\|{#1}\right\|_{L^{#2}}}
\newcommand{\supp}[1]{\text{supp}({#1})}
\newcommand{\Hom}[2]{\text{Hom}_{{#1}}({#2})}
\newcommand{\tr}{\text{tr}}
\newcommand{\field}{\mathbb{F}}
\newcommand{\esssup}{\text{ess sup }}
\newcommand{\essinf}{\text{ess inf }}
\newcommand{\affine}{\mathbb{A}}
\newcommand{\Gal}{\text{Gal}}
\newcommand{\GL}{\text{GL}}
\newcommand{\SL}{\text{SL}}

\newenvironment{solution}
{\begin{proof}[Solution]}
{\end{proof}}

\voffset=-3cm
\hoffset=-2.25cm
\textheight=24cm
\textwidth=17.25cm
\addtolength{\jot}{8pt}
\linespread{1.3}

\begin{document}
%\noindent{\em Liam Hardiman\hfill{Date} }
% Please give relevant information
\begin{center}
{\bf \Large Algebra Qualifying Exams} %Replace N with the appropriate number
\vspace{0.2cm}
\hrule
\end{center}

\section*{Spring 2016}
\begin{enumerate}
	\item 
	\begin{enumerate}
		\item Prove that every subgroup of a cyclic group is cyclic.
		\begin{proof}
			Let $G$ be a cyclic group. First, suppose that $G$ is of infinite order. Then sending the generator of $G$ to $1\in \integers$ gives an isomorphism $G\cong \integers$. We claim that every subgroup of $\integers$ is of the form $n\integers$, which is clearly cyclic. Let $H$ be a subgroup of $G$ and let $n$ be the smallest positive integer in $H$. Then $n\integers \subseteq H$. Suppose the containment is strict and let $m$ be the smallest positive integer in $H\setminus n\integers$. We then have that $1\leq \gcd(m,n)<n$ and by Bezout's identity we have that $\gcd(m,n) = um+vn$ for some $u,v\in \integers$ is in $H$. But this contradicts the minimality of $n$, so we conclude that $H= n\integers$.\\

			\noindent Suppose now that $G = \langle g\rangle$ is finite of order $n$ so that $G\cong \integers/n\integers$. By the fourth isomorphism theorem (or lattice/correspondence theorem) there is a one-to-one correspondence between subgroups of $\integers/n\integers$ and subgroups of $\integers$ containing $n\integers$. We have already shown that all subgroups of $\integers$ are cyclic, and since the homomorphic image of a cyclic group is cyclic ($\varphi(g^k) = \varphi(g)^k$), we have that all subgroups of $\integers/n\integers$ are cyclic.
		\end{proof}

		\item Is the automorphism group of a cyclic group necessarily cyclic?
		\begin{solution}
			This need not be the case. Let $G \cong \integers/N\integers$. As $G$ is cyclic, any automorphism of $G$ is determined by where it sends a generator and it must map a generator to a generator. The generators of $\integers/N\integers$ exactly correspond to the elements of $(\integers/N\integers)^\times$ and we have that $\Aut(\integers/N\integers) \cong (\integers/N\integers)^\times$. If $N$ is a product of distinct primes $p, q>2$ then the Chinese remainder theorem says that
			\[
			(\integers/N\integers)^\times \cong (\integers/p\integers)^\times \times (\integers/q\integers)^\times \cong \integers/(p-1)\integers \times \integers/(q-1)\integers,
			\]
			which is not cyclic
		\end{solution}
	\end{enumerate}

	\item Let $G = \integers/25\integers$ be the cyclic group of order 25.
	\begin{enumerate}
		\item Can $G$ be given the structure of a $\integers[i]$-module?
		\begin{solution}
			Yes it can. Observe that in $\integers/25\integers$, $-1 = 24 = 6\cdot 4 = 9^2\cdot 2^2$, so $18^2 = -1$. Define the action of $a+bi\in \integers[i]$ on $m\in \integers/25\integers$ by 
			\[
			(a+bi)\cdot m = (a+18b)m.
			\]
			This action gives $G$ the structure of a $\integers[i]$-module since
			\begin{align*}
				[(a+bi)+(c+di)]\cdot m &= [(a+c)+(b+d)i]\cdot m\\
				&= [(a+c)+18(b+d)]m\\
				&= (a+bi)\cdot m + (c+di)\cdot m.\\
				(a+bi)\cdot (m+n) &= (a+18b)(m+n)\\
				&= (a+bi)\cdot m + (a+bi)\cdot n.\\
				[(a+bi)(c+di)]\cdot m&= [(ac-bd)+(ad+bc)i]m\\
				&= [(ac-bd)+18(ad+bc)]m\\
				&= (a+bi)\cdot [(c+di)\cdot m].
			\end{align*}
		\end{solution}
		\item Can $G$ be given the structure of a $\integers/5\integers$ module?
		\begin{solution}
			No it cannot. If it could then we would have
			\begin{align*}
				0 &= 0\cdot 1\\
				&= 5\cdot 1\\
				&= (1+1+1+1+1)\cdot 1\\
				&= 1\cdot 1 + 1\cdot 1 + 1\cdot 1+1\cdot 1 + 1\cdot 1\\
				&= 5\\
				&\neq 0.
			\end{align*}
		\end{solution}
	\end{enumerate}

	\item Prove that there is no simple group of order 520.
	\begin{proof}
		Write $520 = 2^3\cdot 5\cdot 13$. By Sylow's theorems, the number of Sylow-5 subgroups, $n_5$ is 1 mod 5 and divides $2^3\cdot 13 = 104$. The only positive integers satisfying these constraints are 1 and 26. The same reasoning shows that $n_{13}$ must be 1 or 40. If either of $n_5$ or $n_{13}$ is 1, then since Sylow-$p$ subgroups are conjugate to one another, then that subgroup would be normal. Suppose then that $n_5 = 26$ and $n_{13} = 40$. Then there are $26\cdot 4 = 104$ elements of order 5 and $40\cdot 12 = 480$ elements of order 13. But $104+480 = 584 > 520$, so this cannot be the case. We conclude that there must be a unique Sylow-5 or Sylow-13 subgroup which is normal.
	\end{proof}

	\item Let $G$ be a finite group acting transitively on a set $X$ with $|X|>1$.
	\begin{enumerate}
		\item State the orbit-stabilizer theorem.
		\begin{solution}
			If a group $G$ acts on a set $X$ then for any $x\in X$, the size of the orbit of $x$ under $G$ is the index of the stabilizer of $x$ under $G$, $|\mcal{O}(x)| = [G: \text{Stab}(x)]$.
		\end{solution}
		\item Show that there is some element of $G$ fixing no element of $X$.
		\begin{proof}
			Since the action of $G$ on $X$ is transitive there is a single orbit. Burnside's lemma (which immediately follows from the orbit-stabilizer theorem) then states that
			\[
			1 = \frac{1}{|G|}\sum_{g\in G}|\{x\in X: g\cdot x = x\}|.
			\]
			If $g\in G$ has fixed points then it contributes at least $1/|G|$ to the above sum. If \textit{every} element has a fixed point then we must have that every term in the sum is $1/|G|$. But the identity element fixes every element in $X$ and $|X|>1$, so we must have that one term in this sum is zero, i.e. at least one element of $G$ is fixed point free.
		\end{proof}
	\end{enumerate}

	\item Let $K$ be a field and let $\overline{K}$ be an algebraic closure of $K$. Assume $\alpha,\beta\in \overline{K}$ have degree 2 and 3 over $K$, respectively.
	\begin{enumerate}
		\item Can $\alpha\beta$ have degree 5 over $K$?
		\begin{solution}
			No it cannot. We have that $K(\alpha\beta)$ is contained in the composite extension $K(\alpha)K(\beta)$. Furthermore, since the degrees of $K(\alpha)$ and $K(\beta)$ over $K$ are relatively prime, we have that the composite extension has degree 6 over $K$. Since $K(\alpha\beta)$ is a subfield of the composite extension, its degree over $K$ must divide 6. Since 5 doesn't divide 6 we have that $K(\alpha\beta)$ cannot have degree 5 over $K$.
		\end{solution}
		\item Can $\alpha\beta$ have degree 6 over $K$?
		\begin{solution}
			Yes it can. Consider $\rationals(\sqrt{2})$ and $\rationals(\sqrt[3]{2})$. $x^2-2$ and $x^3-2$ are irreducible over $\rationals$ by Eisenstein so these extensions are of degree 2 and 3, respectively. We claim that the product $2^{1/2}\cdot 2^{1/3} = 2^{5/6}$ has degree 6 over $\rationals$.\\

			\noindent We start by showing that $\rationals(2^{5/6}) = \rationals(2^{1/6})$. Clearly $\rationals(2^{5/6})\subseteq \rationals(2^{1/6})$. On the other hand, $\frac{1}{2^4}(2^{5/6})^5 = 2^{1/6}$, so $\rationals(2^{1/6})\subseteq \rationals(2^{5/6})$. Finally, $2^{1/6}$ is a root of the polynomial $x^6-2$, which is irreducible over $\rationals$ by Eisenstein, so we have that $2^{1/6}$, and therefore $2^{5/6}$, has degree 6 over $\rationals$.
		\end{solution}
	\end{enumerate}

	\item Let $L/\rationals$ denote a Galois extension with Galois group isomorphic to $A_4$.
	\begin{enumerate}
		\item Does there exist a quadratic extension $K/\rationals$ contained in $L$?
		\begin{solution}
			By the fundamental theorem of Galois theory, a quadratic extension $K/\rationals$ corresponds to a subgroup of $\Gal(L/\rationals)\cong A_4$ of index 2 (order 6). We claim that there is no such subgroup, and therefore no such quadratic sub-extension.\\

			\noindent Any group of order 6 is isomorphic to either $\integers/6\integers$ or $S_3$. $A_4$ consists of double 2-cycles, e.g. $(1\ 2)(3\ 4)$, and 3-cycles, e.g. $(1\ 2\ 3)$, none of which have order 6, so that leaves $S_3$. $S_3$ has three elements of order 2, and since the three double 2-cycles in $A_4$ are exactly its elements of order 2, they must all lie in any subgroup isomorphic to $S_3$. However, the elements of order 2 don't commute with one another in $S_3$ while they do in $A_4$, so we cannot have a subgroup of $A_4$ isomorphic to $S_3$.
		\end{solution}
		\item Does there exist a degree 4 polynomial in $\rationals[x]$ with splitting field equal to $L$?
		\begin{solution}
			
		\end{solution}
	\end{enumerate}
	\item Let $A: V\to V$ be a linear transformation of a vector space $V$ over the field $\rationals$ which satisfies the relation $(A^3+3I)(A^3-2I)=0$. Show that the dimension $\dim_\rationals(V)$ is divisible by 3.
	\begin{proof}
		Since $A$ satisfies the polynomial $(x^3+3)(x^3-2)$, the minimal polynomial of $A$ over $\rationals$ must divide $(x^3+3)(x^3-2)$. Since $x^3+3$ and $x^3-2$ are both irreducible over $\rationals$ by Eisenstein, the minimal polynomial must be either $x^3+3$, $x^3-2$ or $(x^3+3)(x^3-2)$. Furthermore, the minimal polynomial divides the characteristic polynomial whose degree is the dimension of $V$ over $\rationals$. Since the minimal polynomial and characteristic polynomial have the same roots in an algebraic closure, we must have that the characteristic polynomial is of the form $(x^3+3)^a(x^3-2)^b$ for some nonnegative integers $a,b$ at least one of which is positive. Since the degree of such a polynomial is divisible by 3, we must have that the dimension of $V$ over $\rationals$ is divisible by 3.
	\end{proof}

	\item True/False.
	\begin{enumerate}
		\item If $K_1$, $K_2$ are fields and $\varphi: K_1\to K_2$ is a ring homomorphism such that $\varphi(1)=1$, then $\varphi$ is injective.
		\begin{solution}
			True. The kernel of $\varphi$ is an ideal in $K_1$. Since $K_1$ is a field its only ideals are $(0)$ and all of $K_1$. Since $\varphi(1) = 1$, we have that the kernel of $\varphi$ is not all of $K_1$, so it must be trivial, forcing $\varphi$ to be injective.
		\end{solution}
		\item The unit group of $\complex$ is isomorphic to the additive group of $\complex$.
		\begin{solution}
			Assuming ``the unit group of $\complex$'' means the nonzero complex numbers, $\complex^\times$, then this is false. $\complex^\times$ contains an element of order 2, $-1$, whereas the additive group $(\complex, +)$ contains no elements of order 2.
		\end{solution}
		\item Let $n$ be a positive integer. Then $\integers/n\integers\otimes_\integers\rationals = 0$.
		\begin{solution}
			True. For any simple tensor $a\otimes \frac{p}{q}$ in $\integers/n\integers\otimes_\integers \rationals$ we have
			\begin{align*}
				a\otimes \frac{p}{q} &= a\otimes \frac{np}{nq}\\
				&= na\otimes \frac{p}{nq}\\
				&= 0\otimes \frac{p}{nq}\\
				&= 0.
			\end{align*}
		\end{solution}
	\end{enumerate}

	\item For each of the following either give an example or show that none exists.
	\begin{enumerate}
		\item An element $\alpha\in \rationals(\sqrt{2}, i)$ such that $\rationals(\alpha)= \rationals(\sqrt{2}, i)$.
		\begin{solution}
			As this is a finite extension of $\rationals$ and since any such extension is separable as $\rationals$ is perfect, the primitive element theorem guarantees the existence of such an $\alpha$. We claim that $\rationals(\sqrt{2}, i) = \rationals(\sqrt{2}+i)$. The inclusion $\rationals(\sqrt{2}+i)\subseteq \rationals(\sqrt{2}, i)$ is obvious. Note that $(\sqrt{2}+i)^3 = -\sqrt{2}+5i$. This gives
			\[
			(\sqrt{2}+i)^3 + (\sqrt{2}+i) = 6i,
			\]
			so $i\in \rationals(\sqrt{2}+i)$. Subtracting $i$ from $\sqrt{2}+i$ shows that $\sqrt{2}\in \rationals(\sqrt{2}+i)$ as well and we have the reverse inclusion.
		\end{solution}
		\item A tower of field extensions $L\supseteq K'\supseteq K$ such that $L/K'$ and $K'/K$ are Galois extensions but $L/K$ is not Galois.
		\begin{solution}
			Consider the extensions $\rationals(\sqrt{2})/\rationals$ and $\rationals(\sqrt[4]{2})/\rationals(\sqrt{2})$. Both of these extensions are Galois since they are quadratic. However, $\rationals(\sqrt[4]{2})/\rationals$ is not Galois because it doesn't contain the non-real roots of $x^4-2$.
		\end{solution}
	\end{enumerate}

	\item Let $L_1, \ldots, L_r$ be all pairwise non-isomorphic complex irreducible representations of a group $G$ of order 12. What are the possible values for their dimensions $n_i = \dim_\complex L_i$? For each of the possible answers of the form $(n_1, \ldots, n_r)$ give an example of $G$ which has such irreducible representations.
	\begin{solution}
		We have that $n_1^2 + \cdots + n_r^2 = 12$ and that $r$ is the number of conjugacy classes in $G$. Since the one-dimensional trivial representation is always a representation of $G$, we know that at least one of the $n_i$ is 1. We also know that the number of conjugacy classes divides the order of $G$ by Lagrange's theorem. These constraints give us three possible configurations of the $n_i$:
		\[
		(1, 1, 1, 1, 1, 1, 1, 1, 1, 1, 1, 1)\quad (1, 1, 1, 1, 2, 2)\quad (1, 1, 1, 3).
		\]
		Here are some examples from each class.
		\begin{itemize}
			\item[$(1, \ldots, 1)$:]The irreducible representations of an abelian group are all one-dimensional, so a group like $\integers/12\integers$ would fall into this category.

			\item[$(1, 1, 1, 1, 2, 2)$:] We can safely place the dihedral group $D_{12}$ in this category because it has six conjugacy classes and this is the only configuration of the $n_i$ consistent with this.

			\item[$(1, 1, 1, 3)$:] The alternating group $A_4$ is in this category because it has four conjugacy classes.
		\end{itemize}
	\end{solution}
\end{enumerate}

\section*{Fall 2015}
\begin{enumerate}
	\item \begin{enumerate}
		\item Define \textbf{prime ideal}.
		\begin{solution}
			An ideal $P\neq R$ of the commutative ring $R$ is prime if $ab\in R$ implies that $a\in R$ or $b\in R$. Equivalently, $P$ is prime if and only if $R/P$ is an integral domain.
		\end{solution}
		\item Define \textbf{maximal ideal}.
		\begin{solution}
			An ideal $I\neq R$ of the ring $R$ is maximal if $I$ is not properly contained in another ideal (that isn't all of $R$). If $R$ is commutative, then $I$ is maximal if and only if $R/I$ is a field.
		\end{solution}
		\item Give an example of a ring $R$ and ideal $P_1$, $P_2$, and $P_3$ of $R$ such that for the properties ``prime ideal'' and ``maximal ideal'' of $R$,
		\begin{enumerate}
			\item $P_1$ satisfies both properties,
			\item $P_2$ satisfies neither property
			\item $P_3$ satisfies one property but not the other.
		\end{enumerate}
		\begin{solution}
			Let $R = \rationals[x,y]$. The ideal $P_1 = (x,y)$ is maximal since $R/P_1 = \rationals$, which is a field. Since all fields are integral domains, we have that $P_1$ is prime as well.\\

			\noindent The ideal $P_2 = (x^2)$ is not prime since $x\cdot x$ is in $P_2$ but $x$ is not. Since maximal ideals are always prime, this shows that $P_2$ isn't maximal either.\\

			\noindent The ideal $P_3 = (x)$ is prime but not maximal since $R/P_3 = \rationals[y]$, which is an integral domain but not a field.
		\end{solution}
	\end{enumerate}

	\item Show that if a group $G$ has only finitely many subgroups then $G$ is a finite group.
	\begin{proof}
		Suppose $G$ is an infinite group. If $G$ has an element $g$ of infinite order then $\langle g\rangle\cong \integers$, which has infinitely many subgroups. Suppose then that every element of $G$ has finite order. Since $G = \cup_{g\in G}\langle g\rangle$ and every $\langle g\rangle$ is a finite set, the only way to cover the infinite set $G$ by finite sets $\langle g\rangle$ is to have infinitely many distinct $\langle g\rangle$. Thus, an infinite group must have infinitely many subgroups, so a group with finitely many subgroups must be finite.
	\end{proof}

	\item Let $A$ be an $n\times n$ matrix with entires in $\reals$ such that $A^2 = -I$.
	\begin{enumerate}
		\item Prove that $n$ is even.
		\begin{proof}
			Since $A$ satisfies the polynomial $x^2+1$, its minimal polynomial over $\mathbb{R}$ must divide $x^2+1$. This polynomial is irreducible over $\mathbb{R}$ and since the minimal polynomial divides the characteristic polynomial and both polynomials have the same roots in $\complex$, the characteristic polynomial must be $(x^2+1)^d$ for some $d\geq 1$. The degree of the characteristic polynomial, $2d$, is $n$, so $n$ is even.
		\end{proof}
		\item Prove that $A$ is diagonalizable over $\complex$ and describe the corresponding diagonal matrices.
		\begin{proof}
			The minimal polynomial of $A$, $x^2+1$ splits completely into distinct linear factors over $\complex$, $(x+i)(x-i)$, so $A$ is diagonalizable over $\complex$. The Jordan blocks of $A$ are all of size 1, so $A$ is similar to a diagonal matrix with an equal number of $i$'s and $-i$'s along the diagonal. 
		\end{proof}
	\end{enumerate}

	\item Let $G$ be a group of order 70. Prove that $G$ has a normal subgroup of order 35.
	\begin{proof}
		Any subgroup of $G$ with order 35 will be normal as it will have index 2. It then suffices to simply show that $G$ has a subgroup of order 35.\\

		\noindent $70 = 2\cdot 5\cdot 7$. By Sylow's theorem, the number of Sylow-7 subgroups, $n_7$ is 1 mod 7 and divides $2\cdot 5 = 10$. The only possibility is $n_7 = 1$. Similarly we have that $n_5 = 1$ as well. Since the Sylow-$p$ subgroups are conjugate to one another, the fact that we have unique Sylow-5 and Sylow-7 subgroups shows that these subgroups, $H$ and $K$, are normal in $G$. Consequently, the product subgroup $HK$ is a subgroup of $G$ with order
		\[
		|HK| = \frac{|H|\cdot |K|}{|H\cap K|}.
		\]
		Since $H$ and $K$ are cyclic with relatively prime orders they must intersect trivially, so $|HK| = 35$.
	\end{proof}

	\item Construct a Galois extension $F$ of $\rationals$ satisfying $\Gal(F/\rationals)\cong D_8$.
	\begin{solution}
		Consider the splitting field of $p(x) = x^4-2$ over $\rationals$. $p$ has roots $i^k2^{1/4}$ for $k = 0, 1, 2, 3$, so $F = \rationals(2^{1/4}, i)$. $\rationals(2^{1/4})$ is a real degree 4 subextension. $[\rationals(2^{1/4}, i): \rationals(2^{1/4})]=2$, so we have that $[\rationals(2^{1/4}, i):\rationals] = 8$. Since $\Gal(F/\rationals)$ is the Galois group of a degree-4 polynomial, it must be a subgroup of $S_4$. Any order 8 subgroup of $S_4$ is one of its (all isomorphic) Sylow-2 subgroups. The dihedral group $D_8$ is an order 8 subgroup of $S_4$, so we must have that $\Gal(F/\rationals)\cong D_8$.
	\end{solution}

	\item Let $F$ be a field. Prove that every ideal of $F[x]$ is principal.
	\begin{proof}
		Let $I$ be a nonzero proper ideal of $F[x]$ and let $a$ be an element of $I$ of minimal degree. If $\deg a = 0$ then $a$ is a unit, in which case $I = F[x]$. We claim that $a$ divides every element of $I$, so since $(a)$ is clearly contained in $I$, we'll have that $I = (a)$. Let $b$ be any element of $I$. Since $F$ is a field we can perform polynomial division with remainder to obtain unique $q$ and $r$ with $b = aq+r$, where $r$ is either zero or has degree strictly less than that of $a$. $r$ cannot have degree strictly less than that of $a$ because $a$ was chosen to have minimal degree, so we must have that $r=0$ and $a$ divides every element of $I$.
	\end{proof}

	\item Give an example of a module $M$ over a ring $R$ such that $M$ is not finitely generated as an $R$-module.
	\begin{proof}
		Any infinitely generated abelian group will do the trick, e.g. $\bigoplus_{\naturals}\integers$. Abelian groups are $\integers$-modules, so an infinitely generated abelian group is not a finitely generated $\integers$-module.\\

		\noindent Suppose $\bigoplus_\naturals\integers$ is finitely generated. Elements of this module are of the form $(a_i)_{i=1}^\infty$ where all but finitely many of the $a_i$ are zero. If we could find a finite generating set of this module, there would be some maximal $N$ with $a_N$ nonzero an $a_j = 0$ for all $j>N$ and all generators $(a_i)$. But $\bigoplus_\naturals\integers$ clearly contains elements where $a_j\neq 0$ for $j>N$, so this module cannot be finitely generated.
	\end{proof}

	\item Suppose $H$ is a normal subgroup of a finite group $G$.
	\begin{enumerate}
		\item Prove or disprove: If $H$ has order 2, then $H$ is a subgroup of the center of $G$.
		\begin{solution}
			This is true. Write $H = \{e, h\}$ where $h\neq e$. Since $H$ is normal, $ghg^{-1}$ is in $H$ for all $g\in G$. If $ghg^{-1} = e$ then cancellation forces $h = e$, which isn't true, so we must have $ghg^{-1} = h$, which implies that $gh = hg$ for all $g$, which puts $H$ in the center of $G$.
		\end{solution}
		\item Prove or disprove: If $H$ has order 3, then $H$ is a subgroup of the center of $G$.
		\begin{solution}
			This is not true. Consider the symmetric group $S_3 = \langle r, s: r^3 = s^2 = e,\ rs = sr^2\rangle$. The subgroup generated by $r$ has order 3 and is normal since it has index 2 in $S_3$. However, it is not in the center of $S_3$ as $rs = sr^2 \neq sr$.
		\end{solution}
	\end{enumerate}

	\item \begin{enumerate}
		\item What does it mean for a representation to be irreducible?
		\begin{solution}
			A representation of $G$ is irreducible if it contains no proper subrepresentations, i.e. $V$ is irreducible if there is no proper subspace $W$ of $V$ invariant under $G$.
		\end{solution}
		\item Suppose $p$ is a prime. Let $G = \integers/p\integers$ and let $\rho: G\to \text{GL}_2(\mathbb{F}_p)$ be a representation. Show that $\rho$ is reducible.
		\begin{solution}
			Let $V$ be a subspace of $\mathbb{F}_p^2$ fixed by $G$. If we let $G$ act on $V\setminus\{0\}$ by $\rho$, the orbit stabilizer theorem tells us that the size of the orbit $|\mcal{O}_x|$ divides the order of $\integers/p\integers$, $p$, for each $x\in V\setminus \{0\}$. The orbits partition $V\setminus 0$, so if we add up the sizes of each orbit we will get the size of $V\setminus \{0\}$, which is $p^n-1$ for $n=0$, 1, or 2. Since the size of each orbit is a divisor of $p$, there must be at least one orbit of size 1, i.e. a vector fixed by the action of $G$. But then the span of this vector is a subspace of $V$ invariant under $G$: a proper subrepresentation.
		\end{solution}
	\end{enumerate}

	\item \begin{enumerate}
		\item Compute the order of $\text{GL}_4(\mathbb{F}_{3^2})$.
		\begin{solution}
			The first row in an element of $\GL_4(\field_9)$ can be any nonzero vector, of which there are $9^4-1$. The second row can be any vector not in the span of the first one, of which there are $9^4-9$. Continuing in this fashion, the $i$-th row can be any vector not in the span of the first $i-1$ rows, which gives
			\[
			|\GL_4(\field_9)| = (9^4-1)(9^4-9)(9^4-9^2)(9^4-9^3).
			\]
		\end{solution}
		\item Compute the order of $\text{SL}_4(\mathbb{F}_{3^2})$.
		\begin{solution}
			$\SL_4(\field_9)$ is the kernel of the determinant map $\det: \GL_4(\field_9)\to \field_9^\times$. We can construct invertible matrices with any desired nonzero determinant by using diagonal matrices if we like, so this map is surjective. By the first isomorphism theorem we then have
			\[
			\frac{|\GL_4(\field_9)|}{|\SL_4(\field_9)|} = |\field_9^\times|.
			\]
			By part (a) we have
			\[
			|\SL_4(\field_9)| = \frac{1}{8}(9^4-1)(9^4-9)(9^4-9^2)(9^4-9^3).
			\]
		\end{solution}
		\item Show that $\integers[\sqrt{10}]$ is not a UFD.
		\begin{solution}
			Observe that $10 = 2\cdot 5 = \sqrt{10}\cdot \sqrt{10}$. If we can show that 2, 5, and $\sqrt{10}$ are irreducible in $\integers[\sqrt{10}]$ then we will have exhibited two decompositions of 10 into non-associate irreducible factors.\\

			\noindent Define the norm $N(a+b\sqrt{10}) = a^2-10b^2$. It's easy to see that $N(\alpha\beta) = N(\alpha)N(\beta)$ and that $N(\alpha) = \pm 1$ if and only if $\alpha$ is a unit. Now $N(2) = 4=2^2$. If we can show that there are no elements with norm $\pm 2$ then we will have shown that 2 is irreducible.\\

			\noindent Suppose $N(\alpha) = a^2-10b^2 = \pm 2$. Reducing mod 4 we have $a^2 + 2b^2\equiv 2\pmod{4}$. $x^2$ can only be zero or 1 mod 4, the former if $x$ is even and the latter if $x$ is odd. We conclude that $a$ must be even and $b$ odd. Substitute $a=2m$ and $b = 2n+1$ to obtain 
			\begin{align*}
				&4m^2-10(4n^2+4n+1) = \pm 2\\
				\iff & 4(m^2-10n^2-10n)-10 = \pm 2\\
				\iff & m^2-10n^2-10n = 2\text{ or }3.
			\end{align*}
			Reducing mod 10 we have $m^2 \equiv$ 2 or 3 mod 10, which has no solutions. We conclude that no element of $\integers[\sqrt{10}]$ has norm $\pm 2$, so 2 is irreducible.\\

			\noindent Now we'll show that 5 is irreducible. $N(5) = 25 = 5^2$, so let's show that there are no elements with norm $\pm$5. Reducing $a^2-10b^2=\pm 5$ mod 5 we have that $a$ must be a multiple of 5. Substituting $a=5m$ and dividing by 5 gives $5m^2-2b^2 = \pm 1$. Again, reducing mod 5 we obtain
			\begin{align*}
				&3b^2 \equiv \pm 1\pmod{5}\\
				\iff&b^2 \equiv \pm 3\pmod{5},
			\end{align*}
			which has no solutions. We conclude that no element has norm $\pm 5$, so $5$ is irreducible.\\

			\noindent Since $N(\sqrt{10}) = -10 = -2\cdot 5$, we have also shown that $\sqrt{10}$ is irreducible. Thus, the decompositions $10 = 2\cdot 5 = \sqrt{10}\cdot \sqrt{10}$ are non-associate irreducible decompositions, so $\integers[\sqrt{10}]$ is not a UFD.
		\end{solution}
	\end{enumerate}
\end{enumerate}


\section*{Spring 2015}
\begin{enumerate}
	\item Prove that every finite group of order $>2$ has a nontrivial automorphism.
	\begin{proof}
		Let $G$ be a finite group with $|G|>2$. If $G$ is not abelian then conjugation by a non-identity element defines a non-trivial automorphism on $G$. If $G$ is abelian and there is at least one element with order not equal to 2, then $x\mapsto x^{-1}$ defines a non-trivial automorphism of $G$. This leaves the case where $G$ is abelian and every element of $G$ has order 2. By the structure theorem for finitely generated abelian groups we have that
		\[
		G \cong \integers/2\integers \oplus \integers/2\integers \oplus \cdots \oplus \integers/2\integers,
		\]
		where there are finitely many factors. In this case, any non-identity permutation of the generators gives a non-trivial automorphism of $G$.
	\end{proof}

	\item \begin{enumerate}
		\item Define UFD.
		\begin{solution}
			A UFD is an integral domain in which every element can be factored into irreducible elements that are unique up to multiplication by a unit.
		\end{solution}
		\item Define PID.
		\begin{solution}
			A PID is an integral domain in which every ideal is generated by a single element.
		\end{solution}
		\item For the properties ``UFD'' and ``PID'', give an example of a commutative integral domain that
		\begin{enumerate}
			\item satisfies both properties
			\begin{solution}
				Every PID is a UFD, so any PID like $\integers$ will satisfy both properties.
			\end{solution}
			\item satisfies one property but not the other
			\begin{solution}
				$\rationals[x,y]$ is a UFD but not a PID.
			\end{solution}
			\item satisfies neither property
			\begin{solution}
				$\integers[\sqrt{-5}]$ is the go-to example of an integral domain that isn't a UFD since $6 = 2\cdot 3 = (1+\sqrt{-5})(1-\sqrt{-5})$ even though 2, 3, and $1\pm \sqrt{-5}$ are all irreducible.
			\end{solution}
		\end{enumerate}
	\end{enumerate}

	\item \begin{enumerate}
		\item Prove that $\rationals(\sqrt[4]{T})$ is not Galois over $\rationals(T)$ where $T$ is an indeterminate.
		\begin{proof}
			$T^{1/4}$ is a root of the polynomial $x^4-T$ defined over $\rationals(T)$. An automorphism of $\rationals(T^{1/4})$ over $\rationals(T)$ is determined by where it sends $T^{1/4}$ and only two of the four possibilities, the other roots of $x^4-T$, $\{\pm T^{1/4}, \pm iT^{1/4}\}$, lie in the field. Since there are only two automorphisms and the degree of the extension is four, the extension is not Galois.
		\end{proof}
		\item Find the Galois closure of $\rationals(\sqrt[4]{T})$ over $\rationals(T)$ and determine the Galois group both as an abstract group and as a set of explicit automorphisms.
		\begin{proof}
			Since $T^{1/4}$ has minimal polynomial $x^4-T$, the Galois closure is the splitting field of this polynomial. We claim that this field is $\rationals(T^{1/4}, i)$. $\rationals(T^{1/4}, i)$ clearly contains the splitting field and since
			\[
			(iT^{1/4})/T^{1/4} = i,
			\]
			this field is contained in the splitting field, so they are one and the same.\\

			\noindent Now to compute the Galois group. Since $\rationals(T^{1/4})$ is a real field, $\rationals(T^{1/4}, i)$ is quadratic over it, so $[\rationals(T^{1/4}, i): \rationals(T)] = 8$. Since the Galois group permutes the four roots of the polynomial $x^4-T$, it is a subgroup of $S_4$ with order 8. Such a subgroup is one of the Sylow-2 subgroups of $S_4$, all of which are isomorphic to one another. Since $D_8 \leq S_4$ has order 8, we must have that the Galois group, $\Gamma$, is isomorphic to $D_8$.\\

			\noindent Let $\rho$ be the map that multiplies each root $\{\pm T^{1/4}, \pm iT^{1/4}\}$ by $i$ and let $\sigma$ be the map that sends each root to its complex conjugate. These maps clearly fix the base field $\rationals(T)$ and we have $\rho^4 = \sigma^2 = \text{id}$. A quick computation also shows that $\rho\sigma = \sigma\rho^3$: the relations defining $D_8$.
		\end{proof}
	\end{enumerate}

	\item Let $R$ be a commutative ring with multiplicative identity. An element $r\in R$ is called nilpotent if there exists a positive integer $n$ such that $r^n = 0$.
	\begin{enumerate}
		\item Prove that every nilpotent element lies in every prime ideal.
		\begin{proof}
			Let $P$ be a prime ideal, $r$ a nilpotent element of $R$, and $n$ the minimal positive integer such that $r^n = 0$. Since $P$ contains zero and $r\cdot r^{n-1} = 0\in P$, we have that $P$ contains $r$ or $r^{n-1}$ since $P$ is prime. If $P$ contains $r$ then we're done. If $P$ contains $r^{n-1}$, then the primality of $P$ says that $P$ contains one of $r$ or $r^{n-2}$. By induction, $P$ must contain $r$.
		\end{proof}
		\item Assume that every element of $R$ is either nilpotent or a unit. Prove that $R$ has a unique prime ideal.
		\begin{proof}
			Let $P$ be a prime ideal. $P$ can't contain a unit or else it would contain all of $R$, and by part (a) it must contain every nilpotent element. But $R$ has only units and nilpotent elements, so $P$ must consist exactly of the nilpotent elements. There can be only one such $P$.
		\end{proof}
	\end{enumerate}

	\item For every positive integer $n$, denote by $C_n$ a cyclic group of order $n$ and by $D_n$ a dihedral group of order $2n$, so that
	\[
	D_n = \{1, a,a^2,\ldots, a^{n-1}, b, ba, \ldots, ba^{n-1} \},
	\]
	where $a$ has order $n$, $b$ has order 2 and $ab = ba^{-1}$.
	\begin{enumerate}
		\item In the notation explained above, prove that every subgroup of $\langle a\rangle$ is normal in $D_n$.
		\begin{proof}
			$\langle a \rangle$ is normal in $D_n$ since it has index 2. Since this subgroup is cyclic of order $n$, it has a unique subgroup for each divisor $d$ of $n$ and this subgroup has order $d$. If a subgroup is the only subgroup with a given order then it is characteristic because an automorphism must send a subgroup of some order to another subgroup with the same order. We conclude that each subgroup of $\langle a \rangle$ is characteristic in $\langle a \rangle$. A characteristic subgroup of a normal subgroup is normal, so each of these subgroups is normal in $D_n$.
		\end{proof}
		\item If $n = 2m$ with $m$ odd, prove that $D_n = D_{2m} \cong C_2\times D_m$.
		\begin{proof}
			First let's show that $D_{2m}$ contains subgroups isomorphic to $C_2$ and $D_m$. Let $H = \langle b, a^2\rangle$ and let $K = \langle r^m\rangle$. It's clear that $H\cong D_m$ and $K\cong C_2$. Since $m$ is odd, $H\cap K$ contains only the identity element. $H$ is normal in $D_n$ since it has index two and $K$ is normal by part (a).\\

			\noindent Since $H$ and $K$ are normal and $H\cap K$ is trivial, we have that $HK \cong H\times K$. Furthermore, since
			\[
			|HK| = \frac{|H|\cdot |K|}{|H\cap K|} = |H|\cdot |K| = D_n,
			\]
			we have that $D_n \cong H\times K = D_m\times C_2$.
		\end{proof}
		\item Is $D_{12}\cong C_3\times D_4$?
		\begin{proof}
			It is not. Since $C_3$ has no elements of order 2, the elements of $C_3\times D_4$ of order 2 are of the form $(e,k)$, where $k\in D_4$ has order 2. In particular, this product has just as many elements of order 2 as $D_4$ does. $D_4$ (when viewed as the symmetries of the square) has five elements of order 2: the rotation through an angle of $\pi$, the two flips through opposite faces, and the two flips through opposite vertices. However, $D_{12}$, when viewed as the symmetries of a dodecagon, has at least seven elements of order 2: the $\pi$-rotation and six flips through opposite faces.
		\end{proof}
	\end{enumerate}

	\item Suppose that $p$ and $q$ are prime numbers with $p<q$. Prove that no group of order $p^2q$ is simple.
	\begin{proof}
		Let $G$ be a group of order $p^2q$. By Sylow's theorem, the number of Sylow-$q$ subgroups is 1 mod $q$ and divides $p^2$. Since $p<q$, we have that $n_q$ is 1 or $p^2$. If $n_q = 1$ then the unique Sylow-$q$ subgroup is normal and we're done. Suppose then that $n_q = p^2$. Each Sylow-$q$ subgroup is isomorphic to $\integers/q\integers$, so these subgroups intersect trivially, giving us $p^2(q-1) = p^2q-p^2$ non-identity elements. This leaves room for only $p^2$ other elements, and since we have at least one Sylow-$p$ subgroup, which must have order $p^2$, we conclude that there is only one such subgroup, forcing it to be normal.
	\end{proof}

	\item Determine the maximal ideals of the following rings.
	\begin{enumerate}
		\item $\rationals[x]/(x^2-5x+6)$,
		\begin{solution}
			By the Chinese remainder theorem we have
			\[
			\rationals[x]/(x^2-5x+6) \cong \rationals[x]/(x-2) \oplus \rationals[x]/(x-3) \cong \rationals \oplus \rationals.
			\]
			The maximal ideals in a sum of rings $R\oplus S$ are of the form $M\oplus S$ or $R\oplus N$, where $M$ is maximal in $R$ and $N$ is maximal in $S$. Since $\rationals$ is a field, its only maximal ideal is $(0)$, so the maximal ideals of this ring are $\rationals\oplus (0)$ and $(0)\oplus \rationals$.
		\end{solution}
		\item $\rationals[x]/(x^2+4x+6)$.
		\begin{solution}
			The discriminant of this quadratic is $4^2-4(1)(6)<0$, so the quadratic is irreducible. Consequently, this quotient is a field, so its only maximal ideal is $(0)$.
		\end{solution}
	\end{enumerate}

	\item Find two matrices having the same characteristic polynomials and minimal polynomials but different Jordan canonical forms.
	\begin{solution}
		Let the matrix $A$ have invariant factors $\{(x-2)^2,\ (x-2)^2\}$ and let the matrix $B$ have the invariant factors $\{(x-2),\ (x-2),\ (x-2)^2\}$. $A$ and $B$ both have the same minimal polynomial, $(x-2)^2$, and the same characteristic polynomial, $(x-2)^4$, but their Jordan forms are
		\[
		A\sim \begin{bmatrix}
			2 & 1 & 0 & 0\\
			0 & 2 & 0 & 0\\
			0 & 0 & 2 & 1\\
			0 & 0 & 0 & 2
		\end{bmatrix},\quad
		B\sim \begin{bmatrix}
			2 & 0 & 0 & 0\\
			0 & 2 & 0 & 0\\
			0 & 0 & 2 & 1\\
			0 & 0 & 0 & 2
		\end{bmatrix}.
		\]
	\end{solution}
	\item \begin{enumerate}
		\item What does it mean for a field to be perfect?
		\begin{solution}
			A field is perfect if every finite extension of it is separable.
		\end{solution}
		\item Given an example of a perfect field.
		\begin{solution}
			$\rationals$ is perfect. To see this, note that any finite extension of $\rationals$ is of the form $\rationals[x]/(p(x))$, where $p(x)$ is an irreducible polynomial. Irreducible polynomials over $\rationals$ (or any field of characteristic zero) have no repeated roots, so they are separable.
		\end{solution}
		\item Give an example of a field that is not perfect.
		\begin{solution}
			$\field_2(t)$ is not perfect. Consider the polynomial $x^2-t$. This polynomial is irreducible, but its derivative is identically zero (it factors as $(x-\sqrt{t})^2$).
		\end{solution}
	\end{enumerate}

	\item \begin{enumerate}
		\item Classify the conjugacy classes of the symmetric group $S_3$.
		\begin{solution}
			Two elements of a symmetric group are conjugate if and only if they have the same cycle type, so the conjugacy classes of $S_3$ are the identity, the two-cycles, and the three-cycles.
		\end{solution}
		\item Construct the character table of $S_3$.
		\begin{solution}
			$S_3$ has three conjugacy classes, so it has three irreducible complex representations. Any group has the trivial representation and any symmetric group has the alternating representation (which we can verify is irreducible by computing the norm of its character). We can use orthogonality relations to compute the third and final character.
			\[
			\begin{array}{r||c|c|c|}
				S_3 & e & (1\ 2) & (1\ 2\ 3)\\
				\hline\hline
				\text{Trivial}&1&1&1\\
				\hline
				\text{Alternating}&1 & -1 & 1\\
				\hline
				\text{Third Rep}&2&0&-1\\
				\hline
			\end{array}
			\]
		\end{solution}
	\end{enumerate}
\end{enumerate}

\end{document}