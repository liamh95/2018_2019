%% Please change the file name by replacing N with the apporpriate number
%% corresponding to the current homework and XX with your initials.
%% https://www.math.uci.edu/~gpatrick/jsOnline/hw1.html

\documentclass[11pt,letterpaper]{report}
\usepackage{amssymb,amsfonts,color,graphicx,amsmath,enumerate}
\usepackage{tikz} %This package offers the ability to draw pictures
\usepackage{amsthm}

\newcommand{\naturals}{\mathbb{N}}
\newcommand{\integers}{\mathbb{Z}}
\newcommand{\complex}{\mathbb{C}}
\newcommand{\reals}{\mathbb{R}}
\newcommand{\exreals}{\overline{\mathbb{R}}}
\newcommand{\mcal}[1]{\mathcal{#1}}
\newcommand{\mable}{measurable}
\newcommand{\quats}{\mathbb{H}}
\newcommand{\rationals}{\mathbb{Q}}
\newcommand{\norm}{\trianglelefteq}
\newcommand{\Aut}{\text{Aut}}
\newcommand{\disk}{\mathbb{D}}
\newcommand{\halfplane}{\mathbb{H}}
\newcommand{\Lp}[2]{\left\|{#1}\right\|_{L^{#2}}}
\newcommand{\supp}[1]{\text{supp}({#1})}
\newcommand{\Hom}{\text{Hom}}
\newcommand{\tr}{\text{tr}}
\newcommand{\field}[1]{\mathbb{F}_{{#1}}}
\newcommand{\Gal}[1]{\text{Gal}\left({#1}\right)}
\newcommand{\esssup}{\text{ess sup }}
\newcommand{\essinf}{\text{ess inf }}
\newcommand{\affine}{\mathbb{A}}
\newcommand{\projective}{\mathbb{P}}
\newcommand{\Spec}{\text{Spec}}
\newcommand{\Proj}{\text{Proj}}

\newenvironment{solution}
{\begin{proof}[Solution]}
{\end{proof}}

\voffset=-3cm
\hoffset=-2.25cm
\textheight=24cm
\textwidth=17.25cm
\addtolength{\jot}{8pt}
\linespread{1.3}

\begin{document}
\noindent{\em Liam Hardiman\hfill{March 22, 2019} }
% Please give relevant information
\begin{center}
{\bf \Large 233B - Final} %Replace N with the appropriate number
\vspace{0.2cm}
\hrule
\end{center}

%5.6.12, 5.6.13, 6.7.3, 6.7.8, 7.8.8, 7.8.10


% https://www3.nd.edu/~sevens/tanspace.pdf
% https://www.jmilne.org/math/CourseNotes/AG500.pdf
\noindent\textbf{5.6.12}
Let $X$ be a prevariety over an algebraically closed field $k$, and let $P\in X$ be a (closed) point of $X$. Let $D = \Spec\ k[x]/(x^2)$ be the ``double point''. Show that the tangent space $T_{X,P}$ to $X$ at $P$ can be canonically identified with the set of morphisms $D\to X$ that map the unique point of $D$ to $P$.
\begin{proof}
	Let $f:D\to X$ be a morphism mapping $(x)\in D$ to $P\in X$. Because morphisms of schemes correspond to homomorphisms of ringed spaces, we have a map on the stalk, $f^*: \mcal{O}_{X, P}\to k[x]/(x^2)$, that sends the maximal ideal $\mathfrak{m}_P$ to $(x)$. Write $f^*(g) = \alpha(g) + \beta(g)x\in k[x]/(x^2)$ so that $f^*(g) = \beta(g)\in (x)$ for $g\in \mathfrak{m}_P$.  We can then use $f^*$ to build a functional $\varphi: \mathfrak{m}_P\to k$ by $\varphi(g) = \beta(g)$. Now take $g,h\in \mathfrak{m}_P$. We then have
	\begin{align*}
		&f^*(gh) = f^*(g)f^*(h)\\
		\iff& \beta(gh)x = \beta(g)\beta(h)x^2\\
		\iff&\beta(gh) = 0,
	\end{align*}
	so $\mathfrak{m}_P^2\subseteq \ker \beta$ and we can consider $\varphi$ as a functional $\mathfrak{m}_P/\mathfrak{m}_P^2\to k$, an element of the tangent space at $P$. In short, we have constructed a map $\Phi: \Hom(\mcal{O}_{X, P}, k[x]/(x^2))\to \Hom(\mathfrak{m}_P/\mathfrak{m}_P^2, k)\cong T_{X, P}$ that sends $[g\mapsto \alpha(g)+\beta(g)x]$ to $[g+\mathfrak{m}_P^2\mapsto \beta(g)]$.\\

	\noindent One (I) should show that this assignment is injective.\\

	\noindent On the other hand, suppose we have functional $\varphi\in \Hom(\mathfrak{m}_P/\mathfrak{m}_P^2, k) \cong T_{X, P}$. Our goal is to use $\varphi$ to build a morphism $D\to X$ mapping $(x)$ to $P$. Since the stalk $\mcal{O}_{X, P}$ is a local ring with maximal ideal $\mathfrak{m}_P$, we can write $\mcal{O}_{X, P} = k\oplus \mathfrak{m}_P$ and uniquely define a map $f^*: \mcal{O}_{X, P}\to k[x]/(x^2)$ by specifying what it does on the components of this decomposition. Define $f^*: \mcal{O}_{X, P}\to k[x]/(x^2)$ by $f^*(g) = 0$ if $g\in k$ and $f^*(g) = \varphi(g+\mathfrak{m}_P^2)x$ if $g\in \mathfrak{m}_P$. Furthermore, this assignment is inverse to $\Phi$.
\end{proof}

\noindent\textbf{5.6.13}
Let $X$ be an affine variety, let $Y$ be a closed subscheme of $X$ defined by the ideal $I\subset A(X)$, and let $\tilde{X}$ be the blow-up of $X$ at $I$. Show that:
\begin{enumerate}[(i)]
	\item $\tilde{X} = \Proj(\bigoplus_{d\geq 0}I^d)$, where $I^{0}:= A(X)$.
	\item The projection map $\tilde{X}\to X$ is the morphism induced by the ring homomorphism $I^{0}\to \bigoplus_{d\geq 0}I^{d}$.
	\item The exceptional divisor of the blow-up, i.e. the fiber $Y\times_X\tilde{X}$ of the blow-up $\tilde{X}\to X$ over $Y$, is isomorphic to $\Proj(\bigoplus_{d\geq 0}I^{d}/I^{d+1})$.
\end{enumerate}
\begin{proof}
	
\end{proof}


\noindent\textbf{6.7.3}
Let $X\subset \projective^n$ scheme with Hilbert polynomial $\chi$. Define the arithmetic genus of $X$ to be $g(X) = (-1)^{\dim X}\cdot (\chi(0)-1)$.
\begin{enumerate}[(i)]
	\item Show that $g(\projective^n) = 0$.
	\begin{proof}
		We follow the lead of Example 6.1.2 from Gathmann's notes. The coordinate ring of $\projective^n$ is $k[x_0, \ldots, x_n]$. The corresponding Hilbert function $h_{\projective^n}(d)$ then counts the number of monomials in $k[x_0, \ldots, x_n]$ of degree $d$, so we have
		\[
		h_{\projective^n}(d) = \binom{d+n}{n}.
		\]
		This Hilbert function is a polynomial in $d$, so it coincides with the corresponding Hilbert function $\chi_{\projective^n}$ and the genus of $\projective^n$ is given by
		\begin{align*}
			g(\projective^n) &= (-1)^{\dim \projective^n}\left(\binom{n+0}{n} - 1\right)\\
			&= (-1)^n\cdot 0\\
			&= 0.
		\end{align*}
	\end{proof}
	\item If $X$ is a hypersurface of degree $d$ in $\projective^n$, show that $g(X) = \binom{d-1}{n}$. In particular, if $C\subset \projective^2$ is a plane curve of degree $d$, then $g(C) = \frac{1}{2}(d-1)(d-2)$.
	\begin{proof}
		Now we follow example 6.1.8(iii). Since the coordinate ring of $X$ is given by $k[x_0, \ldots, x_n]/(f) = k[x_0, \ldots, x_n]/(f\cdot k[x_0, \ldots, x_n])$ for some polynomial $f$ we have that
		\begin{align*}
			h_X(t) &= \dim_k(k[x_0, \ldots, x_n]/(f\cdot k[x_0, \ldots, x_n]))^{(t)}\\
			&= \dim_k k[x_0, \ldots, x_n]^{(t)} - \dim_kk[x_0, \ldots, x_n]^{(t-\deg f)}\\
			&= \binom{t+n}{n} - \binom{t-d+n}{n}.
		\end{align*}
		This is again a polynomial in $t$, so the Hilbert function and Hilbert polynomial coincide. We then have
		\begin{align*}
			g(X) &= (-1)^{\dim X}\left(\binom{0+n}{n} - \binom{-d+n}{n}-1\right)\\
			&= (-1)^{n-1}\cdot (-1)^n\binom{d-1}{n}\\
			&= \binom{d-1}{n}.
		\end{align*}
		Moving from the first to the second line we used the lesser-known (to me at least) identity $\binom{m}{k} = (-1)^k\binom{k-m-1}{k}$.\\

		\noindent If $C$ is a plane curve in $\projective^2$ then we simply substitute $n=2$ into the above formula to obtain $g(C) = \frac{1}{2}(d-1)(d-2)$ as desired.
	\end{proof}
	\item Compute the arithmetic genus of the union of the three coordinate axes
	\[
	Z(x_1x_2, x_1x_3, x_2x_3)\subset \projective^3.
	\]
	\begin{solution}
		Let $X$ be the union of the three coordinate axes in $\projective^3$. From the definition of the Hilbert function we that $h_X(d)$ is the dimension of the degree $d$ piece of the graded coordinate ring $k[x_0, \ldots, x_3]/(x_1x_2, x_1x_3, x_2x_3)$ -- the number of degree $d$ monomials divisible by at most one of $x_1$, $x_2$, $x_3$ and possibly divisible by $x_0$. These can look like $x_0^{d-k}x_i^k$ for $i=1$, 2, 3 and $1\leq k\leq d$ or $x_0^d$. There are $3d$ monomials in the former category and one in latter, so $h_X(d) = 3d+1$ for sufficiently large $d$. We then have
		\begin{align*}
		g(X) &= (-1)^{\dim X}(3\cdot 0+1-1)\\
		&= 0.
		\end{align*}
	\end{solution}
\end{enumerate}

\noindent\textbf{6.7.8}
Let $C_1 = \{f_1=0\}$ and $C_2 = \{f_2 = 0\}$ be affine curves in $\affine_k^2$, and let $P\in C_1\cap C_2$ be a point. Show that the intersection multiplicity of $C_1$ and $C_2$ at $P$ (i.e. the length of the component at $P$ of the intersection scheme $C_1\cap C_2$) is equal to the dimension of the vector space $\mcal{O}_{\affine^2, P}/(f_1, f_2)$ over $k$.
\begin{proof}
	Write $C_1 = \Spec\ k[x,y]/(f_1)$ and $C_2 = \Spec\ k[x,y]/(f_2)$. The intersection scheme is then given by $C_1\cap C_2 = \Spec\ k[x,y]/(f_1, f_2)$. Essentially, all there is to show is that looking at the component at $P$ of $C_1\cap C_2$ corresponds to localizing $k[x,y]$ at $P$ and then quotienting by $(f_1, f_2)$.\\

	\noindent The component of $C_1\cap C_2$ at $P$ corresponds to (equivalence classes of) quotients $\frac{f}{g}$ with $f,g\in k[x,y]/(f_1, f_2)$ where $g$ does not vanish at $P$. But we obtain the same set by looking at quotients $\frac{f}{g}\in k[x,y]$ where $g$ doesn't vanish at $P$, i.e. the stalk $\mcal{O}_{\affine^2, P}$, and then quotienting by $(f_1, f_2)$, so the component of $C_1\cap C_2$ at $P$ is $\mcal{O}_{\affine^2, P}/(f_1, f_2)$.\\

	\noindent Now the intersection multiplicity of $C_1$ and $C_2$ at $P$ is defined to be the length of the component at $P$ of the intersection scheme $C_1\cap C_2$. We have just shown that this component is $\Spec\ \mcal{O}_{\affine^2, P}/(f_1, f_2)$, so the length is the dimension over $k$ of $\mcal{O}_{\affine^2, P}/(f_1, f_2)$.
\end{proof}

\noindent\textbf{7.8.8}
What is the line bundle on $\projective^n\times \projective^m$ leading to the Segre embedding $\projective^n\times \projective^m\to \projective^N$ by the correspondence of lemma 7.5.14? What is the line bundle leading to the degree-$d$ Veronese embedding $\projective^n\to \projective^N$?
\begin{solution}
	The Segre embedding $S: \projective^n\times \projective^m\to \projective^N$, where $N = (n+1)(m+1)-1$ is given by $S([x_0:\ldots : x_n], [y_0:\ldots:y_m]) = [x_iy_j]$ where $0\leq i\leq m$ and $0\leq j\leq n$. By lemma 7.5.14 we have that the corresponding line bundle is given by $\mcal{L} = S^*\mcal{O}_{\projective^N}(1)$. By the discussion in example 7.2.12 we have that since $S$ is given by homogeneous degree 2 polynomials, $S^*\mcal{O}_{\projective^N}(1) = \mcal{O}_{\projective^m\times \projective^n}(2\cdot 1) = \mcal{O}_{\projective^m\times \projective^n}(2)$. Let's briefly reiterate that discussion here for the sake of completeness.\\

	\noindent Directly computing the pullback $S^*\mcal{O}_{\projective^N}(1)$ gives quotients of the form
	\[
	\frac{f(x_0y_0, \ldots, x_my_n)}{g(x_0y_0, \ldots, x_my_n)}
	\]
	where $\deg f - \deg g = 1$. But this isn't a sheaf of $\mcal{O}_{\projective^m\times \projective^n}$ modules (and therefore not a line bundle) since multiplying a section like $x_0y_0 \in S^*\mcal{O}_{\projective^N}$ by the section $\frac{x_0}{y_0}\in \mcal{O}_{\projective^m\times \projective^n}$ gives $x_0^2$, which is not of the form described by the pullback. The actual definition of the pullback sheaf is given by
	\begin{align*}
	S^*\mcal{O}_{\projective^N} &= S^{-1}\mcal{O}_{\projective^N}\otimes_{S^{-1}\mcal{O}_{\projective^N}}\mcal{O}_{\projective^m\times \projective^n},
	\end{align*}
	which is exactly the set of quotients $\frac{f}{g}$ with $\deg f - \deg g = 2$ since $\deg S = 2$.\\

	\noindent By the same reasoning, since the degree $d$ Veronese embedding $V_d: \projective^n\to \projective^N$, unsurprisingly, has degree $d$, we have that the corresponding line bundle is given by $\mcal{O}_{\projective^n}(d)$.
\end{solution}


\noindent\textbf{7.8.10}
Let $X$ be a smooth projective curve, and let $P\in X$ be a point. Show that there is a rational function on $X$ that is regular everywhere except at $P$.
\begin{proof}
	The Riemann-Roch theorem states that the dimension of the space of global sections of a divisor $D$, $h^0(D)$ satisfies
	\[
	h^0(D) - h^0(K_X-D) = \deg D + 1 - g(X),
	\]
	where $K_X$ is the divisor class associated to the canonical bundle of $X$, $\omega_X$. Let $Q$ be a point on $X$ not equal to $P$ and let $D$ be the divisor $D = kQ - P$ where $k$ is strictly larger than the genus of $X$. We then have that $\deg D = k-1$ and the Riemann-Roch theorem gives
	\begin{align*}
	h^0(D) - h^0(K_X-D)&= (k-1)+1 - g(X)\\
	&=k-g(X)\\
	&>0.
	\end{align*}
	Since $h^0(D)$ and $h^0(K_X-D)$ are both nonnegative, we must have $h^0(D)>0$, so the space of sections with divisor class $D$ is nonempty. But sections in this divisor class are regular everywhere except at $P$, so we have shown that there are functions with the desired property.
\end{proof}

\end{document}