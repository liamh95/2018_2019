%% Please change the file name by replacing N with the apporpriate number
%% corresponding to the current homework and XX with your initials.
%% https://www.math.uci.edu/~gpatrick/jsOnline/hw1.html

\documentclass[11pt,letterpaper]{report}
\usepackage{amssymb,amsfonts,color,graphicx,amsmath,enumerate}
\usepackage{tikz} %This package offers the ability to draw pictures
\usepackage{amsthm}

\newcommand{\naturals}{\mathbb{N}}
\newcommand{\integers}{\mathbb{Z}}
\newcommand{\complex}{\mathbb{C}}
\newcommand{\reals}{\mathbb{R}}
\newcommand{\exreals}{\overline{\mathbb{R}}}
\newcommand{\mcal}[1]{\mathcal{#1}}
\newcommand{\mable}{measurable}
\newcommand{\quats}{\mathbb{H}}
\newcommand{\rationals}{\mathbb{Q}}
\newcommand{\norm}{\trianglelefteq}
\newcommand{\Aut}{\text{Aut}}
\newcommand{\disk}{\mathbb{D}}
\newcommand{\halfplane}{\mathbb{H}}
\newcommand{\Lp}[2]{\left\|{#1}\right\|_{L^{#2}}}
\newcommand{\supp}[1]{\text{supp}({#1})}
\newcommand{\Hom}[2]{\text{Hom}_{{#1}}({#2})}
\newcommand{\tr}{\text{tr}}
\newcommand{\field}[1]{\mathbb{F}_{{#1}}}
\newcommand{\Gal}[1]{\text{Gal}\left({#1}\right)}
\newcommand{\esssup}{\text{ess sup }}
\newcommand{\essinf}{\text{ess inf }}
\newcommand{\affine}{\mathbb{A}}
\newcommand{\projective}{\mathbb{P}}
\newcommand{\Spec}{\text{Spec}}
\newcommand{\Proj}{\text{Proj}}

\newenvironment{solution}
{\begin{proof}[Solution]}
{\end{proof}}

\voffset=-3cm
\hoffset=-2.25cm
\textheight=24cm
\textwidth=17.25cm
\addtolength{\jot}{8pt}
\linespread{1.3}

\begin{document}
\noindent{\em Liam Hardiman\hfill{March 22, 2019} }
% Please give relevant information
\begin{center}
{\bf \Large 233B - Final} %Replace N with the appropriate number
\vspace{0.2cm}
\hrule
\end{center}

%5.6.12, 5.6.13, 6.7.3, 6.7.8, 7.8.8, 7.8.10
\noindent\textbf{5.6.12}
Let $X$ be a prevariety over an algebraically closed field $k$, and let $P\in X$ be a (closed) point of $X$. Let $D = \Spec\ k[x]/(x^2)$ be the ``double point''. Show that the tangent space $T_{X,P}$ to $X$ at $P$ can be canonically identified with the set of morphisms $D\to X$ that map the unique point of $D$ to $P$.
\begin{proof}
	
\end{proof}

\noindent\textbf{5.6.13}
Let $X$ be an affine variety, let $Y$ be a closed subscheme of $X$ defined by the ideal $I\subset A(X)$, and let $\tilde{X}$ be the blow-up of $X$ at $I$. Show that:
\begin{enumerate}[(i)]
	\item $\tilde{X} = \Proj(\bigoplus_{d\geq 0}I^{(d)})$, where $I^{(0)}:= A(X)$.
	\item The projection map $\tilde{X}\to X$ is the morphism induced by the ring homomorphism $I^{(0)}\to \bigoplus_{d\geq 0}I^{(d)}$.
	\item The exceptional divisor of the blow-up, i.e. the fiber $Y\times_X\tilde{X}$ of the blow-up $\tilde{X}\to X$ over $Y$, is isomorphic to $\Proj(\bigoplus_{d\geq 0}I^{(d)}/I^{(d+1)}$.
\end{enumerate}
\begin{proof}
	
\end{proof}


\noindent\textbf{6.7.3}
Let $X\subset \projective^n$ scheme with Hilbert polynomial $\chi$. Define the arithmetic genus of $X$ to be $g(X) = (-1)^{\dim X}\cdot (\chi(0)-1)$.
\begin{enumerate}[(i)]
	\item Show that $g(\projective^n) = 0$.
	\item If $X$ is a hypersurface of degree $d$ in $\projective^n$, show that $g(X) = \binom{d-1}{n}$. In particular, if $C\subset \projective^2$ is a plane curve of degree $d$, then $g(C) = \frac{1}{2}(d-1)(d-1)$.
	\item Compute the arithmetic genus of the union of the three coordinate axes
	\[
	Z(x_1x_2, x_1x_3, x_2x_3)\subset \projective^3.
	\]
\end{enumerate}

\noindent\textbf{6.7.8}
Let $C_1 = \{f_1=0\}$ and $C_2 = \{f_2 = 0\}$ be affine curves in $\affine_k^2$, and let $P\in C_1\cap C_2$ be a point. Show that the intersection multiplicity of $C_1$ and $C_2$ at $P$ (i.e. the length of the component at $P$ of the intersection scheme $C_1\cap C_2$) is equal to the dimension of the vector space $\mcal{O}_{\affine^2, P}/(f_1, f_2)$ over $k$.

\noindent\textbf{7.8.8}
What is the line bundle on $\projective^n\times \projective^m$ leading to the Segre embedding $\projective^n\times \projective^m\to \projective^N$ by the correspondence of ... What is the line bundle leading to the degree-$d$ Veronese embedding $\projective^n\to \projective^N$?


\noindent\textbf{7.8.10}
Let $X$ be a smooth projective curve, and let $P\in X$ be a point. Show that there is a rational function on $X$ that is regular everywhere except at $P$.

\end{document}