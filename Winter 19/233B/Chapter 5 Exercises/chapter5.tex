%% Please change the file name by replacing N with the apporpriate number
%% corresponding to the current homework and XX with your initials.
%% https://www.math.uci.edu/~gpatrick/jsOnline/hw1.html

\documentclass[11pt,letterpaper]{report}
\usepackage{amssymb,amsfonts,color,graphicx,amsmath,enumerate}
\usepackage{tikz} %This package offers the ability to draw pictures
\usepackage{amsthm}

\newcommand{\naturals}{\mathbb{N}}
\newcommand{\integers}{\mathbb{Z}}
\newcommand{\complex}{\mathbb{C}}
\newcommand{\reals}{\mathbb{R}}
\newcommand{\exreals}{\overline{\mathbb{R}}}
\newcommand{\mcal}[1]{\mathcal{#1}}
\newcommand{\mable}{measurable}
\newcommand{\quats}{\mathbb{H}}
\newcommand{\rationals}{\mathbb{Q}}
\newcommand{\norm}{\trianglelefteq}
\newcommand{\Aut}{\text{Aut}}
\newcommand{\disk}{\mathbb{D}}
\newcommand{\halfplane}{\mathbb{H}}
\newcommand{\Lp}[2]{\left\|{#1}\right\|_{L^{#2}}}
\newcommand{\supp}[1]{\text{supp}({#1})}
\newcommand{\Hom}[2]{\text{Hom}_{{#1}}({#2})}
\newcommand{\tr}{\text{tr}}
\newcommand{\field}[1]{\mathbb{F}_{{#1}}}
\newcommand{\Gal}[1]{\text{Gal}\left({#1}\right)}
\newcommand{\esssup}{\text{ess sup }}
\newcommand{\essinf}{\text{ess inf }}
\newcommand{\affine}{\mathbb{A}}
\newcommand{\Spec}{\text{Spec}}

\newenvironment{solution}
{\begin{proof}[Solution]}
{\end{proof}}

\voffset=-3cm
\hoffset=-2.25cm
\textheight=24cm
\textwidth=17.25cm
\addtolength{\jot}{8pt}
\linespread{1.3}

\begin{document}
\noindent{\em Liam Hardiman\hfill{January 28, 2019} }
% Please give relevant information
\begin{center}
{\bf \Large 233B - Chapter 5 Exercises} %Replace N with the appropriate number
\vspace{0.2cm}
\hrule
\end{center}

\noindent\textbf{Exercise 1. }Find all closed points of the real affine plan $\affine^2_\reals$. What are their residue fields?
\begin{solution}
	
\end{solution}

\noindent\textbf{Exercise 2. } Let $f(x,y) = y^2 - x^2 - x^3$. Describe the affine scheme $X = \Spec R/(f)$ set theoretically for the following rings $R$.
\begin{enumerate}[(i)]
	\item $R = \complex[x,y]$
	\begin{solution}
		$y^2 = x^2 + x^3$. It's a self-intersecting elliptic curve. Spec is closed points and zero ideal (the curve itself). To show the latter, show that the polynomial is irreducible. Try to factor it (linear times quadratic). 
	\end{solution}

	\item $R = \complex[[x, y]]$
	\begin{solution}
		If a power series has a const term it's invertible. Poly factors here $(y - \sqrt{x^2+x^3})(y+\sqrt{x^2+x^3})$. Let $P_\pm = (y \pm \sqrt{x^2+x^3})$. These are prime ideals since they eliminate $y$ upon quotienting, leaving you with $\complex[[x]]$, an integral domain. $P_\pm$ both contain the maximal ideal $m = (x,y)$. 
	\end{solution}

	\item $R = \complex[x,y]_{(x,y)}\subseteq \complex(x,y)$. The localization.
	\begin{solution}
		Kill all maximal ideals except $(x,y)$. If $f = \frac{g_1}{h_1}\cdot \frac{g_2}{h_2}$, then $f\cdot h_1\cdot h_2 = g_1\cdot g_2$. Polynomials UFD and $f$ is irreducible, so $f|g_1$ or $f|g_2$, in either cases the factorization is trivial. So the zero ideal is prime in the quotient. 
	\end{solution}
\end{enumerate}

\noindent\textbf{Exercise 3. }
\begin{enumerate}[(i)]
	\item $X$ has infinitely many points, and dim$X = 0$.
	\begin{solution}
		Need infinitely many maximal ideals but no prime ideals. Infinite product of fields.
	\end{solution}

	\item $X$ has exactly one point and dim$X = 1$.
	\begin{solution}
		Can't happen. Dim 1 gives a chain but only one point.
	\end{solution}

	\item $X$ has two points, and dim$X=1$.
	\begin{solution}
		So one maximal ideal and one prime ideal containing it. Unique maximal ideal (local ring) and one prime ideal inside. $\complex[[x]]$. Or a localization of $\complex[x]$ at $x-\alpha$.
	\end{solution}

	\item $X = \Spec(R)$ with $R\subseteq \complex [x]$, and dim$X=2$.
	\begin{solution}
		$\integers[x]$ has dimension 2. $(0)\supset (x)\supset (x,2)$. Or $\rationals[x,\pi]$ since $\pi$ is transcendental. Or $\rationals[t_1, \ldots, t_n]$ where the $t_i$ are independent complex transcendentals over $\rationals$.\\
		\noindent Couldn't happen if $R$ was a $\complex$-subalgebra. It'd contain 1 and $\complex$. $R\supseteq \complex[f]$ where $f\in R$ (of minimal positive degree?). 
	\end{solution}
\end{enumerate}

\noindent\textbf{Problem 6. }$X = $union of three coord lines in $\complex^3$. $Y = \{(x,y)\in \complex^2: xy(x-y) = 0\}$ the union of three concurrent lines in $\complex^2$. Are $X$ and $Y$ isomorphic as schemes? Use tangent spaces at origin.
\begin{solution}
	Tangent spaces are $M/M^2$. 
\end{solution}

\end{document}