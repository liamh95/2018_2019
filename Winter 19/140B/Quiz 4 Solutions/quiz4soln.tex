\documentclass[12pt]{article}  
%%Read the manual for other options. 

\pagestyle{empty} %%Eliminates page numbers
%%\input rmb_macros
%%Collect your favorite macros in a 
%%separate file

%\input amssym.def
%\input amssym
%\input mssymb
%%Defines additional symbols



\usepackage{graphics}
\usepackage{amsmath,amssymb,amsthm, multicol}
\usepackage[pdftex]{graphicx}
\usepackage{epsf}
%%Use to include pictures. 

%\newcommand{\comment}[1]{}
%\newcommand{\sobolev}[2]{W^{#1,#2}}
%\newcommand{\sobolev}[2]{L^#2_#1}
%%Some examples of macros or new commands.

\addtolength{\oddsidemargin}{-.75in}
\addtolength{\evensidemargin}{-.75in}
\addtolength{\textwidth}{1.5in}
\addtolength{\topmargin}{-1in}
\addtolength{\textheight}{2.25in}
%%Set margins, defaults are ok. 

\newenvironment{solution}
{\begin{proof}[Solution]}
{\end{proof}}

\begin{document}
\begin{flushleft} 
%%Paragraphs will not be indented in this 
%%environment
\centerline{\LARGE{Quiz 4}} 
\vspace{5 mm}
{Student ID Number:}\hfill  
%%\hfill forces following text 
%%to right margin
{Name \rule {2 in}{0.01in}}\\
Math 140B, 5PM
\\
%%gives a line of length 2in and 
%%thickness 0.01in
{Please justify all your answers}\hfill {February 15, 2019}
\\
{Please also write your full name on the back} 

\medskip
\end{flushleft}

\begin{enumerate}
	\item \begin{enumerate}
		\item Suppose that $g$ is \textit{continuous} at $x=0$. Prove that $f(x) = xg(x)$ is differentiable at $x = 0$.
		\begin{proof}
			We use the definition of the derivative.
			\[
			\lim_{x\to 0}\frac{f(x)-f(0)}{x-0} = \lim_{x\to 0}\frac{xg(x) - 0}{x} = \lim_{x\to 0}g(x) = g(0).
			\]
			The equality $\lim_{x\to 0}g(x) = g(0)$ follows from the continuity of $g$.
		\end{proof}
		\vfill
		\item Conversely, suppose that $f(0) = 0$ and $f$ is differentiable at $x=0$. Prove that there is a function $g$ that is continuous at $x=0$ and satisfies $f(x) = xg(x)$. \textit{Hint: What should $g(0)$ be?}
		\begin{proof}
			If $g(x)$ is going to satisfy $f(x) = xg(x)$ then we'll definitely have $g(x) = \frac{f(x)}{x}$ for $x\neq 0$. As it stands, this expression isn't defined at $x=0$, but it works everywhere else. If $g$ is to be continuous at zero we should be able to evaluate the limit at zero. Since $f(0)=0$ we have
			\[
			\lim_{x\to 0}g(x) = \lim_{x\to 0}\frac{f(x)}{x} = \lim_{x\to 0}\frac{f(x)-f(0)}{x-0} = f'(0).
			\]
			The last equality follows from the differentiability of $f$ at zero. Since this limit exists, if we define $g$ by
			\[
			g(x) = \begin{cases}
				\frac{f(x)}{x},&\text{if }x\neq 0\\
				f'(0),&\text{if }x=0
			\end{cases},
			\]
			then $g$ will be continuous at zero and $f(x) = xg(x)$.
		\end{proof}
	\end{enumerate}
	\vfill

	\item If $f$ and $g$ are differentiable on $[a,b]$ and $f'(x) = g'(x)$ for all $a<x<b$, show that $g(x) = f(x)+c$ for some constant $c$. Give a proof directly from the mean value theorem.
	\begin{proof}
		Define the function $h(x)= f(x)-g(x)$. Since $f$ and $g$ are continuous and differentiable on $[a,b]$, so is $h$. We can then apply the mean value theorem to $h$. For any $x,y$ satisfying $a\leq x<y\leq b$ we then have
		\[
		\frac{h(x)-h(y)}{x-y} = h'(z)
		\]
		for some $z\in (x,y)$. Now $h'(z) = (f-g)'(z) = f'(z)-g'(z) = 0$ by hypothesis. The right-hand side of the above equation is then zero, so multiplying both sides by $x-y$ shows that $h(x)-h(y) = 0$. Since this holds for any $a\leq x<y\leq b$, we have that $h$ is constant on $[a,b]$. $h$ is the difference between $f$ and $g$, so $f$ and $g$ differ by a constant.
	\end{proof}
	\vfill
\end{enumerate}

%\vfill will divide page evenly
%use \begin{enumerate} environment for ordered lists
\end{document}