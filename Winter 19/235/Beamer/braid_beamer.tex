%% Please change the file name by replacing N with the apporpriate number
%% corresponding to the current homework and XX with your initials.
%% https://www.math.uci.edu/~gpatrick/jsOnline/hw1.html

\documentclass{beamer}
\usepackage{amssymb,amsfonts,color,graphicx,amsmath,enumerate}
\usepackage{tikz} %This package offers the ability to draw pictures
\usepackage{amsthm}

\newcommand{\naturals}{\mathbb{N}}
\newcommand{\integers}{\mathbb{Z}}
\newcommand{\complex}{\mathbb{C}}
\newcommand{\reals}{\mathbb{R}}
\newcommand{\exreals}{\overline{\mathbb{R}}}
\newcommand{\mcal}[1]{\mathcal{#1}}
\newcommand{\mable}{measurable}
\newcommand{\quats}{\mathbb{H}}
\newcommand{\rationals}{\mathbb{Q}}
\newcommand{\norm}{\trianglelefteq}
\newcommand{\Aut}{\text{Aut}}
\newcommand{\disk}{\mathbb{D}}
\newcommand{\halfplane}{\mathbb{H}}
\newcommand{\Lp}[2]{\left\|{#1}\right\|_{L^{#2}}}
\newcommand{\supp}[1]{\text{supp}({#1})}
\newcommand{\Hom}[2]{\text{Hom}_{{#1}}({#2})}
\newcommand{\tr}{\text{tr}}
\newcommand{\field}[1]{\mathbb{F}_{{#1}}}
\newcommand{\Gal}[1]{\text{Gal}\left({#1}\right)}
\newcommand{\esssup}{\text{ess sup }}
\newcommand{\essinf}{\text{ess inf }}
\newcommand{\affine}{\mathbb{A}}

% \newenvironment{solution}
% {\begin{proof}[Solution]}
% {\end{proof}}

% \voffset=-3cm
% \hoffset=-2.25cm
% \textheight=24cm
% \textwidth=17.25cm
% \addtolength{\jot}{8pt}
% \linespread{1.3}

\title{Braid Group Cryptography}
\author{Liam Hardiman}

\usetheme{Frankfurt}


\begin{document}
\maketitle

\begin{frame}
	\frametitle{Finitely Presented Groups}
	A finitely presented group $G=\langle S | R\rangle$ is specified by two sets, $S = \{x_i\}_{i\in I}$ and $R = \{r_j\}_{j\in J}$.\pause
	\begin{itemize}
		\item $S$ is a set of symbols called \textbf{generators}.\pause
		\item $R$ is a set of words in $S$ called \textbf{relators}. A \textbf{word} in $S$ is a finite string consisting of symbols in $S$ and the symbols $x_i^{-1}$, where $x_i\in S$. The empty string, $e$, is also a word.\pause
		\item We form a group by taking all possible words in $S$. The inverse of a word $w$ is formed by writing the symbols in $w$ in reverse order and replacing each $x_j$ appearing in $w$ by $x_j^{-1}$. The group operation is concatenation of words.
	\end{itemize}
\end{frame}

\begin{frame}
	\frametitle{Finitely Presented Groups}
	We form $G$ from $S$ and $R$ by taking all equivalence classes of words in $S$. Two words $v$ and $w$ are equivalent if $v$ can be transformed into $w$ by a finite sequence of these operations.\pause
	\begin{enumerate}
		\item Replacing $x_ix_i^{-1}$ or $x_i^{-1}x_i$ with $e$\pause
		\item Inserting $x_ix_i^{-1}$ or $x_i^{-1}x_i$ at any position\pause
		\item Replacing $r_j$ with $e$\pause
		\item Inserting $r_j$ at any position\pause
	\end{enumerate}
	Equivalently, $G$ is the quotient of the free group on $S$ by the normal closure of $R$. We say $G$ is \textbf{finitely presented} if $S$ and $R$ are finite sets.
\end{frame}

\begin{frame}
	\frametitle{Finitely Presented Groups}
	Some examples of finitely presented groups include...\pause
	\begin{itemize}
		\item Finite groups\pause
		\item The free group $F_n$ on $n$ generators\pause
		\item Finitely generated abelian groups\pause
		\item The braid group $B_n$, $n\geq 0$.\pause
	\end{itemize}
	Nonexamples include\pause
	\begin{itemize}
		\item Any group with infinitely many generators, e.g. $\integers^{\oplus \integers}$\pause
		\item There are finitely generated groups that are not finitely related, e.g. the wreath product of $\integers$ with itself.
	\end{itemize}
\end{frame}

\begin{frame}
	\frametitle{The Word Problem}
	Say we have a finitely presented group $G$.\pause
	\begin{block}{The word problem in $G$}
		\begin{quote}
		\begin{itemize}
			\item[input: ]two words $v, w$ in the generators of $G$
			\item[output: ]\textbf{yes} if $v$ is equivalent to $w$. \textbf{no} otherwise
		\end{itemize}
		\end{quote}
	\end{block}\pause
	\begin{example}[The word problem in $F_2 = \langle a, b\rangle$]
		Iteratively scan through both words, deleting adjacent inverses.\\\pause
		Given $v = aa^{-1}bba$ and $w = babb^{-1}a^{-1}ba$, we have\pause
		\begin{align*}
			\color{red}aa^{-1}\color{black}bba &= bba\\
			b\color{red}a\color{blue}{bb^{-1}}\color{red}a^{-1}\color{black}ba &= bba.
		\end{align*}\pause
		Output \textbf{yes}.
	\end{example}
\end{frame}

\begin{frame}
	\frametitle{The Word Problem}
	In 1955 Pyotr Novikov showed that there are finitely presented groups in which the word problem is \textbf{undecidable} - it is provably impossible to construct an algorithm that always outputs the correct answer.
\end{frame}

\begin{frame}
	\frametitle{The Conjugacy Search Problem}
	Let $G$ be a group.\pause
	\begin{block}{The Conjugacy Search Problem in $G$}
		\begin{quote}
			\begin{itemize}
				\item[input: ]Two conjugate words $u$ and $v$ in the generators of $G$.
				\item[output: ]A word $w$ such that $u = w^{-1}vw = v^w$
			\end{itemize}
		\end{quote}
	\end{block}\pause
	This is analogous to the discrete logarithm problem in a finite abelian group $H$.\pause
	\begin{block}{Discrete Logarithm Problem in $H$}
		\begin{quote}
			\begin{itemize}
				\item[input: ]Elements $g,h$ of $H$ such that $h\in \langle g\rangle$
				\item[output: ]An integer $k$ such that $g^k = h$
			\end{itemize}
		\end{quote}
	\end{block}
\end{frame}

\begin{frame}
	\frametitle{The Braid Group}
	\begin{definition}
		The braid group on $n$ strands, $B_n$ is defined by the presentation
		\begin{multline*}
		B_n = \langle \sigma_1, \sigma_2, \ldots, \sigma_{n-1}\ |\ \sigma_i\sigma_j\sigma_i = \sigma_j\sigma_i\sigma_j,\ |i-j|=1;\\ \sigma_i\sigma_j = \sigma_j\sigma_i,\ |i-j|>1\rangle.
		\end{multline*}
	\end{definition}\pause
	There is, however, a more geometric understanding of the braid group.
\end{frame}

\begin{frame}
	\frametitle{The Braid Group}
	\begin{itemize}
	\item Arrange two sets of $n$ items in vertical columns on opposite sides of the page. Fasten one end of a string to each item on the left side of the page. To each item on the right side attach the other end of one string. This connection is a \textbf{braid}.\pause

	\item The generator $\sigma_i$ represents connecting the $i$-th item on the left to the $i+1$st on the right and the $i+1$st on the left to the $i$-th on the right with the latter string passing over the former.\pause

	\item Two connections that can be made to look the same by tightening the strings are considered the same braid.\pause

	\item Composing two braids consists of drawing them next to one another, gluing the points in the middle, and connecting the strands.
	\end{itemize}
\end{frame}

\begin{frame}
	\frametitle{The Braid Group}
	
\end{frame}

\end{document}