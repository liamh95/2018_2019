\documentclass[11pt]{article}  
%%Read the manual for other options. 

\pagestyle{empty} %%Eliminates page numbers
%%\input rmb_macros
%%Collect your favorite macros in a 
%%separate file

%\input amssym.def
%\input amssym
%\input mssymb
%%Defines additional symbols



\usepackage{graphics}
\usepackage{amsmath,amssymb,amsthm, multicol}
\usepackage[pdftex]{graphicx}
\usepackage{epsf}
%%Use to include pictures. 

%\newcommand{\comment}[1]{}
%\newcommand{\sobolev}[2]{W^{#1,#2}}
%\newcommand{\sobolev}[2]{L^#2_#1}
%%Some examples of macros or new commands.

\addtolength{\oddsidemargin}{-.75in}
\addtolength{\evensidemargin}{-.75in}
\addtolength{\textwidth}{1.5in}
\addtolength{\topmargin}{-1in}
\addtolength{\textheight}{2.25in}
%%Set margins, defaults are ok. 

\newcommand{\integers}{\mathbb{Z}}

\begin{document}
\begin{center}
{\bf \Large Braid Group Cryptography - Liam Hardiman}
\vspace{0.2cm}
\hrule
\end{center}

\subsection*{Background}
%Let $G$ be a finitely presented group with \textbf{generators} $x_1, x_2, \ldots, x_n$ and \textbf{relators} $r_1 = e$, $r_2 = e$, \ldots, $r_m = e$. For each generator $x_i$ there is a corresponding inverse $x_i^{-1}$. A \textbf{word} in $G$ is a finite string made of the symbols $x_i$ and $x_i^{-1}$. The empty string $e$ is a word that will serve as the identity element. Each relator $r_i$ is a word. 
A \textbf{finitely presented group} is specified by the following data.
\begin{enumerate}
	\item \textbf{Generators} $x_1, x_2, \ldots, x_n$. Just a set of symbols.
	\item \textbf{Relators} $r_1 = e$, $r_2 = e$, \ldots, $r_m = e$. More on these in a moment.
\end{enumerate}

\indent For each generator $x_i$ there is a corresponding inverse, $x_i^{-1}$. A \textbf{word} is just a finite string made of the symbols $x_i$ and $x_i^{-1}$. The empty string $e$ is a word and will be the identity element in $G$. The relators are words.\\
\indent $G$ consists of all \textit{equivalence classes} of words, where two words $u$ and $v$ are equivalent if $u$ can be transformed into $v$ by a finite sequence of cancellations or eliminating/introducing relators.\\
\indent Equivalently, $G$ is a quotient of the free group on the generators modulo the normal closure of the relators. 

\subsection*{The Word Problem}
The \textbf{word problem} is the decision problem that asks whether two words $u$ and $v$ are equivalent in $G$. In some finitely presented groups this is straight up \textbf{undecidable} -- it is provably impossible to give an algorithm that always outputs a correct answer.

\subsection*{The Conjugacy Problem}
The \textbf{conjugacy problem} is the decision problem that asks whether two words $u$ and $v$ are \textbf{conjugate} in $G$. In other words, it asks whether there exists a word $w$ such that $u = wvw^{-1}$. 


\end{document}