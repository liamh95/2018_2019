%% Please change the file name by replacing N with the apporpriate number
%% corresponding to the current homework and XX with your initials.
%% https://www.math.uci.edu/~gpatrick/jsOnline/hw1.html

\documentclass[11pt,letterpaper]{report}
\usepackage{amssymb,amsfonts,color,graphicx,amsmath,enumerate}
\usepackage{tikz} %This package offers the ability to draw pictures
\usepackage{amsthm}

\newcommand{\naturals}{\mathbb{N}}
\newcommand{\integers}{\mathbb{Z}}
\newcommand{\complex}{\mathbb{C}}
\newcommand{\reals}{\mathbb{R}}
\newcommand{\exreals}{\overline{\mathbb{R}}}
\newcommand{\mcal}[1]{\mathcal{#1}}
\newcommand{\mable}{measurable}
\newcommand{\quats}{\mathbb{H}}
\newcommand{\rationals}{\mathbb{Q}}
\newcommand{\norm}{\trianglelefteq}
\newcommand{\Aut}{\text{Aut}}
\newcommand{\disk}{\mathbb{D}}
\newcommand{\halfplane}{\mathbb{H}}
\newcommand{\Lp}[2]{\left\|{#1}\right\|_{L^{#2}}}
\newcommand{\supp}[1]{\text{supp}({#1})}
\newcommand{\Hom}[2]{\text{Hom}_{{#1}}({#2})}
\newcommand{\tr}{\text{tr}}
\newcommand{\field}[1]{\mathbb{F}_{{#1}}}
\newcommand{\Gal}[1]{\text{Gal}\left({#1}\right)}
\newcommand{\esssup}{\text{ess sup }}
\newcommand{\essinf}{\text{ess inf }}
\newcommand{\affine}{\mathbb{A}}

\newenvironment{solution}
{\begin{proof}[Solution]}
{\end{proof}}

\voffset=-3cm
\hoffset=-2.25cm
\textheight=24cm
\textwidth=17.25cm
\addtolength{\jot}{8pt}
\linespread{1.3}

\begin{document}
\noindent{\em Liam Hardiman\hfill{February 22, 2019} }
% Please give relevant information
\begin{center}
{\bf \Large 260B - Homework 1} %Replace N with the appropriate number
\vspace{0.2cm}
\hrule
\end{center}

\noindent\textbf{Problem 1. }Define the Sobolev space $H^s(\reals^n)$, $s\geq 0$ to be the set of all functions $u\in L^2(\reals^n)$ such that
\[
\|u\|_{H^s}^2 = \frac{1}{(2\pi)^n}\int |\widehat{u}(\xi)|^2(1+|\xi|^2)^s\ d\xi<\infty.
\]
\begin{enumerate}[(a)]
	\item Show that $H^s(\reals^n)$ is a Hlibert space when equipped with the scalar product
	\[
	(u,v)_{H^s} = \frac{1}{(2\pi)^n}\int \widehat{u}(\xi)\overline{\widehat{v}(\xi)}(1+|\xi|^2)^s\ d\xi.
	\]
	\begin{proof}
		Denote $\langle \xi\rangle:= (1+|\xi|^2)^{1/2}$ (apparently this is sometimes called the ``Japanese bracket'' of $\xi$).\\
		It's clear that the alleged inner product is linear, conjugate symmetric, and positive definite (since the Fourier transform is an isometry from $L^2$ to itself). That it is well-defined follows from H\"older's inequality:
		\begin{align*}
			|(u,v)| &\leq \frac{1}{(2\pi)^n}\int|\widehat{u}(\xi)||\widehat{v}(\xi)|\cdot \langle x\rangle^{2s}\ d\xi\\
			&= \frac{1}{(2\pi)^n}\int (|\widehat{u}(\xi)|\cdot \langle \xi\rangle^s)\cdot (|\widehat{v}(\xi)|\cdot \langle \xi\rangle^s)\ d\xi\\
			&\leq \frac{1}{(2\pi)^n}\Lp{\widehat{u}(\xi)\cdot \langle \xi\rangle^s}{2}\cdot \Lp{\widehat{v}(\xi)\cdot \langle \xi\rangle^s}{2}\\
			&= \|u\|_{H^s}\cdot \|v\|_{H^s}\\
			&<\infty.
		\end{align*}
		The interesting part is showing that this space is complete with respect to this norm. Suppose that $u_n$ is a Cauchy sequence in $H^s(\reals^n)$. Then for $\epsilon>0$ and $m,n$ sufficiently large we have
		\begin{align*}
			\epsilon &\geq \|u_n-u_m\|_{H^s}^2\\
			&= \frac{1}{(2\pi)^n}\int|\widehat{u_n-u_m}(\xi)|^2\cdot \langle \xi\rangle^{2s}\ d\xi\\
			&= \frac{1}{(2\pi)^n}
		\end{align*}
	\end{proof}
\end{enumerate}


\end{document}