%% Please change the file name by replacing N with the apporpriate number
%% corresponding to the current homework and XX with your initials.
%% https://www.math.uci.edu/~gpatrick/jsOnline/hw1.html

\documentclass[11pt,letterpaper]{report}
\usepackage{amssymb,amsfonts,color,graphicx,amsmath,enumerate}
\usepackage{tikz} %This package offers the ability to draw pictures
\usepackage{amsthm}

\newcommand{\naturals}{\mathbb{N}}
\newcommand{\integers}{\mathbb{Z}}
\newcommand{\complex}{\mathbb{C}}
\newcommand{\reals}{\mathbb{R}}
\newcommand{\exreals}{\overline{\mathbb{R}}}
\newcommand{\mcal}[1]{\mathcal{#1}}
\newcommand{\mable}{measurable}
\newcommand{\quats}{\mathbb{H}}
\newcommand{\rationals}{\mathbb{Q}}
\newcommand{\norm}{\trianglelefteq}
\newcommand{\Aut}{\text{Aut}}
\newcommand{\disk}{\mathbb{D}}
\newcommand{\halfplane}{\mathbb{H}}
\newcommand{\Lp}[2]{\left\|{#1}\right\|_{L^{#2}}}
\newcommand{\Hom}[2]{\text{Hom}_{{#1}}({#2})}
\newcommand{\tr}{\text{tr}}
\newcommand{\field}[1]{\mathbb{F}_{{#1}}}
\newcommand{\Gal}[1]{\text{Gal}\left({#1}\right)}
\newcommand{\esssup}{\text{ess sup }}
\newcommand{\essinf}{\text{ess inf }}
\newcommand{\affine}{\mathbb{A}}
\newcommand{\supp}{\text{supp}}

\newenvironment{solution}
{\begin{proof}[Solution]}
{\end{proof}}

\voffset=-3cm
\hoffset=-2.25cm
\textheight=24cm
\textwidth=17.25cm
\addtolength{\jot}{8pt}
\linespread{1.3}

\begin{document}
\noindent{\em Liam Hardiman\hfill{February 22, 2019} }
% Please give relevant information
\begin{center}
{\bf \Large 260B - Homework 1} %Replace N with the appropriate number
\vspace{0.2cm}
\hrule
\end{center}

\noindent\textbf{Problem 1. }Define the Sobolev space $H^s(\reals^d)$, $s\geq 0$ to be the set of all functions $u\in L^2(\reals^d)$ such that
\[
\|u\|_{H^s}^2 = \frac{1}{(2\pi)^d}\int |\widehat{u}(\xi)|^2(1+|\xi|^2)^s\ d\xi<\infty.
\]
\begin{enumerate}[(a)]
	\item Show that $H^s(\reals^d)$ is a Hlibert space when equipped with the scalar product
	\[
	(u,v)_{H^s} = \frac{1}{(2\pi)^d}\int \widehat{u}(\xi)\overline{\widehat{v}(\xi)}(1+|\xi|^2)^s\ d\xi.
	\]
	\begin{proof}
		Denote $\langle \xi\rangle:= (1+|\xi|^2)^{1/2}$ (apparently this is sometimes called the ``Japanese bracket'' of $\xi$).\\
		It's clear that the alleged inner product is linear, conjugate symmetric, and positive definite (since the Fourier transform is an isometry from $L^2$ to itself). That it is well-defined follows from H\"older's inequality:
		\begin{align*}
			|(u,v)| &\leq \frac{1}{(2\pi)^d}\int|\widehat{u}(\xi)||\widehat{v}(\xi)|\cdot \langle x\rangle^{2s}\ d\xi\\
			&= \frac{1}{(2\pi)^d}\int (|\widehat{u}(\xi)|\cdot \langle \xi\rangle^s)\cdot (|\widehat{v}(\xi)|\cdot \langle \xi\rangle^s)\ d\xi\\
			&\leq \frac{1}{(2\pi)^d}\Lp{\widehat{u}(\xi)\cdot \langle \xi\rangle^s}{2}\cdot \Lp{\widehat{v}(\xi)\cdot \langle \xi\rangle^s}{2}\\
			&= \|u\|_{H^s}\cdot \|v\|_{H^s}\\
			&<\infty.
		\end{align*}
		The interesting part is showing that this space is complete with respect to this norm. Suppose that $u_n$ is a Cauchy sequence in $H^s(\reals^d)$. Then for $\epsilon>0$ and $m,n$ sufficiently large we have
		\begin{align*}
			\epsilon &\geq \|u_n-u_m\|_{H^s}^2\\
			&= \frac{1}{(2\pi)^d}\int|\widehat{u_n-u_m}(\xi)|^2\cdot \langle \xi\rangle^{2s}\ d\xi\\
			&= \frac{1}{(2\pi)^d}\int|\widehat{u_n}(\xi)\cdot\langle \xi\rangle^s - \widehat{u_m}(\xi)\cdot \langle \xi\rangle^s|^2\ d\xi.
		\end{align*}
		So the sequence $\widehat{u_n}(\xi)\cdot\langle \xi\rangle^s$ is Cauchy in $L^2$. Since $L^2(\reals^d)$ is complete, $\widehat{u_n}(\xi)\cdot \langle \xi\rangle^s$ converges to some $v\in L^2(\reals^d)$. By H\"older's inequality $v(\xi)\cdot \langle \xi\rangle^{-s}$ is also in $L^2(\reals^d)$, so it has a well-defined inverse Fourier transform.\\

		\noindent We claim that $u_n$ converges to $\mcal{F}^{-1}(v(\xi)\cdot \langle \xi\rangle^{-s})$ in $H^s(\reals^d)$. It was designed for this purpose after all.
		\begin{align*}
			\|u_n - \mcal{F}^{-1}(v(\xi)\cdot \langle \xi\rangle^{-s})\|_{H^s}^2 &= \frac{1}{(2\pi)^d}\int |\widehat{u_n}(\xi) - v(\xi)\cdot \langle \xi\rangle^{-s}|^2\cdot \langle \xi\rangle^{2s}\ d\xi\\
			&= \frac{1}{(2\pi)^d}\int|\widehat{u_n}(\xi)\cdot \langle \xi\rangle^s - v(\xi)|^2\ d\xi\\
			&\to 0.
		\end{align*}
		That $\mcal{F}^{-1}(v(\xi)\cdot \langle \xi\rangle^{-s})$ is in $H^s(\reals^d)$ follows immediately from $v$ being in $L^2(\reals^d)$. Thus, $H^s(\reals^d)$ is complete.
	\end{proof}

	\item When $K\subseteq \reals^d$ is compact we define
	\[
	H^s(K) = \{u\in H^s(\reals^d): \supp(u)\subseteq K\}.
	\]
	Show that $H^s(K)$ is a closed linear subspace of $H^s(\reals^d)$, and hence also a Hilbert space. Show that the inclusion map $H^s(K)\to H^t(\reals^d)$ is compact if $s>t\geq 0$.
	\begin{proof}
		Let $u_n$ be a convergent sequence in $H^s(K)$. By part (a) we know that $u_n$ converges to some $u$ in $H^S(\reals^d)$ (and in $L^2(\reals^d)$). To show that $u$ indeed lives in $H^s(K)$, we need to show that its support is contained in $K$. If $u$'s support \textit{wasn't} contained in $K$ then it would have nonzero integral outside of $K$ just like all of the $u_n$'s. Let's do a computation.
		\begin{align*}
			\int_{\reals^d\setminus K}|u(x)|^2\ dx &\leq \int_{\reals^d\setminus K}|u(x)-u_n(x)|^2\ dx + \int_{\reals^d\setminus K}|u_n(x)|^2\ dx\\
			&= \int_{\reals^d\setminus K}|u(x) - u_n(x)|^2\ dx.
		\end{align*}
		Taking the limit on both sides and using the fact that $u_n$ converges to $u$ in $L^2$ shows that $u$ isn't supported outside of $K$, so $u$ lives in $H^s(K)$ and the space is closed.\\

		Now to show that the inclusion $H^s(K)\to H^t(\reals^d)$ is compact for $s>t\geq 0$. To this end, let $u_j\in H^s(K)$ be a bounded sequence. We claim that the $\widehat{u_j}$'s are smooth. To see this, we expand the exponential into its power series.
		\begin{align*}
		\widehat{u_j}(\xi) &= \int_Ku(x)e^{-ix\cdot \xi}\ dx\\
		&= \int_Ku(x)\left(\sum_{n=0}^\infty \frac{(-ix\cdot \xi)^n}{n!}\right)dx\\
		&= \sum_{n=0}^\infty \int_Ku(x)\frac{(-ix\cdot \xi)^n}{n!}\ dx.
		\end{align*}
		The interchange of summation and integration is justified since $K$ is compact and the power series of the exponential converges uniformly on compact sets. The $x\cdot \xi$ in the integrand can be expanded to show that the above sum is a series of polynomials 
	\end{proof}
\end{enumerate}


\end{document}