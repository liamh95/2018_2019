%% Please change the file name by replacing N with the apporpriate number
%% corresponding to the current homework and XX with your initials.
%% https://www.math.uci.edu/~gpatrick/jsOnline/hw1.html

\documentclass[11pt,letterpaper]{report}
\usepackage{amssymb,amsfonts,color,graphicx,amsmath,enumerate}
\usepackage{tikz} %This package offers the ability to draw pictures
\usepackage{amsthm}

\newcommand{\naturals}{\mathbb{N}}
\newcommand{\integers}{\mathbb{Z}}
\newcommand{\complex}{\mathbb{C}}
\newcommand{\reals}{\mathbb{R}}
\newcommand{\exreals}{\overline{\mathbb{R}}}
\newcommand{\mcal}[1]{\mathcal{#1}}
\newcommand{\mable}{measurable}
\newcommand{\quats}{\mathbb{H}}
\newcommand{\rationals}{\mathbb{Q}}
\newcommand{\norm}{\trianglelefteq}
\newcommand{\Aut}{\text{Aut}}
\newcommand{\disk}{\mathbb{D}}
\newcommand{\halfplane}{\mathbb{H}}
\newcommand{\Lp}[2]{\left\|{#1}\right\|_{L^{#2}}}
\newcommand{\Hom}[2]{\text{Hom}_{{#1}}({#2})}
\newcommand{\tr}{\text{tr}}
\newcommand{\field}[1]{\mathbb{F}_{{#1}}}
\newcommand{\Gal}[1]{\text{Gal}\left({#1}\right)}
\newcommand{\esssup}{\text{ess sup }}
\newcommand{\essinf}{\text{ess inf }}
\newcommand{\affine}{\mathbb{A}}
\newcommand{\supp}{\text{supp}}
\newcommand{\weak}{\rightharpoonup}

\newenvironment{solution}
{\begin{proof}[Solution]}
{\end{proof}}

\voffset=-3cm
\hoffset=-2.25cm
\textheight=24cm
\textwidth=17.25cm
\addtolength{\jot}{8pt}
\linespread{1.3}

\begin{document}
\noindent{\em Liam Hardiman\hfill{February 22, 2019} }
% Please give relevant information
\begin{center}
{\bf \Large 260B - Homework 1} %Replace N with the appropriate number
\vspace{0.2cm}
\hrule
\end{center}

\noindent\textbf{Problem 1. }Define the Sobolev space $H^s(\reals^d)$, $s\geq 0$ to be the set of all functions $u\in L^2(\reals^d)$ such that
\[
\|u\|_{H^s}^2 = \frac{1}{(2\pi)^d}\int |\widehat{u}(\xi)|^2(1+|\xi|^2)^s\ d\xi<\infty.
\]
\begin{enumerate}[(a)]
	\item Show that $H^s(\reals^d)$ is a Hlibert space when equipped with the scalar product
	\[
	(u,v)_{H^s} = \frac{1}{(2\pi)^d}\int \widehat{u}(\xi)\overline{\widehat{v}(\xi)}(1+|\xi|^2)^s\ d\xi.
	\]
	\begin{proof}
		Denote $\langle \xi\rangle:= (1+|\xi|^2)^{1/2}$ (apparently this is sometimes called the ``Japanese bracket'' of $\xi$).\\
		It's clear that the alleged inner product is linear, conjugate symmetric, and positive definite (since the Fourier transform is an isometry from $L^2$ to itself). That it is well-defined follows from H\"older's inequality:
		\begin{align*}
			|(u,v)| &\leq \frac{1}{(2\pi)^d}\int|\widehat{u}(\xi)||\widehat{v}(\xi)|\cdot \langle x\rangle^{2s}\ d\xi\\
			&= \frac{1}{(2\pi)^d}\int (|\widehat{u}(\xi)|\cdot \langle \xi\rangle^s)\cdot (|\widehat{v}(\xi)|\cdot \langle \xi\rangle^s)\ d\xi\\
			&\leq \frac{1}{(2\pi)^d}\Lp{\widehat{u}(\xi)\cdot \langle \xi\rangle^s}{2}\cdot \Lp{\widehat{v}(\xi)\cdot \langle \xi\rangle^s}{2}\\
			&= \|u\|_{H^s}\cdot \|v\|_{H^s}\\
			&<\infty.
		\end{align*}
		The interesting part is showing that this space is complete with respect to this norm. Suppose that $u_n$ is a Cauchy sequence in $H^s(\reals^d)$. Then for $\epsilon>0$ and $m,n$ sufficiently large we have
		\begin{align*}
			\epsilon &\geq \|u_n-u_m\|_{H^s}^2\\
			&= \frac{1}{(2\pi)^d}\int|\widehat{u_n-u_m}(\xi)|^2\cdot \langle \xi\rangle^{2s}\ d\xi\\
			&= \frac{1}{(2\pi)^d}\int|\widehat{u_n}(\xi)\cdot\langle \xi\rangle^s - \widehat{u_m}(\xi)\cdot \langle \xi\rangle^s|^2\ d\xi.
		\end{align*}
		So the sequence $\widehat{u_n}(\xi)\cdot\langle \xi\rangle^s$ is Cauchy in $L^2$. Since $L^2(\reals^d)$ is complete, $\widehat{u_n}(\xi)\cdot \langle \xi\rangle^s$ converges to some $v\in L^2(\reals^d)$. By H\"older's inequality $v(\xi)\cdot \langle \xi\rangle^{-s}$ is also in $L^2(\reals^d)$, so it has a well-defined inverse Fourier transform.\\

		\noindent We claim that $u_n$ converges to $\mcal{F}^{-1}(v(\xi)\cdot \langle \xi\rangle^{-s})$ in $H^s(\reals^d)$. It was designed for this purpose after all.
		\begin{align*}
			\|u_n - \mcal{F}^{-1}(v(\xi)\cdot \langle \xi\rangle^{-s})\|_{H^s}^2 &= \frac{1}{(2\pi)^d}\int |\widehat{u_n}(\xi) - v(\xi)\cdot \langle \xi\rangle^{-s}|^2\cdot \langle \xi\rangle^{2s}\ d\xi\\
			&= \frac{1}{(2\pi)^d}\int|\widehat{u_n}(\xi)\cdot \langle \xi\rangle^s - v(\xi)|^2\ d\xi\\
			&\to 0.
		\end{align*}
		That $\mcal{F}^{-1}(v(\xi)\cdot \langle \xi\rangle^{-s})$ is in $H^s(\reals^d)$ follows immediately from $v$ being in $L^2(\reals^d)$. Thus, $H^s(\reals^d)$ is complete.
	\end{proof}

	\item When $K\subseteq \reals^d$ is compact we define
	\[
	H^s(K) = \{u\in H^s(\reals^d): \supp(u)\subseteq K\}.
	\]
	Show that $H^s(K)$ is a closed linear subspace of $H^s(\reals^d)$, and hence also a Hilbert space. Show that the inclusion map $H^s(K)\to H^t(\reals^d)$ is compact if $s>t\geq 0$.
	\begin{proof}
		Let $u_n$ be a convergent sequence in $H^s(K)$. By part (a) we know that $u_n$ converges to some $u$ in $H^s(\reals^d)$ (and in $L^2(\reals^d)$). To show that $u$ indeed lives in $H^s(K)$, we need to show that its support is contained in $K$. If $u$'s support \textit{wasn't} contained in $K$ then it would have nonzero integral outside of $K$ just like all of the $u_n$'s. Let's do a computation.
		\begin{align*}
			\int_{\reals^d\setminus K}|u(x)|^2\ dx &\leq \int_{\reals^d\setminus K}|u(x)-u_n(x)|^2\ dx + \int_{\reals^d\setminus K}|u_n(x)|^2\ dx\\
			&= \int_{\reals^d\setminus K}|u(x) - u_n(x)|^2\ dx.
		\end{align*}
		Taking the limit on both sides and using the fact that $u_n$ converges to $u$ in $L^2$ shows that $u$ isn't supported outside of $K$, so $u$ lives in $H^s(K)$ and the space is closed.\\

		Now to show that the inclusion $H^s(K)\to H^t(\reals^d)$ is compact for $s>t\geq 0$. To this end, let $u_j\in H^s(K)$ be a bounded sequence, say with $\|u_j\|_{H^s(K)}\leq 1$. We claim that the $\widehat{u_j}$'s are smooth. To see this, we expand the exponential into its power series.
		\begin{align*}
		\widehat{u_j}(\xi) &= \int_Ku(x)e^{-ix\cdot \xi}\ dx\\
		&= \int_Ku(x)\left(\sum_{n=0}^\infty \frac{(-ix\cdot \xi)^n}{n!}\right)dx\\
		&= \sum_{n=0}^\infty \int_Ku(x)\frac{(-ix\cdot \xi)^n}{n!}\ dx.
		\end{align*}
		The interchange of summation and integration is justified since $K$ is compact and the power series of the exponential converges uniformly on compact sets. The $x\cdot \xi$ in the integrand can be expanded to show that the above sum is a series of polynomials. The theory of power series then shows that since the Fourier transform is defined everywhere and is given by this power series, it is smooth.\\

		Our plan is to apply the Arzela-Ascoli theorem to the sequence $\widehat{u_j}$. Let's show that this sequence is uniformly bounded. We use Parseval's theorem and the fact that the $u_j$'s are compactly supported.
		\begin{align*}
			|\widehat{u_j}(\xi)| &= \left|\int_{\reals^d}u_j(x)e^{-ix\cdot \xi}\ dx\right|\\
			&= \left|\int_Ku_j(x)e^{-ix\cdot \xi}\ dx\right|\\
			&\leq \|u_j\|_{L^2(K)}\cdot \|e^{-ix\cdot \xi}\|_{L^2(K)}\\
			&= C_K\|\widehat{u_j}\|_{L^2(\reals^d)}\\
			&\leq C_K\|u_j\|_{H^s(K)}.
		\end{align*}
		Since $\|u_j\|_{H^s(K)}\leq 1$, the Fourier transforms are uniformly bounded. The same argument shows that the partial derivatives of the $\widehat{u_j}$'s are uniformly bounded, which means that the $\widehat{u_j}$'s are Lipschitz continuous with the same Lipschitz constant. Consequently, the $\widehat{u_j}$'s are equicontinuous on compact subsets of $\reals^d$.\\

		By the Arzela-Ascoli theorem, $\widehat{u_j}$ has a uniformly convergent subsequence on every compact subset of $\reals^d$. Let $F_k$ be the closed ball in $\reals^d$ with radius $k$. We get a uniformly convergent subsequence on $F_1$ and from this we can extract a further subsequence that converges uniformly on $F_2$, and so on. Taking the diagonal entries from these subsequences gives a subsequence, $\widehat{u_{j_k}}$, that converges \textit{pointwise} on $\reals^d$.\\

		Finally, we'll show that the corresponding subsequence $u_{j_k}$ converges in $H^t(\reals^d)$.
	\end{proof}
\end{enumerate}

\noindent\textbf{Problem 2. }Let $B_1$ and $B_2$ be Banach spaces and let $T\in \mcal{L}(B_1, B_2)$. Prove that if $T$ is compact then $\|Tu_n\|_{B_2}\to 0$ for every sequence $u_n\in B_1$ such that $u_n\to 0$ in the weak topology $\sigma(B_1, B_1^*)$. prove the converse when $B_1$ is reflexive and $B_1^*$ is separable.
\begin{proof}
	Our plan is to show that any subsequence of $Tu_n$ has a further subsequence converging to zero. To this end, let $Tu_{n_j}$ be a subsequence of $Tu_n$. Since $u_n \rightharpoonup 0$, we also have that $u_{n_j}\weak 0$. By the uniform boundedness principle, $u_{n_j}$ is strongly bounded. Since $T$ is compact, $Tu_{n_j}$ has a strongly convergent subsequence, $Tu_{n_{j_k}}$. This strongly convergent subsequence is also weakly convergent and we can compute its limit. For any continuous linear functional $\eta \in B_2^*$ we have by the weak convergence of $u_n$ to zero
	\[
	\langle Tu_{n_{j_k}}, \eta\rangle_2 = \langle u_{n_{j_k}}, T^*\eta\rangle_1 \to 0.
	\]
	So $Tu_{n_{j_k}}\weak 0$. Since $Tu_{n_{j_k}}$ converges weakly \textit{and} strongly, the limits must be the same. We conclude that $Tu_{n_{j_k}} \to 0$ strongly. Thus, any subsequence of $Tu_n$ contains a further subsequence strongly converging to zero, so $Tu_n \to 0$.\\

	\noindent Conversely suppose that $B_1$ is reflexive, $B_1^*$ is separable, and that for every sequence $u_n\in B_1$ with $u_n\weak 0$ we also have that $Tu_n\to 0$ for some bounded operator $T$. Since $B_1^*$ is separable, we have by Banach-Alaoglu that the unit ball in $\sigma(B_1^{**}, B_1^*)$ is compactly metrizable. But $B_1$ is reflexive, so $B_1^{**}\cong B_1$ and the unit ball in $B_1$ is weakly compact.\\

	\noindent Let $\{u_n\}$ be a sequence in $B_1$ with $\|u_n\|\leq 1$ for all $n$. Sequential compactness is equivalent to compactness in metric spaces, and since the unit ball in $B_1$ is compactly metrizable by the above paragraph, $u_n$ has a subsequence, $u_{n_k}$, that converges weakly to some $u$, i.e. $(u_{n_k} - u)\weak 0$. By hypothesis we then have that $T(u_{n_k} - u) \to 0$, so $Tu_{n_k}$ converges strongly to $Tu$. We have then shown that $Tu_n$ has a strongly convergent subsequence, so $T$ is compact.
\end{proof}

\noindent\textbf{Problem 3. }Let $H$ be a complex separable Hilbert space. An operator $T\in \mcal{L}(H, H)$ is called a Hilbert-Schmidt operator if for some orthonormal basis $\{e_j\}$ of $H$ we have
\[
\sum \|Te_j\|^2<\infty.
\]
\begin{enumerate}[(a)]
	\item Show that if $T$ satisfies the above inequality for some orthonormal basis then it satisfies it for every orthonormal basis and the sum is independent of the choice of basis. Define $\|T\|_{HS}$ to be the square root of this sum.
	\begin{proof}
		Let $f_i$ be another orthonormal basis. By Parseval's theorem we have
		\begin{align*}
			\sum_i \|Tf_i\|^2 &= \sum_i\sum_j|\langle Tf_i, e_j\rangle|^2\\
			&= \sum_i\sum_j|\langle T^*e_j, f_i\rangle|^2\\
			&= \sum_j\|T^*e_j\|^2.
		\end{align*}
		Switching the order of summation is justified by Tonelli's theorem as each term is nonnegative. It looks like we can switch between orthonormal bases at the cost of switching from $T$ to its adjoint, $T^*$. But repeating the above calculation with $f_i = e_i$ shows that $\sum \|T^*e_j\|^2 = \sum \|Te_j\|^2$, so we have that the sum is independent of orthonormal basis.
	\end{proof}

	\item Show that the operator norm of $T$ does not exceed the Hilbert-Schmidt norm.
	\begin{proof}
		Suppose $x\in H$ has norm 1. Write $x = \sum_j \langle x, e_j\rangle e_j$. By H\"older's inequality
		\begin{align*}
			\|Tx\|^2 = \left\|\sum_j\langle x, e_j\rangle Te_j\right\|^2 \leq \sum_j|\langle x, e_j\rangle|^2\cdot \sum_j\|Te_j\|^2 = \sum_j\|Te_j\|^2.
		\end{align*}
		The right-most expression is the Hilbert-Schmidt norm. Since this holds for all unit $x$, we have that the operator norm is bounded by the Hilbert-Schmidt norm.
	\end{proof}

	\item Show that if $T$ is of Hilbert-Schmidt class, then so is $T^*$ and $\|T\|_{HS} = \|T^*\|_{HS}$.
	\begin{proof}
		We proved this when proving part (a).
	\end{proof}

	\item Show that every Hilbert-Schmidt operator is compact.
	\begin{proof}
		Our plan is to write the Hilbert-Schmidt operator, $T$, as a limit of finite rank (and therefore compact) operators. Fix an orthonormal basis $e_j$. Define $T_N$ by
		\begin{equation}\label{approx}
		T_Ne_j = \begin{cases}
			Te_j, &\text{if }1\leq j\leq N\\
			0,&\text{else}
		\end{cases}.
		\end{equation}
		Let's show that $\|T_N-T\|\to 0$ in the operator norm. For any $x\in H$ we have
		\begin{align*}
			\|(T_N - T)x\|^2 &= \left\|(T_N-T)\sum_j\langle x, e_j\rangle e_j\right\|^2\\
			&= \left\|\sum_{j=N+1}^\infty\langle x, e_j\rangle Te_j\right\|^2\\
			&\leq \sum_{j=N+1}^\infty |\langle x, e_j\rangle|^2\cdot \sum_{j=N+1}^\infty \|Te_j\|^2\\
			&\leq \|x\|^2\cdot \sum_{j=N+1}^\infty \|Te_j\|^2
		\end{align*}
		Since $T$ is Hilbert-Schmidt, the last sum here is the tail of a convergent series, so it vanishes as $N\to \infty$. Since the set of compact operators is closed, we have that $T$ is compact.
	\end{proof}

	\item Show that if $T$ is a Hilbert-Schmidt operator and $S\in \mcal{L}(H, H)$ then $ST$ is Hilbert-Schmidt and
	\[
	\|ST\|_{HS}\leq \|S\|\cdot \|T\|_{HS}.
	\]
	\begin{proof}
		A short computation.
		\begin{align*}
			\|ST\|_{HS}^2 = \sum_j\|STe_j\|^2 \leq \|S\|^2\cdot \sum_j\|Te_j\|^2 = \|S\|^2\cdot \|T\|_{HS}^2.
		\end{align*}
	\end{proof}

	\item Let $K\in L^2(\reals^n\times \reals^n)$. Prove that if $f\in L^2(\reals^n)$, then
	\[
	\mcal{K}f(x) = \int K(x,y)f(y)\ dy
	\]
	exists for almost every $x$, and that $\mcal{K}$ is a Hilbert-Schmidt operator from $L^2(\reals^n)$ to itself, with Hilbert-Schmidt norm equal to the norm of $K$ in $L^2(\reals^n\times \reals^n)$. Prove that every Hilbert-Schmidt operator on $L^2(\reals^n)$ is of this form.
	\begin{proof}
		That $\mcal{K}f$ exists a.e. and is in $L^2(\reals^n)$ follows from Fubini-Tonelli: integrating $|K(x,y)|$ with respect to $y$ gives an integrable function in $x$.
		\begin{align*}
			\|\mcal{K}f\|_{L^2}^2 &= \int\left|\int K(x,y)f(y)\ dy\right|^2dx\\
			&\leq \|f\|_{L^2}^2\cdot \|K(x,y)\|_{L^2}^2.
		\end{align*}
		Let $e_j$ be an orthonormal basis for $L^2(\reals^n)$. The theory of tensor products shows that $\overline{e_j(x)}e_k(y)$ is an orthonormal basis for $L^2(\reals^n\times \reals^n)$, so by Parseval,
		\begin{align*}
			\|\mcal{K}\|_{HS}^2 &= \sum_j\|\mcal{K}e_j\|_{L^2}^2\\
			&= \sum_j\sum_k|\langle \mcal{K}e_j, e_k\rangle|^2\\
			&= \sum_{j,k}\left|\int\int K(x,y)e_j(y)\ dy\ \overline{e_k(x)}\ dx\right|^2\\
			&= \sum_{j,k}\left|\int\int K(x,y)e_j(y)\overline{e_k(x)}\ dydx\right|^2\\
			&= \sum_{j,k}\langle K, \overline{e_j}e_k\rangle\\
			&= \|K\|_{L^2}.
		\end{align*}
		Let $T$ be a Hilbert-Schmidt operator on $L^2$. The idea is to approximate $T$ by operators of the desired form and then show that the limit also has the desired form, much in the same way we proved part (d). Fix an orthonormal basis $e_j$ for $L^2(\reals^n)$ and define $T_N$ by (\ref{approx}) on the previous page. For any $f\in L^2(\reals^n)$ we have
		\begin{align*}
			(T_Nf)(x) &= \sum_{j=1}^N\langle f, e_j\rangle Te_j
			= \sum_{j=1}^N\left[\int f(y)\overline{e_j}(y)\ dy\right](Te_j)(x)\\
			&= \int\left[\sum_{j=1}^N(Te_j)(x)\overline{e_j}(y)\right]f(y)\ dy.
		\end{align*}
		This motivates us to define $K_N(x,y) := \sum_{j=1}^N(Te_j)(x)\overline{e_j}(y)$. That $T_N\to T$ in the operator norm follows from our discussion of part (d). It remains to show that $T_N$ converges to an operator of the desired form.   
	\end{proof}
\end{enumerate}

\noindent\textbf{Problem 4. }Let $K$ be a compact self-adjoint operator on a Hilbert space $H$, and assume that $K\geq 0$. Let $\lambda_1 \geq \lambda_2\geq \ldots$ be the sequence of non-zero eigenvalues of $K$, repeated according to their multiplicity and arranged in a decreasing order. Prove the Courant-Fischer minimax formula
\[
\lambda_k = \min_{codim\ k = K-1}\max_{u\in V, \|u\|\leq 1}(Ku, u).
\]
\begin{proof}
	By the spectral theorem for compact self-adjoint operators, we have an orthonormal basis of eigenvectors, $e_j$, of $K$. Let $E_k$ be the span of $e_1, \ldots, e_k$. Then $E_k^\perp$ is spanned by $e_{k+1}, \ldots$ and
	\begin{align*}
	\max_{u\in E_k^\perp, \|u\|\leq 1}(Ku,u) &= \max_{u\in E_k^\perp, \|u\|\leq 1}\sum_{j=k+1}^\infty \lambda_j|(u, e_j)|^2\\
	&\leq\max_{u\in E_k^\perp, \|u\|\leq 1} \lambda_{k+1}\sum_{j = k+1}^\infty |(u, e_j)|^2\\
	&\leq \lambda_{k+1}.
	\end{align*}
	Now suppose that $V$ has codimension $k-1$. We can split $H$ as $H = V\oplus V^\perp = E_k\oplus E_k^\perp$. Since $E_k$ has dimension $k$ and $V^\perp$ has dimension $k-1$, we must have that $E_k$ and $V$ overlap. 
\end{proof}


\end{document}