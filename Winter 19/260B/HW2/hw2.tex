%% Please change the file name by replacing N with the apporpriate number
%% corresponding to the current homework and XX with your initials.
%% https://www.math.uci.edu/~gpatrick/jsOnline/hw1.html

\documentclass[11pt,letterpaper]{report}
\usepackage{amssymb,amsfonts,color,graphicx,amsmath,enumerate}
\usepackage{tikz} %This package offers the ability to draw pictures
\usepackage{amsthm}

\newcommand{\naturals}{\mathbb{N}}
\newcommand{\integers}{\mathbb{Z}}
\newcommand{\complex}{\mathbb{C}}
\newcommand{\reals}{\mathbb{R}}
\newcommand{\exreals}{\overline{\mathbb{R}}}
\newcommand{\mcal}[1]{\mathcal{#1}}
\newcommand{\mable}{measurable}
\newcommand{\quats}{\mathbb{H}}
\newcommand{\rationals}{\mathbb{Q}}
\newcommand{\norm}{\trianglelefteq}
\newcommand{\Aut}{\text{Aut}}
\newcommand{\disk}{\mathbb{D}}
\newcommand{\halfplane}{\mathbb{H}}
\newcommand{\Lp}[2]{\left\|{#1}\right\|_{L^{#2}}}
\newcommand{\supp}[1]{\text{supp}({#1})}
\newcommand{\Hom}[2]{\text{Hom}_{{#1}}({#2})}
\newcommand{\tr}{\text{tr}}
\newcommand{\field}[1]{\mathbb{F}_{{#1}}}
\newcommand{\Gal}[1]{\text{Gal}\left({#1}\right)}
\newcommand{\esssup}{\text{ess sup }}
\newcommand{\essinf}{\text{ess inf }}
\newcommand{\affine}{\mathbb{A}}
\newcommand{\im}{\text{Im}}

\newenvironment{solution}
{\begin{proof}[Solution]}
{\end{proof}}

\voffset=-3cm
\hoffset=-2.25cm
\textheight=24cm
\textwidth=17.25cm
\addtolength{\jot}{8pt}
\linespread{1.3}

\begin{document}
\noindent{\em Liam Hardiman\hfill{March 15, 2019} }
% Please give relevant information
\begin{center}
{\bf \Large 260B - Homework 2} %Replace N with the appropriate number
\vspace{0.2cm}
\hrule
\end{center}

\noindent\textbf{Problem 1. }Let $H$ be a complex separable Hilbert space and let $T_1$, $T_2\in \mcal{L}(H, H)$ be Hilbert-Schmidt operators. Let $T = T_2T_1$. Show that
\[
\tr\ T = \sum(Te_j, e_j)
\]
exists if $e_j$ is an orthonormal basis for $H$, and prov that the sum is independent of the choice of basis.
\begin{proof}
	That the trace exists for any fixed orthonormal basis $e_j$ follows from Cauchy-Schwartz and H\"older's inequality.
	\begin{align*}
		|\tr\ T| &= \left|\sum (Te_j, e_j)\right|\\
		&\leq \sum |(T_1e_j, T_2^*e_j)|\\
		&\leq \sum \|T_1e_j\|\cdot \|T_2^*e_j\|\\
		&\leq \left(\sum \|T_1e_j\|^2\right)^{1/2}\left(\sum \|T_2^*e_j\|^2\right)^{1/2}\\
		&= \|T_1\|_{HS}\cdot \|T_2\|_{HS}<\infty.
	\end{align*}
	On the last line we used the fact that $\|T_2^*\|_{HS} = \|T_2\|_{HS}$. Since the Hilbert-Schmidt norm is independent of the choice of basis, we have that the trace exists regardless of choice of basis.\\

	\noindent It remains to show that the actual value of the trace is basis-independent. Let $e_j$ and $f_k$ be two orthonormal bases of $H$. We use the parallelogram identity and the fact that the Hilbert-Schmidt norm is independent of basis
	\begin{align*}
		2\|T_1\|_{HS}^2 + 2\|T_2\|_{HS}^2 &= \|T_1+T_2\|_{HS}^2 + \|T_1-T_2\|_{HS}^2\\
		&= \left(\|T_1\|_{HS}^2 + \sum_j(T_1e_j, T_2e_j) + \sum_j(T_2e_j, T_1e_j) + \|T_2\|_{HS}^2\right)\\
		&+ \left(\|T_1\|_{HS}^2 - \sum_k(T_1f_k, T_2f_k) - \sum_j(T_2f_k, T_1f_k) + \|T_2\|_{HS}^2\right).
	\end{align*}
	After canceling the $\|T_1\|_{HS}$ and $\|T_2\|_{HS}$ terms from both sides, we see that the real parts of $\sum (Te_j, e_j)$ and $\sum (Tf_k, f_k)$ are equal. The same argument shows that their imaginary parts are equal as well, so the trace is basis-independent.

	% \noindent It remains to show that the actual value of the trace is basis-independent. Let $e_j$ and $f_k$ be two orthonormal bases for $H$. By Parseval we have
	% \begin{align*}
	% 	\sum (Te_j, e_j) &= \sum_j\left(\sum_k(Te_j, f_k)f_k, e_j\right)\\
	% 	&= \sum_j\sum_k(Te_j, f_k)(f_k, e_j).
	% \end{align*}
	% Now the summand is absolutely summable over $j$ and $k$ since
	% \begin{align*}
	% 	\sum_{j,k}|(Te_j, f_k)(f_k,e_j)| &\leq \sum_{j,k}|(T_1e_j, T_2^*f_k)|\\
	% 	&\leq \sum_{j,k}
	% \end{align*}
	% and
	% \begin{align*}
	% 	\sum (Tf_k, f_k) &= \sum_k\left(\sum_j(f_k, e_j)Te_j, f_k\right)\\
	% 	&= \sum_k\sum_j(f_k, e_j)(Te_j, f_k).
	% \end{align*}
	% Now if we can reverse the order of summation we will be done.

\end{proof}


\noindent\textbf{Problem 2. }When $a(x,\xi)\in \mcal{S}(\reals^n_x\times \reals^n_\xi)$, let us consider the Weyl quantization of $a$,
\[
a^w(x, D_x)u(x) = \frac{1}{(2\pi)^n}\int\int e^{i(x-y)\xi}a\left(\frac{x+y}{2}, \xi\right)u(y)\ dyd\xi,
\]
acting on the Schwartz space $\mcal{S}(\reals^n)$.
\begin{enumerate}[(a)]
	\item Show that $a^w(x, D_x)$ is symmetric on $\mcal{S}(\reals^n)$,
	\[
	(a^w(x, D_x)u, v)_{L^2} = (u, a^w(x, D_x)v)_{L^2},\quad u,v\in \mcal{S}(\reals^n)
	\]
	precisely when $a$ is real-valued.
	\begin{proof}
		We have that
		\begin{align*}
			(a^w(x, D_x)u, v)_{L^2} &= \frac{1}{(2\pi)^n}\int\int\int e^{i(x-y)\xi}a\left(\frac{x+y}{2}, \xi\right)u(y)\overline{v(x)}\ dyd\xi dx\\
			&= \frac{1}{(2\pi)^n}\int\int\int \overline{e^{i(y-x)\xi}\overline{a}\left(\frac{x+y}{2}, \xi\right)\overline{u(y)}v(x)}\ dyd\xi dx.
		\end{align*}
		Now $a$, $u$, an $v$ are all Schwartz functions in their respective variables, so their product is absolutely integrable. We can then use Fubini to switch the order of integration.
		\begin{align*}
			(a^w(x, D_x)u, v) &= \frac{1}{(2\pi)^n}\int\int\int \overline{e^{i(y-x)\xi}\overline{a}\left(\frac{x+y}{2}, \xi\right)\overline{u(y)}v(x)}\ dxd\xi dy\\
			&= \frac{1}{(2\pi)^n}\int u(y) \overline{\int\int e^{i(y-x)\xi}\overline{a}\left(\frac{x+y}{2}\xi\right)v(x)}\ dxd\xi dy\\
			&= (u, \overline{a}^w(x, D_x)v).
		\end{align*}
		If $a$ is real valued then we clearly have that $a^w(x, D_x)$ is symmetric. Conversely, if $a^w(x, D_x)$ is symmetric, then we must have that $a^w(x, D_x)u = \overline{a}^w(x, D_x)u$ for all $u\in \mcal{S}(\reals)$. Let's compute the difference:
		\begin{align*}
			[a^w(x, D_x) - \overline{a}^w(x, D_x)]u &= \frac{1}{(2\pi)^n}\int\int e^{i(x-y)\xi}\left[a\left(\frac{x+y}{2},\xi\right) - \overline{a}\left(\frac{x+y}{2},\xi\right)\right]u(x)\ dyd\xi dx\\
			&= 0.
		\end{align*}
		I want to say that this is the Fourier transform of another Fourier transform and then invoke the injectivity of the Fourier transform on $L^2$. That would then force the above integrand to vanish identically for all $u\in \mcal{S}(\reals)$, which would mean that $a = \overline{a}$, so $a$ would be real-valued. I'm not quite sure how to fill in the details.
	\end{proof}
	\item Suppose now that $a = a(\xi)$ is real-valued and depends only on the momentum variable $\xi$. Show that $a^w(D_x)$ is unitarily equivalent to a multiplication operator by $a(\xi)$. What is the spectrum of $a^w(D_x)$.
	\begin{proof}
		Let's compute $a^w(D_x)u$ for $u\in \mcal{S}(\reals)$. We have
		\begin{align*}
			a^w(D_x)u &= \frac{1}{(2\pi)^n}\int\int e^{i(x-y)\xi}a(\xi)u(y)\ dyd\xi\\
			&= \frac{1}{(2\pi)^n}\int e^{ix\xi}a(\xi)\int e^{-iy\xi}u(y)\ dy d\xi\\
			&= \frac{1}{(2\pi)^n}\int e^{ix\xi}a(\xi)\widehat{u}(\xi)\ d\xi\\
			&= \mcal{F}^{-1}(a\cdot \widehat{u})\\
			&= (\mcal{F}^{-1}M_a\mcal{F})u,
		\end{align*}
		where $M_a$ is the multiplication by $a$ operator. Since the Fourier transform is a unitary isomorphism $\mcal{S}(\reals)\to \mcal{S}(\reals)$, we have that $a^w(D_x)$ is unitarily equivalent to $M_a$.\\
		Now let's compute the spectrum. Since $a^w(D_x)$ is unitarily equivalent to multiplication by $a(\xi)$, it suffices to compute the spectrum of $M_a$. The resolvent set $\rho(M_a)$ is the set of complex $\lambda$ such that $(M_a-\lambda)$ is bijective onto $L^2(\reals)$. Now for any $\varphi \in \mcal{S}(\reals)$, we have that
		\[
		(M_a-\lambda)\varphi = a\varphi - \lambda\varphi.
		\]
		Since $a\in \mcal{S}(\reals)$, we have that $a\varphi$ is in $\mcal{S}(\reals)$ as well, as is $\lambda\varphi$. Then the image of $(M_a-\lambda)$ is contained in $\mcal{S}(\reals)$, so this operator cannot possibly biject onto $L^2(\reals)$ for any $\lambda$. The resolvent set is then empty, so the spectrum of $M_a$, and therefore $a^w(D_x)$, is all of $\complex$.
	\end{proof}
\end{enumerate}

\noindent\textbf{Problem 3. }Let us consider the Sturm-Liouville operator
\[
P = -\frac{d}{dt}\left(p(t)\frac{d}{dt}\right)+q(t),
\]
where $p\in C^1(\reals)$, $p>0$, and $q\in C(\reals)$, $q\geq 0$. Show that the operator $P$ equipped with the domain $C_0^\infty(\reals)$, is essentially self-adjoint on $L^2(\reals)$.
\begin{proof}
	$P$ is self-adjoint if and only if its closure, $\overline{P}$, is equal to its adjoint, $P^*$. We will show this by showing that $P+I$, equipped with the domain of $P$, is essentially self-adjoint. We claim that it suffices to show that the image of $C_0^\infty$ under $P+1$ is dense in $L^2(\reals)$. We will prove this claim after we have shown the density.\\

	\noindent We have the splitting $L^2(\reals) = \ker(P+I) \oplus \overline{\im(P+I)}$. If we can show that $P+I$ is injective then we will have shown that its image is dense in $L^2(\reals)$. Suppose that for $u\in L^2(\reals)$ we have that $(P+I)u = 0$ in the distributional sense. Then the real and imaginary parts of $(P+I)u$ vanish, so we can restrict our attention to real valued functions $u\in L^2(\reals)$.\\

	\noindent Since $u$ satisfies a second-order ODE it must have $C^2(\reals)$ regularity (why?). We claim that if $u$ is in both $C^2(\reals)$ and $L^2(\reals)$, and it satisfies $(P+I)u \geq 0$ then $u\geq 0$. To see this, suppose that $u(x_0)<0$ for some $x_0$. If this point is a local minimum of $u$ then $u'(x_0) = 0$ and $u''(x_0)>0$. From the definition of our operator $P$ and the fact that $p$ and $q$ are nonnegative we have
	\[
	-p(x_0)u''(x_0) = (P+1)u(x_0)-(q(x_0)+1)u(x_0) > 0.
	\]
	But this would force $u''(x_0)$ to be negative, contradicting the assumption that $x_0$ is a local minimum of $u$. We have then shown that if $u(x_0)<0$ then $x_0$ cannot be a local minimum.\\

	\noindent Now let $R>|x_0|$ and let $x_1$ be such that $u(x_1) = \min_{|x|\leq R}u(x)<0$. By our above discussion, $x_1$ must lie on the boundary of $[-R, R]$, or else $u$ would have a local minimum at a point where it assumes a negative value. Suppose $x_1 = R$ and let $x>R$. If $u(x)>u(R)$ then $u$ would have need to have a local minimum on $[R, x]$, again contradicting our above discussion. But then $u(x)\leq u(R)\leq u(x_0)<0$. This shows that $u$ is away from zero for $x$ sufficiently large, contradicting the assumption that $u\in L^2(\reals)$. A similar contradiction arises if $x_0 = -R$. We conclude that $u$ cannot assume negative values if $(P+I)u \geq 0$.\\

	\noindent If $(P+I)u = 0$ then $(P+I)u\geq 0$ and $(P+I)(-u)\geq 0$. By our previous discussion we have $u\geq 0$ and $-u\geq 0$, so $u = 0$. We have then shown that $(P+I)$ is injective, so its image must be dense.
\end{proof}

\noindent\textbf{Problem 4. }Let $T$ be a closed densely defined operator on a complex separable Hilbert space. Show that the operators $T^*T$ and $TT^*$ are self-adjoint, when equipped with their natural domains of definition.
\begin{proof}
	We'll show that $T^*T$ is self-adjoint. Applying this result to $T^*$ will show that $TT^*$ is self-adjoint too.\\

	\noindent We have that $V(G(T))^\perp = G(T^*)$ where $V:H\times H\to H\times H$ sends $(u, v)$ to $(v, -u)$ and $G(T)$ is the graph of $T$. Showing that $T$ is self-adjoint amounts to showing that $G(T) = V(G(T))^\perp$. This orthogonality statement says that $H\oplus H = G(T^*)\oplus V(G(T))$. So for any $w\in H$ there are unique $u\in \mcal{D}(T^*)$ and $v\in \mcal{D}(T)$ with
	\begin{equation}\label{oplus}
	H\oplus H\ni (0, w)= (u, T^*u) + V(v, Tv) = (u, T^*u) + (Tv, -v).
	\end{equation}
	Looking at each component, we have that $Tv = -u$ and $T^*u = w+v$. Substitution gives $T^*Tv = -(w+v)$, so $v$ is in $\mcal{D}(T^*T)$. Since $w$ was arbitrary, we also have that $T^*T+I$ is surjective. Suppose $(T^*T+I)x = (T^*T+I)y$. By the uniqueness of $u$ and $v$ in (\ref{oplus}), we must then have that $x = y$.\\

	\noindent Since $T^*T+I$ is bijective, it has an inverse $(T^*T+I)^{-1}: H\to \mcal{D}(T^*T)$. Given $x\in H$ we can write $x = (T^*T+I)y$ for some $y\in \mcal{D}(T^*T)$ by surjectivity. Since $T$ is closed we also have that $T^{**} = T$. In the following computation the parentheses denote inner products of elements in $H$.
	\begin{align*}
		((T^*T+I)^{-1}x, x) &= (y, (T^*T+I)y)\\
		&= (Ty, Ty) + \|y\|^2\\
		&= ((T^*T+I)y, y)\\
		&= (x, (T^*T+I)^{-1}x).
	\end{align*}
	Thus, $(T^*T+I)$ is symmetric with domain $H$, so it is self-adjoint. Now let's look at the graph of $T^*T+I$. Since $(T^*T+I)^{-1}$ is self-adjoint we have
	\begin{align*}
		V(G(T^*T+I))^\perp &= G(-(T^*T+I)^{-1})^\perp\\
		&= V(G(-(T^*T+I)^{-1}))\\
		&= G(T^*T+I),
	\end{align*}
	so $T^*T+I$ is self-adjoint. We claim that this forces $T^*T$ to be self-adjoint as well. Clearly $T^*T$ is symmetric if and only if $T^*T+I$ is symmetric, and $\mcal{D}(T^*T)= \mcal{D}((T^*T)^*)$ if and only if $\mcal{D}(T^*T+I) = \mcal{D}((T^*T+I)^*)$ follows from the Riesz representation theorem.
\end{proof}
	
\end{document}