\documentclass[12pt]{article}  
%%Read the manual for other options. 

\pagestyle{empty} %%Eliminates page numbers
%%\input rmb_macros
%%Collect your favorite macros in a 
%%separate file

%\input amssym.def
%\input amssym
%\input mssymb
%%Defines additional symbols



\usepackage{graphics}
\usepackage{amsmath,amssymb,amsthm, multicol}
\usepackage[pdftex]{graphicx}
\usepackage{epsf}
%%Use to include pictures. 

%\newcommand{\comment}[1]{}
%\newcommand{\sobolev}[2]{W^{#1,#2}}
%\newcommand{\sobolev}[2]{L^#2_#1}
%%Some examples of macros or new commands.

\addtolength{\oddsidemargin}{-.75in}
\addtolength{\evensidemargin}{-.75in}
\addtolength{\textwidth}{1.5in}
\addtolength{\topmargin}{-1in}
\addtolength{\textheight}{2.25in}
%%Set margins, defaults are ok. 

\begin{document}
\begin{center}
{\bf \Large Continued Fractions 2}
\vspace{0.2cm}
\hrule
\end{center}

\begin{enumerate}
	\item Let $c_1, c_2, \ldots, c_n$ be positive integers (except for maybe $c_1$). Define the sequences $p_n$, $q_n$ by
	\begin{align*}
		p_{-1}&=0 & p_0&=1 & p_n&= c_np_{n-1}+p_{n-2}\\
		q_{-1}&=1 & q_0&=0 & q_n&= c_nq_{n-1}+q_{n-2}.
	\end{align*}
	Prove that for any positive real number $x$,
	\[
	[c_1;c_2, \ldots, c_n, x] = \frac{xp_{n-1}+p_{n-2}}{xq_{n-1}+q_{n-2}}.
	\]

	\vfill

	\item Recall that the $n$-th convergent of $[c_1; c_2, \ldots]$ is defined to be $\frac{p_n}{q_n}$. Using the previous exercise, show that $[c_1; c_2, \ldots, c_n] = \frac{p_n}{q_n}$.
	\vfill
	\item Find the continued fraction expansions of $\sqrt{2}-1$ and $\frac{1}{\sqrt{3}}$. Compute the first three convergents.
	\vfill
	\item Given that two irrational numbers have identical convergents $p_1/q_1, p_2/q_2, \ldots$, up to $p_n/q_n$, prove that their continued fraction expansions are identical up to $c_n$.
	\vfill
	\item Evaluate the infinite continued fractions $[2;1, 1, \ldots]$, $[2;3, 1, 1, \ldots]$, and $[1;3, 1, 2, 1, 2, \ldots]$. Assume for now that the notion of an infinite continued fraction makes sense.
	\vfill
\end{enumerate}

\end{document}