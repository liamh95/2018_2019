\documentclass[12pt]{article}  
%%Read the manual for other options. 

\pagestyle{empty} %%Eliminates page numbers
%%\input rmb_macros
%%Collect your favorite macros in a 
%%separate file

%\input amssym.def
%\input amssym
%\input mssymb
%%Defines additional symbols



\usepackage{graphics}
\usepackage{amsmath,amssymb,amsthm, multicol}
\usepackage[pdftex]{graphicx}
\usepackage{epsf}
%%Use to include pictures. 
\newcommand{\integers}{\mathbb{Z}}

%\newcommand{\comment}[1]{}
%\newcommand{\sobolev}[2]{W^{#1,#2}}
%\newcommand{\sobolev}[2]{L^#2_#1}
%%Some examples of macros or new commands.

\addtolength{\oddsidemargin}{-.75in}
\addtolength{\evensidemargin}{-.75in}
\addtolength{\textwidth}{1.5in}
\addtolength{\topmargin}{-1in}
\addtolength{\textheight}{2.25in}
%%Set margins, defaults are ok. 

\begin{document}
\begin{flushleft} 
%%Paragraphs will not be indented in this 
%%environment
\centerline{\LARGE{Quiz 4}} 
\vspace{5 mm}
{Student ID Number:}\hfill  
%%\hfill forces following text 
%%to right margin
{Name \rule {2 in}{0.01in}}\\
Math 180B, 3PM
\\
%%gives a line of length 2in and 
%%thickness 0.01in
{Please justify all your answers}\hfill {May 16, 2019}
\\
{Please also write your full name on the back} 

\medskip
\end{flushleft}

In these problems we'll show that if $d\in \integers$ is not a perfect cube then $\sqrt[3]{d}$ is irrational.
\begin{enumerate}
	\item Suppose $\sqrt[3]{d} = a/b$ with $a,b$ nonzero integers and $d$ isn't a perfect cube. Let
	\[
	A = \begin{pmatrix}
		0 & 0 & d\\
		1 & 0 & 0\\
		0 & 1 & 0
	\end{pmatrix},\quad v = \begin{pmatrix}
		a^2\\
		ab\\
		b^2
	\end{pmatrix}.
	\]
	Compute the eigenvalues of $A$ and show that $v$ is an eigenvector of $A$. \textit{Hint: When showing that $v$ is an eigenvector, use the fact that $d = a^3/b^3$.}
	\vfill

	\item Let $\ell\in \integers$ be such that $\ell < \sqrt[3]{d}<\ell+1$. Show that
	\[
	(A-\ell I)^rv = (\sqrt[3]{d}-\ell)^rv
	\]
	for all $r\geq 1$. \textit{Hint: Show it for $r=1$ first using part (a).}
	\vfill
	\item Prove that $(A-\ell I)^rv$ is a vector with integer entries for all $r\geq 1$. What can you say about the size of $\sqrt[3]{d}-\ell$? Take the limit as $r\to \infty$ to derive a contradiction, i.e. no such $a$ and $b$ exist.
	\vfill
\end{enumerate}

%\vfill will divide page evenly
%use \begin{enumerate} environment for ordered lists
\end{document}