\documentclass[12pt]{article}  
%%Read the manual for other options. 

\pagestyle{empty} %%Eliminates page numbers
%%\input rmb_macros
%%Collect your favorite macros in a 
%%separate file

%\input amssym.def
%\input amssym
%\input mssymb
%%Defines additional symbols



\usepackage{graphics}
\usepackage{amsmath,amssymb,amsthm, multicol}
\usepackage[pdftex]{graphicx}
\usepackage{epsf}
%%Use to include pictures. 

%\newcommand{\comment}[1]{}
%\newcommand{\sobolev}[2]{W^{#1,#2}}
%\newcommand{\sobolev}[2]{L^#2_#1}
%%Some examples of macros or new commands.

\addtolength{\oddsidemargin}{-.75in}
\addtolength{\evensidemargin}{-.75in}
\addtolength{\textwidth}{1.5in}
\addtolength{\topmargin}{-1in}
\addtolength{\textheight}{2.25in}
%%Set margins, defaults are ok. 
\newcommand{\integers}{\mathbb{Z}}

\begin{document}
\begin{center}
{\bf \Large Pell's Equation 1}
\vspace{0.2cm}
\hrule
\end{center}

\begin{enumerate}
	\item Recall Pell's equation is given by
	\[
	x^2-dy^2 = N,
	\]
	where $d$ and $N$ are integers. Show that if $d<0$ then this equation has only finitely many solutions.
	\vfill
	\item Let $d$ be a squarefree integer. Recall that $\integers[\sqrt{d}] = \{a + b\sqrt{d}: a,b\in \integers\}$. Define the conjugate of $z = a+b\sqrt{d}$ to be $\overline{z} = a-b\sqrt{d}$ and the function $N: \integers[\sqrt{d}]\to \integers$ (the norm) on $z=a+b\sqrt{d}$ by
	\[
	N(z) = z\overline{z} = a^2 - db^2.
	\]
	\begin{enumerate}
		\item Show that $a_1+b_1\sqrt{d} = a_2+b_2\sqrt{d}$ if and only if $a_1=a_2$ and $b_1 = b_2$.
		\vfill
		\item Show that the norm is multiplicative, i.e. $N(zw) = N(z)N(w)$ for all $z,w\in \integers[\sqrt{d}]$.
		\vfill
		\item Show that solutions to Pell's equation correspond to elements of $\integers[\sqrt{d}]$ with norm 1.
		\vfill
		\item Show that $\alpha \in \integers[\sqrt{d}]$ is a unit if and only if it has norm $\pm 1$.
		\vfill
	\end{enumerate}
	\vfill
	\item Prove Brahmagupta's composition rule: if $x_1, x_2, y_1, y_2$ satisfy
	\[
	x_1^2-dy_1^2 = a,\quad x_2^2-dy_2^2 = b,
	\]
	then the composition
	\[
	(x_1, y_1)\cdot (x_2, y_2):= (x_1x_2+dy_1y_2, x_1y_2+y_1x_2)
	\]
	is a solution to
	\[
	x_3^2 -dy_3^2 = ab.
	\]
	\vfill

	\item Let $U^+$ be the set of all $\alpha \in \integers[\sqrt{d}]$, $d>0$ squarefree, with $N(\alpha) = 1$. Prove that $U^+$ is an infinite abelian group under multiplication. Furthermore, prove that it is generated by $\epsilon$ and $-1$, where $\epsilon$ is the smallest nontrivial element with $N(\epsilon) = 1$ and $\epsilon>1$. 
	\vfill\null\pagebreak

	\item \begin{enumerate}
		\item Let $\alpha$ be an irrational number and $n$ a positive integer. Show that there exist $p\in \integers$ and $q\in \{1, 2, \ldots, n\}$ such that $|\alpha - p/q| < \frac{1}{(n+1)q}$.
		\vfill
		\item If $\alpha$ is a real number show that there are infinitely many pairs of positive integers $(p, q)$ satisfying $|\alpha-\frac{p}{q}|<\frac{1}{q^2}$.
		\vfill
		\item Using the previous two parts of this exercise, conclude that Pell's equation has a nontrivial solution.
		\vfill
	\end{enumerate}
\end{enumerate}

\end{document}