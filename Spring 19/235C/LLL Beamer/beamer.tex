%% Please change the file name by replacing N with the apporpriate number
%% corresponding to the current homework and XX with your initials.
%% https://www.math.uci.edu/~gpatrick/jsOnline/hw1.html

\documentclass{beamer}
\usepackage{amssymb,amsfonts,color,graphicx,amsmath,enumerate,mathtools}
\usepackage{tikz} %This package offers the ability to draw pictures
\usepackage{amsthm}

\newcommand{\naturals}{\mathbb{N}}
\newcommand{\integers}{\mathbb{Z}}
\newcommand{\complex}{\mathbb{C}}
\newcommand{\reals}{\mathbb{R}}
\newcommand{\exreals}{\overline{\mathbb{R}}}
\newcommand{\mcal}[1]{\mathcal{#1}}
\newcommand{\mable}{measurable}
\newcommand{\quats}{\mathbb{H}}
\newcommand{\rationals}{\mathbb{Q}}
\newcommand{\norm}{\trianglelefteq}
\newcommand{\Aut}{\text{Aut}}
\newcommand{\disk}{\mathbb{D}}
\newcommand{\halfplane}{\mathbb{H}}
\newcommand{\Lp}[2]{\left\|{#1}\right\|_{L^{#2}}}
\newcommand{\supp}[1]{\text{supp}({#1})}
\newcommand{\Hom}[2]{\text{Hom}_{{#1}}({#2})}
\newcommand{\tr}{\text{tr}}
\newcommand{\field}[1]{\mathbb{F}_{{#1}}}
\newcommand{\Gal}[1]{\text{Gal}\left({#1}\right)}
\newcommand{\esssup}{\text{ess sup }}
\newcommand{\essinf}{\text{ess inf }}
\newcommand{\affine}{\mathbb{A}}
\DeclareMathOperator*{\argmin}{arg\,min}


\title{The LLL Algorithm}
\author{Liam Hardiman}
\usetheme{Frankfurt}


% Roadmap
% Motivation
%	good basis, bad basis
% LLL
%	Gram Schmidt
%	The algorithm
%		description
%		runtime
%	return to good basis bad basis
% Coppersmith
%	improvement
%	ROCA
%  

\AtBeginSection[]{
	\begin{frame}<beamer>
		\tableofcontents[currentsection]
	\end{frame}
}


\begin{document}
\maketitle

\section{Motivation}
\begin{frame}
	\frametitle{Two lattices}
	\begin{itemize}
		\item Recall that the \textbf{lattice}, $L$, generated by the linearly independent vectors $x_1, x_2, \ldots, x_n\in \reals^n$ is the $\integers$-span of these vectors:
		\[
		L = \{c_1x_1 + c_2x_2 + \cdots + c_nx_n: c_i \in \integers, 1\leq i \leq n\}.
		\]\pause

		\item Consider the lattices, $L$ and $M$, generated by the rows of the matrices $X$ and $Y$, respectively.	
		\[
		X = \begin{bmatrix*}[r]
			-168 & 602 & 58\\
			157 & -564 & -57\\
			594 & -2134 & -219\\
		\end{bmatrix*}, \quad
		Y = \begin{bmatrix*}[r]
			-6 & 6 & -4\\
			9 & 4 & 1\\
			-1 & 8 & 6\\
		\end{bmatrix*}.
		\]
	\end{itemize}
\end{frame}

\begin{frame}
	\frametitle{Two lattices}
	\begin{itemize}
		\item Each row of $X$ is an integer linear combination of the rows of $Y$, so $L\subseteq M$:\pause
		\begin{align*}
		\begin{bmatrix*}[r]
			-168\\602\\58
		\end{bmatrix*}^T &= 14 \begin{bmatrix*}[r]
			4\\2\\-9
		\end{bmatrix*}^T + 50 \begin{bmatrix*}[r]
			-1\\8\\6
		\end{bmatrix*}^T -29 \begin{bmatrix*}[r]
			6\\ -6\\ 4
		\end{bmatrix*}^T,\\	
		\begin{bmatrix*}[r]
			157\\ -564\\ -57
		\end{bmatrix*}^T &= -13 \begin{bmatrix*}[r]
			4\\2\\-9
		\end{bmatrix*}^T - 47 \begin{bmatrix*}[r]
			-1\\8\\6
		\end{bmatrix*}^T + 26 \begin{bmatrix*}[r]
			6\\-6\\4
		\end{bmatrix*}^T,\\
		\begin{bmatrix*}[r]
			594\\-2134\\-219
		\end{bmatrix*}&= -49 \begin{bmatrix*}[r]
			4\\2\\-9
		\end{bmatrix*}^T - 178 \begin{bmatrix*}[r]
			-1\\8\\6
		\end{bmatrix*} + 102 \begin{bmatrix*}[r]
			6\\-6\\4
		\end{bmatrix*}^T.
		\end{align*}
	\end{itemize}
\end{frame}

\begin{frame}
	\frametitle{Two lattices}
	\begin{itemize}
		\item In particular, we have the matrix equation\\
	\end{itemize}
	\begin{align*}
	UY&= X,\\
	\begin{bmatrix*}[r]
		14 & 50 & -29\\
		-13 & -47 & 27\\
		-49 & -178 & 102
	\end{bmatrix*}
	\begin{bmatrix*}[r]
		4 & 2 & -9\\
		-1 & 8 & -6\\
		6 & -6 & 4
	\end{bmatrix*}&=
	\begin{bmatrix*}[r]
		-168 & 602 & 58\\
		157 & -564 & -57\\
		594 & -2134 & -219
	\end{bmatrix*}.
	\end{align*}\pause

	\begin{itemize}
		\item $\det U = -1$, so $U^{-1}$ is an integer matrix as well. This gives us another matrix equation, $Y = U^{-1}X$.\pause
		\item Since the entries of $U^{-1}$ are integers, this equation expresses the rows of $Y$ as integer linear combinations of the rows of $X$, so $M\subseteq L$.
	\end{itemize}
\end{frame}

\begin{frame}
	\frametitle{Two lattices}
	\begin{itemize}
		\item Even though the rows of $X$ and $Y$ generate the same lattice, something about the $Y$-basis ``feels'' nicer.
		\[
		X = \begin{bmatrix*}[r]
			-168 & 602 & 58\\
			157 & -564 & -57\\
			594 & -2134 & -219\\
		\end{bmatrix*}, \quad
		Y = \begin{bmatrix*}[r]
			-6 & 6 & -4\\
			9 & 4 & 1\\
			-1 & 8 & 6\\
		\end{bmatrix*}.\pause
		\]
		\item Two qualities that make a basis desirable are:
		\begin{itemize}
			\item Length: how long are the basis vectors?
			\item Orthogonality: are the basis vectors nearly orthogonal to each other?
		\end{itemize}
	\end{itemize}
\end{frame}

\begin{frame}
	\frametitle{What makes a basis ``nice''?}
	\begin{itemize}
		\item Suppose $v_1, v_2, \ldots, v_n\in \reals^n$ are pairwise orthogonal.\pause
		\item If $x = c_1v_1+c_2v_2 + \cdots + c_nv_n$, $c_i\in \integers$, is in the lattice $L$ generated by $v_1, \ldots, v_n$ then
		\[
		|x|^2 = c_1^2|v_1|^2 + c_2^2|v_2|^2 + \cdots + c_n^2|v_n|^2.\pause
		\]
		\item This completely solves the shortest vector problem (SVP) since
		\[
		\argmin_{x\in L}|x| = \argmin_{x\in \{\pm v_1, \pm v_2, \ldots, \pm v_n\}}|x|.
		\]
	\end{itemize}
\end{frame}

\begin{frame}
	\frametitle{What makes a basis ``nice''?}
	\begin{itemize}
		\item Say we want to find a vector in $L$ that is closest to
		\[
		x = t_1v_1 + t_2v_2 + \cdots + t_nv_n,
		\]
		where the $t_i$ are \textit{real} numbers.\pause
		\item If $y = c_1v_1+c_2v_2 + \cdots + c_nv_n$, $c_i\in \integers$, is any vector in $L$ then by the orthogonality of the $v_i$ we have
		\[
		|x-y|^2 = (t_1-c_1)^2|v_1|^2 + (t_2-c_2)^2|v_2|^2 + \cdots + (t_n-c_n)^2|v_n|^2.\pause
		\]
		\item If we take $c_i$ to be the closest integer to $t_i$ then we solve the closest vector problem (CVP).
	\end{itemize}
\end{frame}

\section{Setup}
\begin{frame}
	\frametitle{Gram-Schmidt}
		\begin{definition}
			Let $x_1, \ldots, x_m \in \reals^n$ be a basis for a nonzero subspace, $H$. The \textbf{Gram-Schmidt process} produces an orthogonal basis for $H$:
			\begin{align*}
				x_1^* &= x_1\\
				x_2^* &= x_2 - \frac{x_2\cdot x_1^*}{x_1^*\cdot x_1^*}x_1^*\\
				x_3^* &= x_3 - \frac{x_3\cdot x_1^*}{x_1^*\cdot x_1^*}x_1^* - \frac{x_3\cdot x_2^*}{x_2^*\cdot x_2^*}x_2^*\\
				\vdots\\
				x_m^* &= x_m - \frac{x_m\cdot x_1^*}{x_1^*\cdot x_1^*}x_1^* - \cdots - \frac{x_m\cdot x_{m-1}^*}{x_{m-1}^*\cdot x_{m-1}^*}x_{m-1}^*.
			\end{align*}
			We call $\{x_1^*, \ldots, x_m^*\}$ the \textbf{Gram-Schmidt orthogonalization (GSO)} of $\{x_1, \ldots, x_m\}$.
		\end{definition}
\end{frame}

\begin{frame}
	\frametitle{Gram-Schmidt}
	\begin{itemize}
		\item 
	\end{itemize}
\end{frame}

\end{document}