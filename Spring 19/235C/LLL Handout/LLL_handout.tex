%% Please change the file name by replacing N with the apporpriate number
%% corresponding to the current homework and XX with your initials.
%% https://www.math.uci.edu/~gpatrick/jsOnline/hw1.html

\documentclass[11pt,letterpaper]{article}
\usepackage{amssymb,amsfonts,color,graphicx,amsmath,enumerate}
\usepackage{tikz} %This package offers the ability to draw pictures
\usepackage{amsthm,mathtools}

\newcommand{\naturals}{\mathbb{N}}
\newcommand{\integers}{\mathbb{Z}}
\newcommand{\complex}{\mathbb{C}}
\newcommand{\reals}{\mathbb{R}}
\newcommand{\exreals}{\overline{\mathbb{R}}}
\newcommand{\mcal}[1]{\mathcal{#1}}
\newcommand{\mable}{measurable}
\newcommand{\quats}{\mathbb{H}}
\newcommand{\rationals}{\mathbb{Q}}
\newcommand{\norm}{\trianglelefteq}
\newcommand{\Aut}{\text{Aut}}
\newcommand{\disk}{\mathbb{D}}
\newcommand{\halfplane}{\mathbb{H}}
\newcommand{\Lp}[2]{\left\|{#1}\right\|_{L^{#2}}}
\newcommand{\supp}[1]{\text{supp}({#1})}
\newcommand{\Hom}[2]{\text{Hom}_{{#1}}({#2})}
\newcommand{\tr}{\text{tr}}
\newcommand{\field}[1]{\mathbb{F}_{{#1}}}
\newcommand{\Gal}[1]{\text{Gal}\left({#1}\right)}
\newcommand{\esssup}{\text{ess sup }}
\newcommand{\essinf}{\text{ess inf }}
\newcommand{\affine}{\mathbb{A}}

\newenvironment{solution}
{\begin{proof}[Solution]}
{\end{proof}}

\newtheorem{theorem}{Theorem}[section]
\newtheorem{corollary}{Corollary}[theorem]
\newtheorem{lemma}{Lemma}[section]

\theoremstyle{definition}
\newtheorem{definition}{Definition}[section]

\voffset=-3cm
\hoffset=-2.25cm
\textheight=24cm
\textwidth=17.25cm
\addtolength{\jot}{8pt}
\linespread{1.3}

\begin{document}
\noindent{\em Liam Hardiman\hfill{MATH 235C} }
% Please give relevant information
\begin{center}
{\bf \Large The LLL Algorithm} %Replace N with the appropriate number
\vspace{0.2cm}
\hrule
\end{center}
%LLL
%	Gram-Schmidt
%	
%Coppersmith
\section{Motivation}
The rows of the following matrix form a basis for a lattice $L$ in $\reals^4$:
\[
X = \begin{bmatrix*}[r]
	-2 & 7 & 7 & -5\\
	3 & -2 & 6 & -1\\
	2 & -8 & -9 & -7\\
	8 & -9 & 6 & -4
\end{bmatrix*}.
\]
One can check that the rows of the following matrix also form a basis for the same lattice:
\[
Y = \begin{bmatrix*}[r]
	-13071 & -5406 & -9282 & -2303\\
	-20726 & -8571 & -14772 & -3651\\
	-2867 & -1186 & -2043 & -505\\
	-14338 & -5936 & -10216 & -2525
\end{bmatrix*}.
\]
Intuitively, the rows of $Y$ seem to be a ``worse'' basis for $L$ than those of $X$. Here we make precise the notion of a ``nice'' basis and introduce a polynomial time algorithm that transforms a ``bad'' basis into a ``good'' one. 

\section{Lattices in $\reals^n$}
\begin{definition}
	Let $n\geq 1$ and let $x_1, \ldots, x_n$ be a basis of $\reals^n$. The \textbf{lattice} with dimension $n$ and basis $x_1, \ldots, x_n$ is the set $L$ of all linear combinations of the basis vectors with integral coefficients:
	\[
	L = \integers x_1 + \integers x_2 + \cdots + \integers x_n = \left\{\sum_{i=1}^na_ix_i: a_1, \ldots, a_n\in \integers\right\}.
	\]
	Let $X$ be the matrix whose rows are the basis vectors $x_1, \ldots, x_n$. The \textbf{determinant} of the lattice $L$ is
	\[
	\det(L) = |\det(X)|.
	\]
\end{definition}

\begin{lemma}
	Let $x_1, \ldots, x_n$ be a basis for the lattice $L\subset \reals^n$ and let $y_1, \ldots, y_n$ be a collection of $n$ vectors in $L$. Let $X$ and $Y$ be the $n\times n$ matrices with rows $x_1, \ldots, x_n$ and $y_1, \ldots, y_n$, respectively. Then $y_1, \ldots, y_n$ is a basis for $L$ if and only if there is an $n\times n$ matrix $C$ with integer entries and $\det(C) = \pm 1$ such that $Y = CX$.
\end{lemma}



\end{document}