\documentclass[12pt]{article}  
%%Read the manual for other options. 

\pagestyle{empty} %%Eliminates page numbers
%%\input rmb_macros
%%Collect your favorite macros in a 
%%separate file

%\input amssym.def
%\input amssym
%\input mssymb
%%Defines additional symbols



\usepackage{graphics}
\usepackage{amsmath,amssymb,amsthm, multicol,mathtools}
\usepackage[pdftex]{graphicx}
\usepackage{epsf}
%%Use to include pictures. 

%\newcommand{\comment}[1]{}
%\newcommand{\sobolev}[2]{W^{#1,#2}}
%\newcommand{\sobolev}[2]{L^#2_#1}
%%Some examples of macros or new commands.

\addtolength{\oddsidemargin}{-.75in}
\addtolength{\evensidemargin}{-.75in}
\addtolength{\textwidth}{1.5in}
\addtolength{\topmargin}{-1in}
\addtolength{\textheight}{2.25in}
%%Set margins, defaults are ok.
\newenvironment{solution}
{\begin{proof}[Solution]}
{\end{proof}}

\begin{document}
\begin{flushleft} 
%%Paragraphs will not be indented in this 
%%environment
\centerline{\LARGE{Quiz 7}} 
\vspace{5 mm}
{Student ID Number:}\hfill  
%%\hfill forces following text 
%%to right margin
{Name \rule {2 in}{0.01in}}\\
Math 3A, 6PM
\\
%%gives a line of length 2in and 
%%thickness 0.01in
{Please justify all your answers}\hfill {December 6, 2018}
\\
{Please also write your full name on the back} 

\medskip
\end{flushleft}

\begin{enumerate}
	\item Is
	\[
	\left\{\begin{bmatrix*}[r]
		1\\1\\1
	\end{bmatrix*}, \begin{bmatrix*}[r]
		1\\-1\\0
	\end{bmatrix*}, \begin{bmatrix*}[r]
		1\\1\\-2
	\end{bmatrix*}\right\}
	\]
	an orthogonal basis for $\mathbb{R}^3$?
	\begin{solution}
		Yes it is. Call this set $\{v_1, v_2, v_3\}$.
		\begin{align*}
			v_1\cdot v_2 &= 1(1) + 1(-1) + 1(0) = 0\\
			v_1\cdot v_3 &= 1(1) + 1(1) + 1(-2) = 0\\
			v_2\cdot v_3 &= 1(1) + (-1)1 + 0(-2) = 0
		\end{align*}
		Since all the pairwise dot products are zero the set is orthogonal. An orthogonal set is linearly independent. Since we have three linearly independent vectors in $\mathbb{R}^3$, we have a basis.
	\end{solution}

	\vfill

	\item True or False? Explain.
	\begin{enumerate}
		\item Let $u,v\in \mathbb{R}^n$. If $u$ is orthogonal to $v$ then $v$ is orthogonal to $u$.
		\begin{solution}
			True. $u$ is orthogonal to $v$ means that $u\cdot v = 0$. Since $v\cdot u = u\cdot v = 0$, we have that $v$ is orthogonal to $u$ as well.
		\end{solution}

		\vfill

		\item Let $A$ be an $n\times n$ matrix whose columns form an orthonormal basis for $\mathbb{R}^n$. Then $A^TA = I$.
		\begin{solution}
			Say $A$ has columns $v_1, \cdots, v_n$. Let's compute the product $A^TA$.
			\[
			A^TA = \begin{bmatrix}
				\text{---} & v_1 & \text{---}\\
				\text{---} & v_2 & \text{---}\\
				&\vdots&\\
				\text{---} & v_n & \text{---}
			\end{bmatrix}\begin{bmatrix}
				\mid &  \mid & & \mid\\
				v_1 & v_2 & \cdots &v_n\\
				\mid & \mid & &\mid
			\end{bmatrix}=
			\begin{bmatrix}
				v_1\cdot v_1 & v_1\cdot v_2 & \cdots & v_1\cdot v_n\\
				v_2\cdot v_1 & v_2\cdot v_2 & \cdots & v_2\cdot v_n\\
				\vdots &&\ddots & \vdots\\
				v_n\cdot v_1 & v_n\cdot v_2 & \cdots & v_n\cdot v_n
			\end{bmatrix}.
			\]
			Since the columns of $A$ are orthonormal we have that $v_i\cdot v_j = 0$ if $i\neq j$ (orthogonal) and $v_i\cdot v_i = 1$ (normal). Consequently, the diagonal entries in the above product are 1 and every other entry is zero. We then have that $A^TA = I$, the identity matrix.
		\end{solution}
		\vfill
	\end{enumerate}
\end{enumerate}


%\vfill will divide page evenly
%use \begin{enumerate} environment for ordered lists
\end{document}