\documentclass[12pt]{article}  
%%Read the manual for other options. 

\pagestyle{empty} %%Eliminates page numbers
%%\input rmb_macros
%%Collect your favorite macros in a 
%%separate file

%\input amssym.def
%\input amssym
%\input mssymb
%%Defines additional symbols



\usepackage{graphics}
\usepackage{amsmath,amssymb,amsthm, multicol, systeme}
\usepackage[pdftex]{graphicx}
\usepackage{epsf}
%%Use to include pictures. 

%\newcommand{\comment}[1]{}
%\newcommand{\sobolev}[2]{W^{#1,#2}}
%\newcommand{\sobolev}[2]{L^#2_#1}
%%Some examples of macros or new commands.

\addtolength{\oddsidemargin}{-.75in}
\addtolength{\evensidemargin}{-.75in}
\addtolength{\textwidth}{1.5in}
\addtolength{\topmargin}{-1in}
\addtolength{\textheight}{2.25in}
\sysdelim..
%%Set margins, defaults are ok. 







%Section 1.4: 1,3,4,5,8,9,13,15,17,21,23,29,30.

%Section 1.5: 1,4,5,9,12,14,17,19,22,24,25

\begin{document}
\begin{center}
{\bf \Large Matrix Equations and Solution Sets}
\vspace{0.2cm}
\hrule
\end{center}

\begin{multicols*}{2}
	\begin{enumerate}
		\item Compute the matrix product. If the product is undefined, explain why.
		\begin{enumerate}
			\item \[
			\begin{bmatrix}
				-1 & 6\\
				5 & -3
			\end{bmatrix}\begin{bmatrix}
				4\\-7
			\end{bmatrix}
			\]

			\vfill

			\item \[
			\begin{bmatrix}
				5 & 5 & -1\\
				1 & -7 & 0
			\end{bmatrix}
			\begin{bmatrix}
				12\\4
			\end{bmatrix}
			\]

			\vfill
		\end{enumerate}

		\item Write the system as a vector equation and then as a matrix equation.
		\begin{enumerate}
			\item \[
			\begin{array}{rcrcr}
				2x & + & 5y & = & -10\\
				-2x & + & 4y & = & 0
			\end{array}
			\]
			\vfill

			\item \[
			\begin{array}{rcrcrcr}
				x & - & 7y & &&=&12\\
				-2x & + & 5y & + & 13z & =& 0\\
				&&6y & - & z & = & -1
			\end{array}
			\]
			\vfill
		\end{enumerate}

		\item Let $u = \begin{bmatrix}
			4\\-1\\4
		\end{bmatrix}$ and $A = \begin{bmatrix}
			2 & 5 & -1\\
			0 & 1 & -1\\
			1 & 2 & 0
		\end{bmatrix}$. Is $u$ in the subset of $\mathbb{R}^3$ spanned by the columns of $A$? What does this say about the solution set for the equation $Ax = u$?
		\vfill\null\columnbreak

		\item Let $A = \begin{bmatrix}
			1 & -2 & -1\\
			-2 & 2 & 0\\
			4 & -1 & 3
		\end{bmatrix}$ and $b = \begin{bmatrix}
			b_1\\b_2\\b_3
		\end{bmatrix}$.
		Show that the equation $Ax=b$ does not have a solution for all possible $b$. Describe the set of all $b$ for which $Ax=b$ \textit{does} have a solution.
		\vfill

		\item Write the solution set of the given homogeneous system in parametric vector form.
		\[
		\begin{array}{rcrcrcr}
			x_1 & + & 2x_2 & - & 3x_3 &=&0\\
			2x_1 & + & x_2 & - & 3x_3 & = & 0\\
			-x_1 & + & x_2 & && = & 0
		\end{array}
		\]
		\vfill

		\item Write the solution set to $Ax=0$ in parametric vector form where $A$ is given by
		\[
		A = \begin{bmatrix}
			1 & 2 & 0 & -1\\
			-2 & -3 & 4 & 5\\
			2 & 4 & 0 & -2
		\end{bmatrix}.
		\]
		\vfill

		\item Describe the solutions of the following system in parametric vector form. Give a geometric description of the solution set.
		\[
		\begin{array}{rcrcrcrcr}
			x_1 & + & 3x_2 & - & 4x_3 & + & 4x_4 & = & 4\\
			x_1 & + & 4x_2 & - & 7x_3 & +&6x_4 & = & 3
		\end{array}
		\]
		\vfill\null\pagebreak

		\item Find a parametric equation of the line $\ell$ through $p=\begin{bmatrix}
			3\\-3
		\end{bmatrix}$ and $q=\begin{bmatrix}
			4\\1
		\end{bmatrix}$.
		\vfill

		\item Suppose that the equation $Ax=b$ is consistent for some $b$, and let $p$ be a solution. Prove that the solution set of $Ax=b$ is the set of all vectors of the form $w=p+v_h$, where $v_h$ is any solution of the homogeneous equation $Ax=0$.
		\vfill

		\item (Bonus) Suppose $A$ is a $3\times 3$ matrix and $b$ is a vector in $\mathbb{R}^3$ such that the equation $Ax=b$ does \textit{not} have a solution. Does there exist a vector $y$ in $\mathbb{R}^3$ such that the equation $Ax=y$ has a \textit{unique} solution?
		\vfill
	\end{enumerate}
\end{multicols*}

\end{document}