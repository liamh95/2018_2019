\documentclass[12pt]{article}  
%%Read the manual for other options. 

\pagestyle{empty} %%Eliminates page numbers
%%\input rmb_macros
%%Collect your favorite macros in a 
%%separate file

%\input amssym.def
%\input amssym
%\input mssymb
%%Defines additional symbols



\usepackage{graphics}
\usepackage{amsmath,amssymb,amsthm, multicol, mathtools}
\usepackage[pdftex]{graphicx}
\usepackage{epsf}
%%Use to include pictures. 

%\newcommand{\comment}[1]{}
%\newcommand{\sobolev}[2]{W^{#1,#2}}
%\newcommand{\sobolev}[2]{L^#2_#1}
%%Some examples of macros or new commands.

\addtolength{\oddsidemargin}{-.75in}
\addtolength{\evensidemargin}{-.75in}
\addtolength{\textwidth}{1.5in}
\addtolength{\topmargin}{-1in}
\addtolength{\textheight}{2.25in}
%%Set margins, defaults are ok. 


%Section 2.8: 1,2,5,7,9,11,12,13,16,17,19,21,23,24,30

\begin{document}
\begin{center}
{\bf \Large Subspaces}
\vspace{0.2cm}
\hrule
\end{center}

\begin{multicols*}{2}
	\begin{enumerate}
		\item Let $v_1 = \begin{bmatrix*}[r]
			1\\-3\\2\\3
		\end{bmatrix*}$, $v_2 = \begin{bmatrix*}[r]
			4\\-4\\5\\7
		\end{bmatrix*}$, $v_3 = \begin{bmatrix*}[r]
			5\\-3\\6\\5
		\end{bmatrix*}$, and $u = \begin{bmatrix*}[r]
			-1\\-7\\-1\\2
		\end{bmatrix*}$. Determine if $u$ is in the subspace of $\mathbb{R}^4$ generated by $\{v_1, v_2, v_3\}$.

		\vfill

		\item Let $u = \begin{bmatrix*}[r]
			-5\\5\\3
		\end{bmatrix*}$ and $A = \begin{bmatrix*}[r]
			-2 & -2 & 0\\
			0 & 3 & -5\\
			6 & 3 & 5
		\end{bmatrix*}$. Is $u$ in Nul $A$?

		\vfill

		\item Let $A = \begin{bmatrix*}[r]
			1 & 2 & 3\\
			4 & 5 & 7\\
			-5 & -1 & 0\\
			2 & 7 & 11\\
			3 & 3 & 4
		\end{bmatrix*}$. Find a nonzero vector in Nul $A$ and a nonzero vector in Col $A$.

		\vfill

		\item Do $\begin{bmatrix*}[r]
			4\\-2
		\end{bmatrix*}$ and $\begin{bmatrix*}[r]
			16\\-3
		\end{bmatrix*}$  form a basis for $\mathbb{R}^2$?

		\vfill

		\item Do $\begin{bmatrix*}[r]
			1\\-3\\4
		\end{bmatrix*}$, $\begin{bmatrix*}
			-1\\2\\2
		\end{bmatrix*}$, and $\begin{bmatrix*}[r]
			1\\-4\\10
		\end{bmatrix*}$ form a basis for $\mathbb{R}^3$? How about just the first two vectors?

		\vfill\null\columnbreak

		\item True or false?
		\begin{enumerate}
			\item A subset $H$ of $\mathbb{R}^n$ is a subspace if the zero vector is in $H$.
			\vfill
			\item Let $H$ be a subspace of $\mathbb{R}^n$. If $x$ is in $H$, and $y$ is in $\mathbb{R}^n$, then $x+y$ is in $H$.
			\vfill
			\item The solution set to $Ax=b$, where $A$ is an $m\times n$ matrix, forms a subspace of $\mathbb{R}^n$.
		\end{enumerate}
		\vfill

		\item Find a basis for the column space and null space of the matrix.
		\[
		A = \begin{bmatrix*}[r]
			3 & 4 & 0 & 7\\
			1 & -5 & 2 & -2\\
			-1 & 4 & 0 & 3\\
			1 & -1 & 2 & 2
		\end{bmatrix*}.
		\]

		\vfill

		\item Suppose $F$ is a $5\times 5$ matrix whose column space is not equal to $\mathbb{R}^5$. What can be said about $F$'s nullspace?

		\vfill

		\item What can be said about the shape of an $m\times n$ matrix $A$ when the columns of $A$ form a basis for $\mathbb{R}^m$?

		\vfill

		\item If $B$ is a $6\times 6$ matrix and Nul $B$ is not the zero subspace, what can be said about Col $B$?

		\vfill
	\end{enumerate}
\end{multicols*}

\end{document}