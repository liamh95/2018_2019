\documentclass[12pt]{article}  
%%Read the manual for other options. 

\pagestyle{empty} %%Eliminates page numbers
%%\input rmb_macros
%%Collect your favorite macros in a 
%%separate file

%\input amssym.def
%\input amssym
%\input mssymb
%%Defines additional symbols



\usepackage{graphics}
\usepackage{amsmath,amssymb,amsthm, multicol,mathtools,enumerate}
\usepackage[pdftex]{graphicx}
\usepackage{epsf}
%%Use to include pictures. 

%\newcommand{\comment}[1]{}
%\newcommand{\sobolev}[2]{W^{#1,#2}}
%\newcommand{\sobolev}[2]{L^#2_#1}
%%Some examples of macros or new commands.

\addtolength{\oddsidemargin}{-.75in}
\addtolength{\evensidemargin}{-.75in}
\addtolength{\textwidth}{1.5in}
\addtolength{\topmargin}{-1in}
\addtolength{\textheight}{2.25in}
%%Set margins, defaults are ok. 

\begin{document}
\begin{center}
{\bf \Large Dimension and Determinants}
\vspace{0.2cm}
\hrule
\end{center}

%Section 2.9:1,2,3,4,7,9,10,12,13,15,17,18,20,25
%Section 3.1: 2,3,4,5,10,12,14,15,19,21,22,25,27,29,31,32
\begin{multicols*}{2}
	\begin{enumerate}
		\item The vector $x$ is in a subspace $H$ with a basis $\mathcal{B} = \{b_1, b_2\}$. Find the $\mathcal{B}$-coordinate vector of $x$.
		\begin{enumerate}[(a)]
			\item $b_1 = \begin{bmatrix*}[r]
				1\\4\\-3
			\end{bmatrix*}$, $b_2 = \begin{bmatrix*}[r]
				-2\\-7\\5
			\end{bmatrix*}$, $x=\begin{bmatrix*}[r]
				2\\9\\-7
			\end{bmatrix*}$.
			\vfill

			\item $b_1 = \begin{bmatrix*}[r]
				-3\\2\\-4
			\end{bmatrix*}$, $b_2= \begin{bmatrix*}[r]
				7\\-3\\5
			\end{bmatrix*}$, $x=\begin{bmatrix*}[r]
				5\\0\\-2
			\end{bmatrix*}$.
		\end{enumerate}
		\vfill
		\item We're given a matrix $A$ and an echelon form of $A$. Find bases for Col$(A)$ and Nul$(A)$, and then state the dimension of these subspaces.
		\begin{align*}
		A &= \begin{bmatrix*}[r]
			2 & 4 & -5 & 2 & -3\\
			3 & 6 & -8 & 3 & -5\\
			0 & 0 & 9 & 0 & 9\\
			-3 & -6 & -7 & -3 & -10
		\end{bmatrix*} \\&\sim \begin{bmatrix*}[r]
			1 & 2 & -5 & 1 & -4\\
			0 & 0 & 5 & 0 & 5\\
			0 & 0 & 0 & 0 & 0\\
			0 & 0 & 0 & 0 & 0
		\end{bmatrix*}.
		\end{align*}
		\vfill

		\item Find a basis for the subspace spanned by the given vectors. What is the dimension of the subspace?
		\begin{gather*}
			\begin{bmatrix*}[r]
				1\\-1\\-2\\3
			\end{bmatrix*},\ \begin{bmatrix*}[r]
				2\\-3\\-1\\4
			\end{bmatrix*},\ \begin{bmatrix*}[r]
				0\\-1\\3\\-2
			\end{bmatrix*}\\
			\begin{bmatrix*}[r]
				-1\\4\\-7\\7
			\end{bmatrix*},\ \begin{bmatrix*}[r]
				3\\-7\\6\\-9
			\end{bmatrix*}.
		\end{gather*}
		\vfill\null\columnbreak

		\item Suppose a $4\times 7$ matrix $A$ has three pivot columns. Is Col$(A) = \mathbb{R}^3$? What is the dimension of Nul$(A)$? Explain
		\vfill

		\item If the subspace of all solutions of $Ax=0$ has a basis consisting of three vectors and if $A$ is a $5\times 7$ matrix, what is the rank of $A$?

		\vfill

		\item Construct a $3\times 4$ matrix with rank 1.

		\vfill

		\item If a $9\times 8$ matrix has rank 7, then what is the dimension of the solution space to $Ax=0$?

		\vfill\null\pagebreak

		%Section 3.1: 2,3,4,5,10,12,14,15,19,21,22,25,27,29,31,32
		\item Compute the determinant using the cofactor expansion in two different ways. Try to use as few computations as possible. For the $3\times 3$s try the upward and downward product rule (sometimes called the rule of Sarrus).
		\begin{gather*}
		\text{(a)\ }\begin{vmatrix*}[r]
			3 & 0 & 4\\
			2 & 3 & 2\\
			0 & 5 & -1
		\end{vmatrix*}\qquad \text{(b)\ }
		\begin{vmatrix*}[r]
			5 & -2 & 4\\
			0 & 3 & -5\\
			2 & -4 & 7
		\end{vmatrix*}\\
		\text{(c)\ }\begin{vmatrix*}[r]
			6 & 0 & 0 & 5\\
			1 & 7 & 2 & -5\\
			2 & 0 & 0 & 0\\
			8 & 3 & 1 & 8
		\end{vmatrix*}\quad \text{(d)\ }\begin{vmatrix*}[r]
			3 & 5 & -8 & 4\\
			0 & -2 & 3 & -7\\
			0 & 0 & 1 & 5\\
			0 & 0 & 0 & 2
		\end{vmatrix*}
		\end{gather*}

		\vfill

		\item Here you're given a matrix and one obtained from it by a row operation. How does this affect its determinant?
		\begin{enumerate}
		\item \[
		\begin{bmatrix*}[r]
			a & b\\
			c & d
		\end{bmatrix*}\sim \begin{bmatrix*}[r]
			a & b\\
			kc & kd
		\end{bmatrix*}
		\]
		\vfill

		\item \[
		\begin{bmatrix*}[r]
			1 & 1 & 1\\
			-3 & 8 & -4\\
			2 & -3 & 2
		\end{bmatrix*}\sim \begin{bmatrix*}[r]
			k & k & k\\
			-3 & 8 & -4\\
			2 & -3 & 2
		\end{bmatrix*}
		\]
		\vfill

		\item \[
		\begin{bmatrix*}[r]
			a & b & c\\
			3 & 2 & 2\\
			6 & 5 & 6
		\end{bmatrix*}\sim \begin{bmatrix*}[r]
			3 & 2 & 2\\
			a & b & c\\
			6 & 5 & 6
		\end{bmatrix*}
		\]
		\vfill
		\end{enumerate}

		\item Let $A = \begin{bmatrix*}[r]
			3 & 1\\
			4 & 2
		\end{bmatrix*}$. Write $5A$. Is $\det 5A = 5 \det A$?
		\vfill\null\columnbreak

		\item What is the determinant of an elementary row replacement matrix?

		\vfill

		\item What affect does adding two copies of one row to a different row have on the determinant of a matrix? What about columns?
		\vfill
	\end{enumerate}
\end{multicols*}
%Section 3.1: 2,3,4,5,10,12,14,15,19,21,22,25,27,29,31,32

\end{document}