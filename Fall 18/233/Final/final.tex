%% Please change the file name by replacing N with the apporpriate number
%% corresponding to the current homework and XX with your initials.
%% https://www.math.uci.edu/~gpatrick/jsOnline/hw1.html

\documentclass[11pt,letterpaper]{report}
\usepackage{amssymb,amsfonts,color,graphicx,amsmath,enumerate,tikz-cd}
\usepackage{tikz} %This package offers the ability to draw pictures
\usepackage{amsthm}

\newcommand{\naturals}{\mathbb{N}}
\newcommand{\integers}{\mathbb{Z}}
\newcommand{\complex}{\mathbb{C}}
\newcommand{\reals}{\mathbb{R}}
\newcommand{\exreals}{\overline{\mathbb{R}}}
\newcommand{\mcal}[1]{\mathcal{#1}}
\newcommand{\mable}{measurable}
\newcommand{\quats}{\mathbb{H}}
\newcommand{\rationals}{\mathbb{Q}}
\newcommand{\norm}{\trianglelefteq}
\newcommand{\Aut}{\text{Aut}}
\newcommand{\disk}{\mathbb{D}}
\newcommand{\halfplane}{\mathbb{H}}
\newcommand{\Lp}[2]{\left\|{#1}\right\|_{L^{#2}}}
\newcommand{\supp}[1]{\text{supp}({#1})}
\newcommand{\Hom}[2]{\text{Hom}_{{#1}}({#2})}
\newcommand{\tr}{\text{tr}}
\newcommand{\field}[1]{\mathbb{F}_{{#1}}}
\newcommand{\Gal}[1]{\text{Gal}\left({#1}\right)}
\newcommand{\esssup}{\text{ess sup }}
\newcommand{\essinf}{\text{ess inf }}
\newcommand{\affine}{\mathbb{A}}
\newcommand{\proj}{\mathbb{P}}

\newenvironment{solution}
{\begin{proof}[Solution]}
{\end{proof}}

\voffset=-3cm
\hoffset=-2.25cm
\textheight=24cm
\textwidth=17.25cm
\addtolength{\jot}{8pt}
\linespread{1.3}

\begin{document}
\noindent{\em Liam Hardiman\hfill{December 5, 2018} }
% Please give relevant information
\begin{center}
{\bf \Large 233A - Final} %Replace N with the appropriate number
\vspace{0.2cm}
\hrule
\end{center}

\noindent\textbf{1.4.6}
Let $Y$ be a subspace of a topological space $X$. Show that $Y$ is irreducible if and only if the closure of $Y$ in $X$ is irreducible.
\begin{proof}
	First suppose that $Y$ is irreducible. If $\overline{Y}$ (the closure of $Y$ in $X$) were reducible, then we could write $\overline{Y} = \tilde{F}_1\cup \tilde{F}_2$, where $\tilde{F}_1$ and $\tilde{F}_2$ are nonempty (relatively) closed subsets of $\overline{Y}$. In particular, this means that we can write $\overline{Y}\subseteq F_1\cup F_2$, where $F_1$ and $F_2$ are closed in $X$ and $Y$ is not entirely contained in either $F_1$ or $F_2$. If $Y$ is contained in say $F_1$, then $\overline{Y}\subseteq \overline{F}_1 = F_1$, which contradicts the reducibility of $\overline{Y}$, so $Y$ isn't contained in $F_1$. By symmetry, $Y$ is not contained in $F_2$ either. But we have
	\[
	Y\subseteq \overline{Y}\subseteq F_1\cup F_2.
	\]
	This shows that $Y$ is contained in the union of closed (in $X$) subsets, but is contained in neither set individually, contradicting the irreducibility of $Y$. We conclude that $\overline{Y}$ is also irreducible.\\

	\noindent Conversely, suppose that $\overline{Y}$ is irreducible but $Y$ is reducible. Then $Y\subseteq F_1\cup F_2$, where $F_1$ and $F_2$ are closed in $X$ and $Y$ is contained in neither $F_1$ nor $F_2$. When we take the closure of both sides of this inclusion we get
	\[
	\overline{Y}\subseteq \overline{F_1\cup F_2} = \overline{F_1}\cup \overline{F_2} = F_1\cup F_2.
	\]
	Since $\overline{Y}$ is irreducible, it must be contained in $F_1$ or $F_2$, say $F_1$. But then $Y\subseteq \overline{Y}\subseteq F_1$, contradicting our assumption about $Y$ not being contained in $F_1$. We conclude that $Y$ is irreducible.
\end{proof}

\noindent\textbf{2.6.13}
Let $X$ and $Y$ be prevarieties with affine open covers $\{U_i\}$ and $\{V_j\}$, respectively. Construct the product prevariety $X\times Y$ by gluing the affine varieties $U_i\times V_j$ together. Moreover, show that there are projection morphisms $\pi_X:X\times Y\to X$ and $\pi_Y:X\times Y\to Y$ satisfying the usual universal property for products.
\begin{proof}
	The affine varieties $U_i\times V_j$ (as the product of two affine varieties is an affine variety) form a finite affine open cover for $X\times Y$ as a topological space. The idea now is to glue the sets $U_i\times V_j$ and $U_k\times V_l$ along the identity morphism on the intersection $(U_i\cap U_k)\times (V_j\cap V_l)$. Let $f_{ijkl}:U_i\times V_j\to U_k\times V_l$ be the identity morphism on the intersection. Then we clearly have that $f_{ijkl} = (f_{klij})^{-1}$ and the cocycle condition holds on triple intersections.\\

	\noindent Let's show that $X\times Y$ is irreducible. Suppose that $X\times Y = F_1\cup F_2$ where $F_1$ and $F_2$ are closed, no-one properly containing $X\times Y$. For any fixed $y\in Y$, the map $\iota_y: X\to X\times Y$ that sends $x$ to $(x,y)$ is continuous. Consequently, the preimage, $\iota_y^{-1}(F_i)$ is closed in $X$ for $i=1,2$ and all $y$. Since the arbitrary intersection of closed sets is closed, we have that the covering of $X\times Y$ by closed sets induces a covering of $X$ by closed sets. But $X$ is irreducible, so this covering must be trivial. We conclude that $X\times Y$ is irreducible. So far we have that $X\times Y$ has an affine open covering, and is irreducible.\\

	\noindent We build the ring of regular functions on $X\times Y$ locally. Say $U$ is open in $X\times Y$ and contains $x\in U_i\times V_j$ for some $i,j$. We say that a function $f$ is regular on $U$ if its restriction to $U\cap U_i\times V_j$ is regular when considered as a function on the variety $U_i\times V_j$. The sheaf properties of the rings of functions on $U_i\times V_j$ are inherited.\\

	\noindent Let's show that our projection maps, $\pi_X$ and $\pi_Y$ are indeed morphisms. That they are continuous is clear. Say, $U\subseteq X$ is open and $f: U\to k$ is a regular function on $X$. Take $P\in \pi_X^{-1}(U)$ and write $P = (x, y)$ where $x\in X$ and $y\in Y$. The pullback, $\pi_X^*$ behaves as follows:
	\[
	(\pi_X^*f)(P) = f\circ \pi_X(P) = f(x).
	\]
	Since $f$ is regular, this shows that $\pi_X^*f$ is regular. Since $\pi_X$ pulls regular functions back to regular functions, it is a morphism. The same holds for $\pi_Y$.\\

	\noindent Finally, let's show that our projections satisfy the universal property of products. Suppose we're given a prevariety $Z$ and morphisms $f: Z\to X$ and $f: Z\to Y$. We need to show that there is a unique morphism $h: Z\to X\times Y$ that makes the left diagram commute.
	\[
	\begin{tikzcd}
		Z
		\arrow[drr, bend left, "g"]
		\arrow[ddr, bend right, "f"]
		\arrow[dr, dotted, "h" description] & & \\
			& X \times Y
			\arrow[r, "\pi_Y"]
			\arrow[d, "\pi_X"]& Y\\
			& X
	\end{tikzcd}\qquad 
	\begin{tikzcd}
		Z_{i,j}
		\arrow[drr, bend left, "g|_{Z_{i,j}}"]
		\arrow[ddr, bend right, "f_{Z_{i,j}}"]
		\arrow[dr, dotted, "h_{i,j}" description] & & \\
			& U_i \times V_j
			\arrow[r, "\pi_Y|_{U_i\times V_j}"]
			\arrow[d, "\pi_X|_{U_i\times V_j}"]& V_j\\
			& U_i
	\end{tikzcd}
	\]
	Define $h(z) = (f(z), g(z))$. This map clearly makes the diagram commute, at least set-theoretically. It remains to show that $h$ is a morphism and that it is unique. The idea is to pass to the universal property of the product varieties $U_i\times V_j$. Since the morphisms $f$ and $g$ are continuous, for any $U_i\times V_j$ we have that $f^{-1}(U_i)\cap g^{-1}(V_j)$ is an affine open set in $Z$. But then this set can be covered by an affine variety, say $Z_{i,j}$. The restrictions of $f$ and $g$ to $Z_{i,j}$ induce a unique map $h_{i,j}: Z_{i,j}\to U_i\times V_j$ by the universal property of products of affine varieties, shown in the diagram on the right. The $Z_{i,j}$ cover $Z$, so the $h_{i,j}$ weave together to agree with $h$. Since each $h_{i,j}$ is a unique morphism, we have that $h$ is a unique morphism too.
\end{proof}


\noindent\textbf{3.5.5}
Let $V$ be the vector space over $k$ of homogeneous degree-2 polynomials in three variables $x_0, x_1, x_2$ and let $\proj(V) \cong \proj^5$ be its projectivization.
\begin{enumerate}[(i)]
	\item Show that the space of conics in $\proj^2$ can be identified with an open subset $U$ of $\proj^5$. What geometric objects can be associated to the points in $\proj^5\setminus U$?
	\begin{proof}
		A conic in $\proj^2$ is determined by a homogeneous quadratic equation
		\[
		f(x,y,z) = ax^2 + bxy + cy^2 + dxz+eyz+fz^2 = 0.
		\]
		Equivalently, we can represent this equation with the matrix equation
		\begin{equation}\label{matrix}
		\begin{bmatrix}
			x & y & z
		\end{bmatrix}\begin{bmatrix}
			a & b/2 & d/2\\
			b/2 & c & e/2\\
			d/2 & e/2 & f
		\end{bmatrix}\begin{bmatrix}
			x\\y\\z
		\end{bmatrix} = 0.
		\end{equation}
		This matrix is symmetric, so we can change coordinates so that the above matrix is diagonal, giving the equation $AX^2+BY^2 +CZ^2 = 0$, for some $A,B,C$. If two of $A,B,C$ are zero, say $A=B=0$, then the conic $CZ^2 = CZ\cdot Z$ is reducible. If just $A = 0$, then $BY^2 + CZ^2 = (\sqrt{B}Y + \sqrt{-C}Z)(\sqrt{B}Y - \sqrt{-C}Z)$ is again reducible. These correspond to degenerate conics. If one or two of $A,B,C$ are zero, then the diagonal matrix with entries $A,B,C$ has determinant zero. But the determinant is invariant under coordinate changes, so the matrix in equation (\ref{matrix}) also has determinant zero.\\

		\noindent Suppose that none of $A,B,C$ are zero. If $AX^2+BY^2+CZ^2$ were reducible, we could write
		\[
		AX^2+BY^2+CZ^2 = (\sqrt{A}X + g(Y,Z))(\sqrt{A}X+h(Y,Z)).
		\]
		Multiplying this out shows that $g(Y,Z)+h(Y,Z) = 0$ and $g(Y,Z)h(Y,Z) = BY^2+CZ^2$. This would imply that
		\[
		-g(Y,Z)^2 = (\sqrt{B}Y + \sqrt{-C}Z)(\sqrt{B}Y-\sqrt{-C}Z).
		\]
		$k[Y,Z]$ is a unique factorization domain, but the left-hand side of this equation is a square and the right-hand side isn't (under the modest assumption that the characteristic of $k$ is not 2). This shows that $AX^2+BY^2+CZ^2$ is irreducible, and therefore corresponds to a non-degenerate conic. By a similar argument used in the degerate case, this shows implies that the determinant of the matrix in (\ref{matrix}) is nonzero.\\

		\noindent We have shown that our conic is non-degenerate if and only if the determinant of the matrix in (\ref{matrix}) is non-vanishing. The determinant is a polynomial in the in the coefficients $(a:b:c:d:e:f)\in \proj^5$, so the non-vanishing locus, and therefore the set of non-degenerate conics, corresponds to an open set in $\proj^5$.
	\end{proof}

	\item Show that it is a linear condition in $\proj^5$ for the conics to pass through a given point in $\proj^2$. If $P\in \proj^2$ is a point, show that there is a linear subspace $L\subseteq \proj^5$ such that the conics passing through $P$ are exactly those in $U\cap L$. What happens in $\proj^5\setminus U$?
	\begin{proof}
		Suppose that the conic determined by $f(x,y,z) = ax^2+bxy+cy^2 + dxz+eyz+fz^2$ passes through the point $(x_0, y_0, z_0)\in \proj^2$. Then the coefficients $a,b,c,d,e,f$ satisfy the linear equation
		\[
		ax_0^2+bx_0y_0+cy_0^2 + dx_0z_0+ey_0z_0+fz_0^2 = 0.
		\]
		Call the set of all coefficients satisfying the above equation $L$. Since $L$ is the solution set to a homogeneous linear equation, it is a linear subspace of $\proj^5$.
		As the set $U$ corresponds to the non-degenerate conics in $\proj^2$, we have that $U\cap L$ corresponds to the non-degenerate conics passing through $(x_0: y_0: z_0)$. On the other hand, $(\proj^5\setminus U)\cap L$ corresponds to the degenerate conics passing through $(x_0:y_0:z_0)$.
	\end{proof}

	\item Prove that there is a unique conic through any five points in $\proj^2$, as long as no three of them lie on a line. What happens if three of them do lie on a line?
	\begin{proof}
		
	\end{proof}
\end{enumerate}

\noindent\textbf{4.6.10}
Let $X\subseteq \affine^n$ be an affine variety, and let $Y_1, Y_2\subsetneq X$ be irreducible, closed subsets, no-one contained in the other. Let $\tilde{X}$ be the blow-up of $X$ at the (possibly non-radical) ideal $I(Y_1)+I(Y_2)$. Then the strict transforms of $Y_1$ and $Y_2$ on $\tilde{X}$ are disjoint.


\end{document}