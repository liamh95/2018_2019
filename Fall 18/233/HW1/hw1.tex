%% Please change the file name by replacing N with the apporpriate number
%% corresponding to the current homework and XX with your initials.
%% https://www.math.uci.edu/~gpatrick/jsOnline/hw1.html

\documentclass[11pt,letterpaper]{report}
\usepackage{amssymb,amsfonts,color,graphicx,amsmath,enumerate}
\usepackage{tikz} %This package offers the ability to draw pictures
\usepackage{amsthm}

\newcommand{\naturals}{\mathbb{N}}
\newcommand{\integers}{\mathbb{Z}}
\newcommand{\complex}{\mathbb{C}}
\newcommand{\reals}{\mathbb{R}}
\newcommand{\exreals}{\overline{\mathbb{R}}}
\newcommand{\mcal}[1]{\mathcal{#1}}
\newcommand{\mable}{measurable}
\newcommand{\quats}{\mathbb{H}}
\newcommand{\rationals}{\mathbb{Q}}
\newcommand{\norm}{\trianglelefteq}
\newcommand{\Aut}{\text{Aut}}
\newcommand{\disk}{\mathbb{D}}
\newcommand{\halfplane}{\mathbb{H}}
\newcommand{\Lp}[2]{\left\|{#1}\right\|_{L^{#2}}}
\newcommand{\supp}[1]{\text{supp}({#1})}
\newcommand{\Hom}[2]{\text{Hom}_{{#1}}({#2})}
\newcommand{\tr}{\text{tr}}
\newcommand{\field}[1]{\mathbb{F}_{{#1}}}
\newcommand{\Gal}[1]{\text{Gal}\left({#1}\right)}
\newcommand{\esssup}{\text{ess sup }}
\newcommand{\essinf}{\text{ess inf }}
\newcommand{\affine}{\mathbb{A}}
\newenvironment{solution}
{\begin{proof}[Solution]}
{\end{proof}}

\voffset=-3cm
\hoffset=-2.25cm
\textheight=24cm
\textwidth=17.25cm
\addtolength{\jot}{8pt}
\linespread{1.3}

\begin{document}
\noindent{\em Liam Hardiman\hfill{October 9, 2018} }
% Please give relevant information
\begin{center}
{\bf \Large 233 - Homework 1} %Replace N with the appropriate number
\vspace{0.2cm}
\hrule
\end{center}

\noindent\textbf{1.4.1} Let $X_1, X_2\subset \affine^n$ be algebraic sets. Show that
\begin{enumerate}[(i)]
	\item $I(X_1\cup X_2) = I(X_1)\cap I(X_2)$.
	\item $I(X_1\cap X_2) = \sqrt{I(X_1)+I(X_2)}$.
\end{enumerate}
\begin{proof}
	\begin{enumerate}[(i)]
		\item Suppose $f\in k[x_1, \ldots, x_n]$ vanishes on $X_1\cup X_2$. Then it must vanish on $X_1$ and $X_2$, so $I(X_1\cup X_2)\subseteq I(X_1)\cap I(X_2)$. Conversely, suppose that $f$ vanishes on $X_1$ and $X_2$. Then it vanishes on their union as well, so $I(X_1\cup X_2)\supseteq I(X_1)\cap I(X_2)$, and we're done.

		\item Since $X_1$ and $X_2$ are algebraic sets, we have that $X_1 = Z(J_1)$ and $X_2 = Z(J_2)$ for some ideals $J_1, J_2\subseteq k[x_1, \ldots, k_n]$. By Hilbert's Nullstellensatz we have that
		\[
		\sqrt{I(X_1)+I(X_2)} = \sqrt{I(Z(J_1))+I(Z(J_2))} = \sqrt{\sqrt{J_1} + \sqrt{J_2}}.
		\]
		Now $Z(J_i) = Z(\sqrt{J_i})$, so we can take $J_1$ and $J_2$ to be radical, which gives
		\[
		\sqrt{I(X_1)+I(X_2)} = \sqrt{J_1+J_2}.
		\]
		On the other hand, we have, again by Nullstellensatz
		\[
		I(X_1)\cap I(X_2) = I(Z(J_1))\cap I(Z(J_2)) = I(Z(J_1+J_2)) = \sqrt{J_1+J_2},
		\]
		and we're done.
	\end{enumerate}
\end{proof}

\noindent\textbf{1.4.2}
Let $X\subseteq \affine^3$ be the union of the three coordinate axes. Determine generators for the ideal $I(X)$. Show that $I(X)$ cannot be generated by fewer than three elements, although $X$ has codimension 2 in $\affine^3$.
\begin{proof}
	The $z$-axis is the set of points where $x=y=0$. In order for a polynomial, $p$, to vanish here we need $p(0, 0, z) = 0$ for all $z$. This tells us that $p$ can contain no constant term and that any monomial divisible by $z$ must also be divisible by $x$ or $y$. Thus, any monomial vanishing on the $z$ axis must be divisible by $x$ or $y$. The same argument shows that any monomial vanishing on the $x$-axis must be divisible by $y$ or $z$ and any monomial vanishing on the $y$ axis must be divisible by $x$ or $z$. By problem 1.4.1, we're interested in the ideal $(x,y)\cap (y, z)\cap (x, z)$.\\
	Going piece by piece we have
	\[
	I = (x,y)\cap (y,z)\cap (x,z) = (x,y)\cap (xy, z) = (xy, xz, yz).
	\]
	Now we show that this ideal cannot be generated by fewer than three elements of $k[x,y,z]$. It's clearly not generated by a single element because $xy$, $xz$, and $yz$ don't have a common factor. Suppose that $I = (f,g)$. 
\end{proof}

f = Pxy + Qxz + Ryz

\end{document}