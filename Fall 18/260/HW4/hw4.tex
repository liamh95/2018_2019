%% Please change the file name by replacing N with the apporpriate number
%% corresponding to the current homework and XX with your initials.
%% https://www.math.uci.edu/~gpatrick/jsOnline/hw1.html

\documentclass[11pt,letterpaper]{report}
\usepackage{amssymb,amsfonts,color,graphicx,amsmath,enumerate}
\usepackage{tikz} %This package offers the ability to draw pictures
\usepackage{amsthm}

\newcommand{\naturals}{\mathbb{N}}
\newcommand{\integers}{\mathbb{Z}}
\newcommand{\complex}{\mathbb{C}}
\newcommand{\reals}{\mathbb{R}}
\newcommand{\exreals}{\overline{\mathbb{R}}}
\newcommand{\mcal}[1]{\mathcal{#1}}
\newcommand{\mable}{measurable}
\newcommand{\quats}{\mathbb{H}}
\newcommand{\rationals}{\mathbb{Q}}
\newcommand{\norm}{\trianglelefteq}
\newcommand{\Aut}{\text{Aut}}
\newcommand{\disk}{\mathbb{D}}
\newcommand{\halfplane}{\mathbb{H}}
\newcommand{\Lp}[2]{\left\|{#1}\right\|_{L^{#2}}}
\newcommand{\supp}[1]{\text{supp}({#1})}
\newcommand{\Hom}[2]{\text{Hom}_{{#1}}({#2})}
\newcommand{\tr}{\text{tr}}
\newcommand{\field}[1]{\mathbb{F}_{{#1}}}
\newcommand{\Gal}[1]{\text{Gal}\left({#1}\right)}
\newcommand{\esssup}{\text{ess sup }}
\newcommand{\essinf}{\text{ess inf }}
\newcommand{\affine}{\mathbb{A}}
\newcommand{\img}{\text{Im}}
\newcommand{\dist}{\text{dist}}
\newcommand{\spec}{\text{Spec}}

\newenvironment{solution}
{\begin{proof}[Solution]}
{\end{proof}}

\voffset=-3cm
\hoffset=-2.25cm
\textheight=24cm
\textwidth=17.25cm
\addtolength{\jot}{8pt}
\linespread{1.3}

\begin{document}
\noindent{\em Liam Hardiman\hfill{December 7, 2018} }
% Please give relevant information
\begin{center}
{\bf \Large 260A - Homework 4} %Replace N with the appropriate number
\vspace{0.2cm}
\hrule
\end{center}

\noindent\textbf{Problem 1. }Let $E$ and $F$ be two Banach spaces, and let $T\in \mcal{L}(E,F)$. Prove that $\img(T)$ is closed if and only if there exists a constant $C>0$ such that
\[
\dist(x, \ker T)\leq C\cdot \|Tx\|,\quad \forall x\in E.
\]
\begin{proof}
	% By passing to $E/\ker T$, we can assume that $T$ is injective. In this setting, we're trying to show that the image of $T$ is closed if and only if there is some positive $C$ such that
	% \[
	% \|x\|\leq C\cdot \|Tx\|
	% \]
	% for all $x\in E$. If the image of $T$ is closed, then $F/\img T$ is a Banach space
	First suppose that the given inequality holds for some $C>0$. Let $Tx_n$ be a convergent sequence in the image of $T$. Then the sequence of $x_n+\ker T$'s converges in the quotient $E/\ker T$ by the given inequality. Since $T$ is continuous, $\ker T$ is closed and the quotient $E/\ker T$ is complete. Thus, $x_n+\ker T$ converges to some $x+\ker T$. By continuity, $Tx_n$ then converges to $Tx$, which is in the image of $T$. Thus, the image of $T$ is closed.\\

	\noindent Conversely, suppose that $\img(T)$ is closed. Then the image is a Banach space. By the first isomorphism theorem, $T$ induces an isomorphism $\tilde{T}: E/\ker T \to \img T$. By the open mapping theorem, $\tilde{T}$ is a homeomorphism, and the statement that $\tilde{T}$ is continuous is equivalent to the desired inequality.
\end{proof}

\noindent\textbf{Problem 2. }
Prove that if $H$ is a Hilbert space and $B$ is a Banach space, then the space $\mcal{L}_c(B, H)$ of compact operators $B\to H$ is the closure of the set of operators in $\mcal{L}(B, H)$ which are of finite rank.
\begin{proof}
Suppose $T$ is a compact operator $B\to H$. By the compactness of $T$, for any $n>0$ we can find a finite covering of $\overline{T[B(0,1)]}$, the closure of the image of the unit ball in $B$, by balls of radius $\frac{1}{n}$. Say $\overline{T[B(0,1)]}\subseteq \cup_{j=1}^{M_n}B(y_j, \frac{1}{n})$ for some finite $M_n>0$. Let $P_n$ be the projection onto the vectors $y_1, \ldots, y_{M_n}$. Then $P_nT$ is clearly of finite rank as $P_n$ has finite rank.\\

\noindent Now given any $x\in B(0,1)\subseteq B$, we can find a $y_j$ with $\|Tx-y_j\|_H \leq \frac{1}{n}$. We use this to show that the $P_nT$'s approximate $T$. We have
\begin{align*}
	\|P_nTx - Tx\|_H &\leq \|P_nTx - y_j\|_H + \|y_j - Tx\|_H\\
	&\leq \|Tx-y_j\|_H + \|y_j-Tx\|_H\\
	&\leq \frac{2}{n}.
\end{align*}
The fact that $\|P_nTx-y_j\|_H \leq \|Tx-y_j\|_H$ follows from the fact that $P_n$ projects onto the space spanned by the $y_k$'s. Sending $n\to \infty$ shows that the finite rank $P_nT$'s approximate $T$, so the compact operators $B\to H$ are in the closure of of the set of finite rank operators in $\mcal{L}(B, H)$.\\

\noindent Conversely, suppose that $T_n$ is a sequence of finite rank operators in $\mcal{L}(B, H)$ that converges to $T\in \mcal{L}(B, H)$. Choose $N$ large so that $\|T_n - T\|<\epsilon$ for all $n>N$. Since finite-rank operators are compact, for any $n$ we can cover $\overline{T_n[B(0,1)]}$ by finitely many $\epsilon$-balls. Since $\|T_nx-Tx\|_H<\epsilon$ for any $x\in B(0,1)$, we have that the $2\epsilon$-balls with the same centers cover $\overline{T[B(0,1)]}$. Sine $\epsilon$ was arbitrary, this shows that the closed image of the unit ball under $T$ is compact, so $T$ is a compact operator.
\end{proof}

\noindent\textbf{Problem 3. }Let $B$ be a complex Banach space, $B\neq \{0\}$, and let $T\in \mcal{L}(B, B)$. Prove the following.
\begin{enumerate}[(i)]
	\item There exists a non-empty compact set $\spec(T)\subseteq \complex$, called the spectrum of $T$, such that the resolvent $R(z) = (T-zI)^{-1}\in \mcal{L}(B, B)$ exists if and only if $z\notin \spec(T)$.
	\begin{proof}
		
	\end{proof}
\end{enumerate}

\noindent\textbf{Problem 4. }
Let $E = L^p[0,1]$ with $1\leq p<\infty$. Given $u\in E$, set
\[
Tu(x) = \int_0^xu(t)\ dt.
\]
\begin{enumerate}[(i)]
	\item Prove that $T:E\to E$ is compact.
	\begin{proof}
		Fix $u\in L^p[0,1]$ and suppose that $x_n\to x$ in $[0,1]$. Since $|\chi_{[0, x_n]}(t)u(t)|\leq |u(t)|\in L^p[0,1]$ for all $n$, the dominated convergence theorem tells us that
		\[
			Tu(x_n) = \int_0^1\chi_{[0, x_n]}(t)u(t)\ dt \to \int_0^1 \chi_{[0, x]}(t)u(t)\ dt = Tu(x).
		\]
		That is, $T$ maps $E$ into $C[0,1]$. Suppose we're given a bounded sequence $u_n\in L^p[0,1]$, i.e. $\Lp{u_n}{p}\leq M<\infty$. We then have
		\begin{align*}
		|Tu_n(x)| &= \left|\int_0^1\chi_{[0,x]}(t)u_n(t)\ dt\right|\\
		&\leq \int_0^1\chi_{[0,x]}(t)|u_n(t)|\ dt\\
		&\leq x^{1/q}\cdot \Lp{u}{p}\\
		&\leq M,
		\end{align*}
		where $\frac{1}{p}+\frac{1}{q} = 1$ (the inequality still holds if $p = 1$). Thus, the sequence of continuous functions $Tu_n$ is uniformly bounded. Now fix $\epsilon>0$. For any $x<y\in [0,1]$ and $p>1$ we have
		\begin{align*}
			|Tu_n(x) - Tu_n(y)| & = \left|\int_0^1\chi_{[x,y]}(t)u_n(t)\ dt\right|\\
			&\leq \int_0^1\chi_{[x,y]}(t)|u_n(t)|\ dt\\
			&\leq |y-x|^{1/q}\cdot \Lp{u_n}{p}\\
			&\leq |y-x|^{1/q}\cdot M.
		\end{align*}
		We can choose $|y-x|$ small enough so that the above quantity is bounded by $\epsilon$, which shows that the sequence of continuous functions $Tu_n$ is equicontinuous (I'm not sure how get this to work for $p=1$). By the Arzela-Ascoli theorem we have that $Tu_n$ has a uniformly convergent subsequence. Since uniform convergence implies $L^p$ convergence, we have that $Tu_n$ has a convergent subsequence in $E$, so $T$ is a compact operator.
	\end{proof}

	\item Compute the eigenvalues of $T$ and the spectrum of $T$.
	\begin{solution}
		In our discussion of part (i) we showed that $T$ maps $L^p[0,1]$ into $C[0,1]$. In particular, if $Tu = \lambda u$, then $u$ must be continuous. But the fundamental theorem of calculus tells us that the integral of a continuous function is differentiable, so $u$ is actually differentiable. Differentiating both sides of the eigenvalue equation gives $u = \lambda u'$. If $\lambda \neq 0$, then the solutions to this differential equation are of the form $u(x) = Ce^{x/\lambda}$, $C\in \complex$. However, we must also have
		\[
		\lambda u(0) = Tu(0) = \int_0^0u(t)\ dt = 0,
		\]
		so $u(0) = Ce^0 = 0$. But then $C$ must be zero, which would force $u$ to be identically zero. We conclude that there are no eigenvectors for $\lambda \neq 0$. If $\lambda = 0$ then any $L^p$ function with vanishing integral is an eigenfunction with eigenvalue zero.
	\end{solution}
\end{enumerate}

\noindent\textbf{Problem 5. }
Let $X$, $Y$, and $Z$ be three Banach spaces with norms $\|\cdot \|_X$, $\|\cdot \|_Y$, and $\|\cdot \|_Z$. Assume that $X\subseteq Y$ with compact injection and that $Y\subseteq Z$ with continuous injection. Prove that for any $\epsilon>0$ there exists $C_\epsilon\geq 0$ such that
\[
\|u\|_Y\leq \epsilon\|u\|_X+C_\epsilon\|u\|_Z
\]
for all $u\in X$.
\begin{proof}
	Suppose the proposition were false: that for some $\epsilon$ and for every $C\geq 0$ there exists a $u_C$ such that
	\[
	\|u_C\|_Y> \epsilon\|u_C\|_X+C\|u_C\|_Z
	\]
	for all $x\in X$. Set $C = n$ and let $u_n$ be a sequence in $X$ such that the above equality holds, i.e.
	\begin{equation}\label{contra_ineq}
	\|u_n\|_Y > \epsilon \|u_n\|_X + n\|u_n\|_Z.
	\end{equation}
	We can assume without loss of generality that the sequence $u_n$ has norm 1 in $X$, since replacing $u_n$ with $\frac{u_n}{\|u_n\|_X}$ gives the same inequality after multiplying through by $\|u_n\|_X$. By the compactness of the injection of $X$ into $Y$, we have that $u_n$ has a convergent subsequence in $Y$. Without loss of generality, assume then that $u_n$ converges in $Y$. Rearranging (\ref{contra_ineq}) gives
	\begin{gather*}
	n\|u_n\|_Z < \|u_n\|_Y - \epsilon \|u_n\|_X \leq \|u_n\|_Y
	\\ \iff
	\|u_n\|_Z <\frac{1}{n}\|u_n\|_Y.
	\end{gather*}
	\noindent Since $u_n$ converges in $Y$, the right-hand side of the above inequality must go to zero. Since $Y$ continuously embeds into $Z$ and $u_n$ converges in $Y$, we must have that $u_n$ converges to zero in both $Y$ and $Z$. But then the left-hand side of (\ref{contra_ineq}) will tend to 0 and the right-hand side will tend to $\epsilon$: a contradiction. We conclude that the proposition is true.\\

	\noindent In class we showed (using the Arzela-Ascoli theorem) that $C^1([0,1])$ compactly embeds into $C([0,1])$. We also have that $C([0,1])$ continuously embeds into $L^1([0,1])$ by $\int_0^1|f|\ dx \leq \|f\|_\infty$. By the proposition we then have that for all $\epsilon>0$ there is some $C_\epsilon$ with
	\[
	\max_{x\in [0,1]}|f(x)| \leq \epsilon\cdot \max_{x\in [0,1]}|f'(x)| + C_\epsilon\Lp{f}{1}
	\]
	for all $f\in C^1([0,1])$.
\end{proof}

\end{document}