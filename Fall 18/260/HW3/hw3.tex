%% Please change the file name by replacing N with the apporpriate number
%% corresponding to the current homework and XX with your initials.
%% https://www.math.uci.edu/~gpatrick/jsOnline/hw1.html

\documentclass[11pt,letterpaper]{report}
\usepackage{amssymb,amsfonts,color,graphicx,amsmath,enumerate,array}
\usepackage{tikz} %This package offers the ability to draw pictures
\usepackage{amsthm}

\newcommand{\naturals}{\mathbb{N}}
\newcommand{\integers}{\mathbb{Z}}
\newcommand{\complex}{\mathbb{C}}
\newcommand{\reals}{\mathbb{R}}
\newcommand{\exreals}{\overline{\mathbb{R}}}
\newcommand{\mcal}[1]{\mathcal{#1}}
\newcommand{\mable}{measurable}
\newcommand{\quats}{\mathbb{H}}
\newcommand{\rationals}{\mathbb{Q}}
\newcommand{\norm}{\trianglelefteq}
\newcommand{\Aut}{\text{Aut}}
\newcommand{\disk}{\mathbb{D}}
\newcommand{\halfplane}{\mathbb{H}}
\newcommand{\Lp}[2]{\left\|{#1}\right\|_{L^{#2}}}
\newcommand{\supp}[1]{\text{supp}({#1})}
\newcommand{\Hom}[2]{\text{Hom}_{{#1}}({#2})}
\newcommand{\tr}{\text{tr}}
\newcommand{\field}[1]{\mathbb{F}_{{#1}}}
\newcommand{\Gal}[1]{\text{Gal}\left({#1}\right)}
\newcommand{\esssup}{\text{ess sup }}
\newcommand{\essinf}{\text{ess inf }}
\newcommand{\affine}{\mathbb{A}}

\newcolumntype{L}{>{$}l<{$}} % math-mode version of "l" column type
\newcolumntype{C}{>{$}c<{$}} % math-mode version of "l" column type

\newenvironment{solution}
{\begin{proof}[Solution]}
{\end{proof}}

\voffset=-3cm
\hoffset=-2.25cm
\textheight=24cm
\textwidth=17.25cm
\addtolength{\jot}{8pt}
\linespread{1.3}

\begin{document}
\noindent{\em Liam Hardiman\hfill{November 26, 2018} }
% Please give relevant information
\begin{center}
{\bf \Large 260A - Homework 3} %Replace N with the appropriate number
\vspace{0.2cm}
\hrule
\end{center}

\noindent\textbf{Problem 1. }
Let $(b_1, b_2, \ldots)$ be a sequence of complex numbers such that $\sum_{n=1}^\infty b_nc_n$ is convergent for every $c = (c_1, c_2, \ldots)\in \ell^2$. Show that $b\in \ell^2$.
\begin{proof}
	Consider the sequence of maps $T_n: \ell^2\to \complex$ that send $(c_1, \ldots)$ to $\sum_{j=1}^n b_jc_j$. Since each $T_n$ is just a finite sum, we have that the $T_n$'s form a sequence of bounded linear operators on $\ell^2$. Furthermore, this sequence is pointwise bounded: given any $(c_1, c_2, \ldots)\in \ell^2$, since $\sum_{j=1}^\infty b_jc_j$ converges, we have that the sequence of partial sums $|T_n(c_1, c_2, \ldots)| = |\sum_{j=1}^n b_jc_j|$ is bounded. By the uniform boundedness principle, we have that
	\[
	\sup_{n\in \naturals,\ \|(c_1, c_2, \ldots)\|_2 = 1}|T_n(c_1, c_2, \ldots)| = \sup_{n\in \naturals}\|T_n\| = \sum_{j=1}^\infty |b_j|<\infty,
	\]
	so $(b_1, b_2, \ldots)\in \ell^2$.
\end{proof}

\noindent\textbf{Problem 2. }
Let $M$ be a measurable subset of $\reals^n$ with finite positive measure. Prove that $L^q(M)$ is of the first category in $L^p(M)$ if $1\leq p<q\leq \infty$.
\begin{proof}
	Since $M$ has finite measure, we have that $L^q(M)\subseteq L^p(M)$ whenever $1\leq p<q\leq \infty$. Consider the injection $\iota: L^q(M)\to L^p(M)$ that simply sends $f\in L^p(M)$ to itself. By the generalized H\"older inequality we have that $\Lp{\iota(f)}{p} = \Lp{f}{p}\leq \mu(M)^{1/r}\Lp{f}{q}$, where $\frac{1}{p} = \frac{1}{q} + \frac{1}{r}$. This shows that $\iota$ is bounded, and therefore continuous. Since $L^p(M)$ and $L^q(M)$ are Banach spaces, the open mapping theorem tells us that the image of $\iota$ is either surjective and open or of the first category in $L^p(M)$.\\

	\noindent Our plan is to show that $\iota$ is \textit{not} surjective, i.e. that $L^p(M)\setminus L^q(M)$ is nonempty. We'll do this by showing that $L^q(M) = L^p(M)$ would forbid the existence of subsets of $M$ with arbitrarily small measure. Since sets of positive measure in $\mathbb{R}^n$ \textit{do} contain sets of arbitrarily small measure, we'll conclude that $L^q(M)\neq L^p(M)$.\\

	\noindent If the embedding $\iota: L^q(M)\to L^p(M)$ were surjective, then the open mapping theorem would imply that $\iota$ is actually a homeomorphism. In particular, its inverse, $\iota': L^p(M)\to L^q(M)$ is a bounded operator. Let $A$ be a subset of $M$ with positive, finite measure and define the function
	\[
	f_A(x) = \frac{1}{m(A)^{1/p}}\cdot \chi_A(x).
	\]
	It's clear that $\Lp{f_A}{p} = 1$ and that $\Lp{f_A}{q} = \frac{1}{m(A)^{1/p-1/q}}$. Since $\iota'$ is bounded, we have
	\[
	0<\Lp{f_A}{q} \leq \|\iota'\|\cdot \Lp{f_A}{p} \implies 0<\frac{1}{\|\iota'\|^{\frac{pq}{q-p}}}\leq m(A).
	\]
	This puts a positive lower bound on the measure of subsets of $M$. But $M$, as a subset of $\reals^n$ with positive measure, contains set of arbitrarily small measure. We conclude that $L^q(M)$ is of the first category of $L^p(M)$.
\end{proof}


\noindent\textbf{Problem 3. }
Let $(X, \mcal{A}, \mu)$ be a finite measure space. Assume that $E$ is a closed subspace of $L^2(X, \mu)$, and that $E$ is contained in $L^\infty(X, \mu)$. Prove that $E$ is finite dimensional.
\begin{proof}
	By H\"older's inequality, the embedding $\iota: L^\infty(X)\to L^2(X)$, $f\mapsto f$ is continuous, i.e., $\Lp{f}{\infty}\leq C\cdot \Lp{f}{2}$ for some finite $C>0$. Let $e_1, \ldots e_n$ be an orthonormal set in $E$. 
\end{proof}

\end{document}