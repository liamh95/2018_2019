%% Please change the file name by replacing N with the apporpriate number
%% corresponding to the current homework and XX with your initials.
%% https://www.math.uci.edu/~gpatrick/jsOnline/hw1.html

\documentclass[11pt,letterpaper]{report}
\usepackage{amssymb,amsfonts,color,graphicx,amsmath,enumerate,array}
\usepackage{tikz} %This package offers the ability to draw pictures
\usepackage{amsthm}

\newcommand{\naturals}{\mathbb{N}}
\newcommand{\integers}{\mathbb{Z}}
\newcommand{\complex}{\mathbb{C}}
\newcommand{\reals}{\mathbb{R}}
\newcommand{\exreals}{\overline{\mathbb{R}}}
\newcommand{\mcal}[1]{\mathcal{#1}}
\newcommand{\mable}{measurable}
\newcommand{\quats}{\mathbb{H}}
\newcommand{\rationals}{\mathbb{Q}}
\newcommand{\norm}{\trianglelefteq}
\newcommand{\Aut}{\text{Aut}}
\newcommand{\disk}{\mathbb{D}}
\newcommand{\halfplane}{\mathbb{H}}
\newcommand{\Lp}[2]{\left\|{#1}\right\|_{L^{#2}}}
\newcommand{\supp}[1]{\text{supp}({#1})}
\newcommand{\Hom}[2]{\text{Hom}_{{#1}}({#2})}
\newcommand{\tr}{\text{tr}}
\newcommand{\field}[1]{\mathbb{F}_{{#1}}}
\newcommand{\Gal}[1]{\text{Gal}\left({#1}\right)}
\newcommand{\esssup}{\text{ess sup }}
\newcommand{\essinf}{\text{ess inf }}
\newcommand{\affine}{\mathbb{A}}
\newcommand{\torus}{\mathbb{T}}

\newcolumntype{L}{>{$}l<{$}} % math-mode version of "l" column type
\newcolumntype{C}{>{$}c<{$}} % math-mode version of "l" column type

\newenvironment{solution}
{\begin{proof}[Solution]}
{\end{proof}}

\voffset=-3cm
\hoffset=-2.25cm
\textheight=24cm
\textwidth=17.25cm
\addtolength{\jot}{8pt}
\linespread{1.3}

\begin{document}
\noindent{\em Liam Hardiman\hfill{November 26, 2018} }
% Please give relevant information
\begin{center}
{\bf \Large 260A - Homework 3} %Replace N with the appropriate number
\vspace{0.2cm}
\hrule
\end{center}

\noindent\textbf{Problem 1. }
Let $(b_1, b_2, \ldots)$ be a sequence of complex numbers such that $\sum_{n=1}^\infty b_nc_n$ is convergent for every $c = (c_1, c_2, \ldots)\in \ell^2$. Show that $b\in \ell^2$.
\begin{proof}
	Consider the sequence of maps $T_n: \ell^2\to \complex$ that send $(c_1, \ldots)$ to $\sum_{j=1}^n b_jc_j$. Since each $T_n$ is just a finite sum, we have that the $T_n$'s form a sequence of bounded linear operators on $\ell^2$. Furthermore, this sequence is pointwise bounded: given any $(c_1, c_2, \ldots)\in \ell^2$, since $\sum_{j=1}^\infty b_jc_j$ converges, we have that the sequence of partial sums $|T_n(c_1, c_2, \ldots)| = |\sum_{j=1}^n b_jc_j|$ is bounded. By the uniform boundedness principle, we have that
	\[
	\sup_{n\in \naturals,\ \|(c_1, c_2, \ldots)\|_2 = 1}|T_n(c_1, c_2, \ldots)| = \sup_{n\in \naturals}\|T_n\| = \sum_{j=1}^\infty |b_j|<\infty,
	\]
	so $(b_1, b_2, \ldots)\in \ell^2$.
\end{proof}

\noindent\textbf{Problem 2. }
Let $M$ be a measurable subset of $\reals^n$ with finite positive measure. Prove that $L^q(M)$ is of the first category in $L^p(M)$ if $1\leq p<q\leq \infty$.
\begin{proof}
	Since $M$ has finite measure, we have that $L^q(M)\subseteq L^p(M)$ whenever $1\leq p<q\leq \infty$. Consider the injection $\iota: L^q(M)\to L^p(M)$ that simply sends $f\in L^p(M)$ to itself. By the generalized H\"older inequality we have that $\Lp{\iota(f)}{p} = \Lp{f}{p}\leq \mu(M)^{1/r}\Lp{f}{q}$, where $\frac{1}{p} = \frac{1}{q} + \frac{1}{r}$. This shows that $\iota$ is bounded, and therefore continuous. Since $L^p(M)$ and $L^q(M)$ are Banach spaces, the open mapping theorem tells us that the image of $\iota$ is either surjective and open or of the first category in $L^p(M)$.\\

	\noindent Our plan is to show that $\iota$ is \textit{not} surjective, i.e. that $L^p(M)\setminus L^q(M)$ is nonempty. We'll do this by showing that $L^q(M) = L^p(M)$ would forbid the existence of subsets of $M$ with arbitrarily small measure. Since sets of positive measure in $\mathbb{R}^n$ \textit{do} contain sets of arbitrarily small measure, we'll conclude that $L^q(M)\neq L^p(M)$.\\

	\noindent If the embedding $\iota: L^q(M)\to L^p(M)$ were surjective, then the open mapping theorem would imply that $\iota$ is actually a homeomorphism. In particular, its inverse, $\iota': L^p(M)\to L^q(M)$ is a bounded operator. Let $A$ be a subset of $M$ with positive, finite measure and define the function
	\[
	f_A(x) = \frac{1}{m(A)^{1/p}}\cdot \chi_A(x).
	\]
	It's clear that $\Lp{f_A}{p} = 1$ and that $\Lp{f_A}{q} = \frac{1}{m(A)^{1/p-1/q}}$. Since $\iota'$ is bounded, we have
	\[
	0<\Lp{f_A}{q} \leq \|\iota'\|\cdot \Lp{f_A}{p} \implies 0<\frac{1}{\|\iota'\|^{\frac{pq}{q-p}}}\leq m(A).
	\]
	This puts a positive lower bound on the measure of subsets of $M$. But $M$, as a subset of $\reals^n$ with positive measure, contains set of arbitrarily small measure. We conclude that $L^q(M)$ is of the first category of $L^p(M)$.
\end{proof}


\noindent\textbf{Problem 3. }
Let $(X, \mcal{A}, \mu)$ be a finite measure space. Assume that $E$ is a closed subspace of $L^2(X, \mu)$, and that $E$ is contained in $L^\infty(X, \mu)$. Prove that $E$ is finite dimensional.
\begin{proof}
	By H\"older's inequality, the embedding $\iota: L^\infty(X)\to L^2(X)$, $f\mapsto f$ is continuous, i.e., $\Lp{f}{2}\leq \mu(X)^{1/2}\cdot \Lp{f}{\infty}$. When we restrict $\iota$ to $E$ we obtain a continuous surjection from $E$ to itself. By the open mapping theorem, $\iota:E\to E$ is a homeomorphism, so there is some positive $C>0$ with $\Lp{f}{\infty}\leq C\cdot \Lp{f}{2}$ for any $f\in E$. Now let $e_1, \ldots, e_n$ be an orthonormal set in $E$ and fix $a\in \complex^n$. Then for all $x$ in $S_a$, where $S_a$ has $\mu$-full measure in $X$, we have 
	\begin{align*}
	|a_1e_1(x) + \cdots + a_ne_n(x)|^2 &\leq \Lp{a_1e_1 + \cdots + a_ne_n}{\infty}^2\\
	&\leq C^2\cdot \Lp{a_1e_1 + \cdots a_ne_n}{2}^2\\
	&= C^2\cdot(|a_1|^2 + \cdots + |a_n|^2).
	\end{align*}
	We'd like to replace the $a_i$'s with $\overline{e_i(x)}$'s, but here $x$ depends on $a$. We accomplish this through a limiting process (Alec Fox showed me how to do this).\\

	\noindent Let $Q$ be a countable dense subset of $\complex^n$. The intersection $S := \cap_{q\in Q}S_q$ has full measure in $X$. Now for any $a\in \complex^n$, we can find a sequence $q^{(k)}$ in $Q$ that limits to $a$. For any $k$ and $x\in S$ we have by the above inequalities
	\[
	\left|\sum_{j=1}^nb_j^{(k)}e_j(x)\right|^2 \leq C^2\cdot \sum_{j=1}^n|b_j^{(k)}|^2.
	\]
	Taking the limit $k\to \infty$ gives
	\[
	\left|\sum_{j=1}^na_je_j(x)\right|^2 \leq C^2\cdot \sum_{j=1}^n|a_j|^2.
	\]
	Now for any $x\in S$, which is $\mu$-almost all of $X$, we can substitute $a_j = \overline{e_j(x)}$ into the above inequality to obtain (by the orthonormality of the $e_j$'s)
	\[
	\sum_{j=1}^n|e_j(x)|^2 \leq C^2.
	\]
	Integration gives
	\[
	n = \sum_{j=1}^n\int_X|e_j(x)|^2\ d\mu \leq \int_XC^2\ d\mu = C^2\cdot \mu(X)<\infty,
	\]
	so $E$ is finite-dimensional.
\end{proof}

\noindent\textbf{Problem 4. }
Let $X$ be a locally compact and locally convex space.
\begin{enumerate}[(i)]
	\item Let $U$ be a compact neighborhood of the origin. Show that one can find $x_1, \ldots, x_n$ so that $U\subseteq \cup_{j=1}^n(x_j + \frac{1}{2}U)$, and thus, a finite dimensional space, $M$, with $U\subseteq M+\frac{1}{2}U$.
	\begin{proof}
		Cover $U$ with dilates of itself: $U\subseteq \cup_{x\in U}(x+\frac{1}{2}U^\circ)$. This is indeed an open cover since $U$, as a compact neighborhood, has nonempty interior. By compactness, we can extract a finite subcover, based around the points $x_1, \ldots, x_n$:
		\begin{align*}
			U \subseteq \bigcup_{j=1}^nx_j+\frac{1}{2}U^\circ\subseteq \bigcup_{j=1}^nx_j+\frac{1}{2}U.
		\end{align*}
		Let $M$ be the linear span of the $x_j$'s, $M:= \langle x_1, \ldots, x_j\rangle$. Since $M$ is the span of finitely many vectors, it is finite dimensional and we clearly have the inclusion
		\[
		U\subseteq \bigcup_{j=1}^nx_j+\frac{1}{2}U\subseteq M+\frac{1}{2}U.
		\] 
	\end{proof}

	\item Prove that $U\subseteq M+\frac{1}{2^m}U$ for any $m$.
	\begin{proof}
		In the above construction, it would appear the our choice of finite dimensional space, $M$, depends on our choice of cover. An induction on $m$ will show that it doesn't. The base case $m=1$ follows from part (i). Now assume that $U\subseteq M+\frac{1}{2^m}U$. We dilate both sides of this inclusion to obtain
		\[
		\frac{1}{2}U\subseteq \frac{1}{2}M + \frac{1}{2^{m+1}}U = M+\frac{1}{2^{m+1}}U.
		\]
		By part (i) we then have
		\[
		U\subseteq M+\frac{1}{2}U\subseteq M + \left(M + \frac{1}{2^{m+1}}U\right) = M +\frac{1}{2^{m+1}}U.
		\]
		By induction, the proposition holds for all $m$.
	\end{proof}

	\item Prove that $U\subseteq \overline{M}$.
	\begin{proof}
		Take $x\notin \overline{M}$. Then there is some balanced neighborhood of the origin, $V$, with $x\notin M+V$. Since balanced neighborhoods are absorbing, we have that $U\subseteq \cup_{n=1}^\infty nV$. The compactness of $U$ and the fact that this union is increasing (since $V$ is balanced) tells us that $U\subseteq 2^NV$ for some large $N$. By part (ii) we have
		\begin{align*}
		U &\subseteq M+\frac{1}{2^N}U\\
		&\subseteq M+\frac{1}{2^N}(2^NV)\\
		&= M+V.
		\end{align*}
		Since $x\notin M+V$, we conclude that $x\notin U$. This shows that $U\subseteq \overline{M}$.
	\end{proof}

	\item Conclude that $\overline{M} = X = M$.
	\begin{proof}
		
	\end{proof}
\end{enumerate}

\noindent\textbf{Problem 5. }
Let $a_n$, $n\in \integers$, be a sequence of complex numbers such that $a_nb_n$ is the sequence of Fourier coefficients of a continuous function on $\reals/2\pi\integers$ when this is true for the sequence $b_n$, $n\in \integers$. Prove that there is a measure with Fourier coefficients $a_n$, $n\in \integers$.
\begin{proof}
	Denote $\reals/2\pi \integers$ by $\torus$. The plan is to use the closed graph theorem and Riesz-Markov-Kakutani. Define the map $T: C(\torus)\to C(\torus)$ that maps $f$ to the continuous function with Fourier coefficients $a_n\widehat{f}(-n)$ (the reason for the negative sign will become clear). That $T$ is well defined follows from the fact that the assignment of a continuous function to its Fourier coefficients is injective and from the hypothesis that $a_n\widehat{f}(-n)$ is indeed the set of Fourier coefficients of a continuous function.\\

	\noindent That $T$ is linear follows simply from the linearity of the integral. We'll use the closed graph theorem to show that $T$ is continuous. Suppose that $f_j\to f$ in $C(\torus)$ and $Tf_j\to g$ in $C(\torus)$. Let's look at the Fourier coefficients of $g$.
	\begin{align*}
		\widehat{g}(n) &= \frac{1}{2\pi}\int_\torus g(x)e^{-inx}\ dx\\
		&= \lim_{j\to \infty}\frac{1}{2\pi}\int_\torus (Tf_j)(x)e^{-inx}\ dx\\
		&= \lim_{j\to \infty}\widehat{Tf_j}(n)\\
		&= a_n\cdot \lim_{j\to \infty}\widehat{f_j}(-n)\\
		&= a_n\cdot \lim_{j\to \infty}\frac{1}{2\pi}\int_\torus f_j(x)e^{inx}\ dx\\
		&= a_n\cdot \frac{1}{2\pi}\int_\torus f(x)e^{inx}\ dx\\
		&= a_n\widehat{f}(-n)\\
		&= \widehat{Tf}(n).
	\end{align*}
	The movement of limits through integrals follows from the fact that convergence in $C(\torus)$ is uniform on a finite measure space. Again by the uniqueness of Fourier coefficients for continuous functions, we have that $Tf = g$, so by the closed graph theorem, $T$ is continuous.\\

	\noindent Now define the map $S: C(\torus)\to \complex$ by $Sf = (Tf)(0)$. Evaluation at zero is a continuous linear functional on $C(\torus)$ and the composition of continuous functions is continuous, so $S$ is a continuous linear functional on $C(\torus)$. By the Riesz-Markov-Kakutani representation theorem, there is a unique regular Borel measure $\mu$ on $\torus$ such that $Sf = \int_\torus f\ d\mu$. This measure has the desired property since its Fourier coefficients are given by
	\begin{align*}
		\int_\torus e^{-inx}\ d\mu &= S(e^{-inx})\\
		&= (a_ne^{inx})(0)\\
		&= a_n.
	\end{align*}
\end{proof}

\end{document}