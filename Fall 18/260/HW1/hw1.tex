%% Please change the file name by replacing N with the apporpriate number
%% corresponding to the current homework and XX with your initials.
%% https://www.math.uci.edu/~gpatrick/jsOnline/hw1.html

\documentclass[11pt,letterpaper]{report}
\usepackage{amssymb,amsfonts,color,graphicx,amsmath,enumerate}
\usepackage{tikz} %This package offers the ability to draw pictures
\usepackage{amsthm}

\newcommand{\naturals}{\mathbb{N}}
\newcommand{\integers}{\mathbb{Z}}
\newcommand{\complex}{\mathbb{C}}
\newcommand{\reals}{\mathbb{R}}
\newcommand{\exreals}{\overline{\mathbb{R}}}
\newcommand{\mcal}[1]{\mathcal{#1}}
\newcommand{\mable}{measurable}
\newcommand{\quats}{\mathbb{H}}
\newcommand{\rationals}{\mathbb{Q}}
\newcommand{\norm}{\trianglelefteq}
\newcommand{\Aut}{\text{Aut}}
\newcommand{\disk}{\mathbb{D}}
\newcommand{\halfplane}{\mathbb{H}}
\newcommand{\Lp}[2]{\left\|{#1}\right\|_{L^{#2}}}
\newcommand{\supp}[1]{\text{supp}({#1})}
\newcommand{\Hom}[2]{\text{Hom}_{{#1}}({#2})}
\newcommand{\tr}{\text{tr}}
\newcommand{\field}[1]{\mathbb{F}_{{#1}}}
\newcommand{\Gal}[1]{\text{Gal}\left({#1}\right)}
\newcommand{\esssup}{\text{ess sup }}
\newcommand{\essinf}{\text{ess inf }}
\newcommand{\affine}{\mathbb{A}}
\newcommand{\pnorm}[2]{\left\|{#1}\right\|_{{#2}}}

\newenvironment{solution}
{\begin{proof}[Solution]}
{\end{proof}}

\voffset=-3cm
\hoffset=-2.25cm
\textheight=24cm
\textwidth=17.25cm
\addtolength{\jot}{8pt}
\linespread{1.3}

\begin{document}
\noindent{\em Liam Hardiman\hfill{October 26, 2018} }
% Please give relevant information
\begin{center}
{\bf \Large 260A - Homework 1} %Replace N with the appropriate number
\vspace{0.2cm}
\hrule
\end{center}

\noindent\textbf{Problem 1. }
\begin{enumerate}[(i)]
	\item Show that $\ell^p$, $1\leq p\leq \infty$, is a Banach space.
	\item Prove that $\ell^\infty = (\ell^1)^*$, but $(\ell^\infty)^*\neq \ell^1$.
\end{enumerate}
\begin{proof}
	\begin{enumerate}[(i)]
		\item Let $a = (a^{(n)})$ and $b = (b^{(n)})$ be in $\ell^p$, $1<p<\infty$. We have by H\"older's inequality for any complex $\lambda$
		\begin{align*}
			\pnorm{a+\lambda b}{p}^p &= \sum_{n=1}^\infty|a^{(n)}+\lambda b^{(n)}|^p\\
			&= \sum_{n=1}^\infty|a^{(n)}+\lambda b^{(n)}|\cdot |a^{(n)} + \lambda b^{(n)}|^{p-1}\\
			&\leq \sum_{n=1}^\infty |a^{(n)}|\cdot |a^{(n)}+\lambda b^{(n)}|^{p-1} + |\lambda|\sum_{n=1}^\infty|b^{(n)}|\cdot |a^{(n)}+\lambda b^{(n)}|^{p-1}\\
			&\leq (\pnorm{a}{p} + |\lambda|\pnorm{b}{p})\left(\sum_{n=1}^\infty |a^{(n)}+\lambda b^{(n)}|^{(p-1)\frac{p}{p-1}}\right)^{\frac{p-1}{p}}\\
			&= (\pnorm{a}{p} + |\lambda|\pnorm{b}{p})\pnorm{a+\lambda b}{p}^{p-1},
		\end{align*}
		Which shows that $\pnorm{a+\lambda b}{p}\leq \pnorm{a}{p}+|\lambda|\pnorm{b}{p}<\infty$. This shows both that $\ell^p$, $1<p<\infty$, is a vector space (as linear combinations of elements of $\ell^p$ have finite $p$-norm) and that the $p$-norm satisfies the triangle inequality (take $\lambda = 1$).\\
		$\ell^1$ is a vector space and the $\|\cdot \|_1$ norm satisfies the triangle inequality thanks to the triangle inequality on $\complex$:
		\begin{align*}
			\pnorm{a+\lambda b}{1} &= \sum_{n=1}^\infty |a^{(n)}+\lambda b^{(n)}|\\
			&\leq \sum_{n=1}^\infty|a^{(n)}| + |\lambda|\sum_{n=1}^\infty |b^{(n)}|\\
			&= \pnorm{a}{p} + |\lambda|\pnorm{b}{p}.
		\end{align*}
		Similarly, for $a,b\in \ell^\infty$ and $\lambda\in \complex$ we have
		\begin{align*}
			\pnorm{a+\lambda b}{\infty} = \sup_{n\geq 1}|a^{(n)}+\lambda b^{(n)}| \leq \sup_{n\geq 1}(|a^{(n)}| + |\lambda||b^{(n)}|)\leq \sup_{n\geq 1}|a^{(n)}| + |\lambda|\sup_{n\geq 1}|b^{(n)}| = \pnorm{a}{\infty}+|\lambda|\pnorm{b}{\infty}.
		\end{align*}
		We then have that $\ell^p$ is a normed complex vector space. We now need to show completeness. First let's treat the case of $p<\infty$. Suppose that $\{a_n\}$ is a Cauchy sequence in $\ell^p$ (here $a_i^{(j)}$ is the $j$-th entry in the $i$-th element of the sequence). Since this sequence is Cauchy we have that for any $\epsilon>0$ we can find $N\in \naturals$ so that for all $m,n>N$
		\[
		\pnorm{a_m-a_n}{p}<\epsilon \iff \sum_{k=1}^\infty |a_m^{(k)} - a_n^{(k)}|^p<\epsilon^p.
		\]
		Since each term in the above sum is nonnegative, we must have that $|a_m^{(k)} - a_n^{(k)}|<\epsilon$ for each $k$. In particular, we have that for any fixed $k$, $\{a_n^{(k)}\}$ is a Cauchy sequence of complex numbers. Since $\complex$ is complete, we have that $a_n^{(k)}\to a^{(k)}\in \complex$ as $n\to \infty$.\\
		Let $a$ be the sequence of complex numbers whose $k$-th entry is built from our original Cauchy sequence by $a^{(k)} = \lim_{n\to \infty}a_n^{(k)}$. Our plan is to show that $a_n\to a$ in $\ell^p$ and that $a$ is in $\ell^p$. Fix $\epsilon>0$. Then for some $N$ we have that $\pnorm{a_m - a_n}{p}<\epsilon$ for all $m,n>N$. Our trick is to pass to a finite sum and then take limits in a particular order. For any $L>0$ and $m,n$ sufficiently large we have
		\[
		\sum_{k=0}^L|a_m^{(k)} - a_n^{(k)}|^p \leq \pnorm{a_m-a_n}{p}^p < \epsilon^p.
		\]
		Now the right-hand side does not depend on $m$, so taking $m\to \infty$ gives
		\[
		\sum_{k=0}^L|a^{(k)} - a_n^{(k)}|^p <\epsilon^p.
		\]
		Then we take $L\to \infty$ which gives $\pnorm{a-a_n}{p}<\epsilon$, so $a_n\to a$ in $\ell^p$. We can use this to show that $a$ is in $\ell^p$ since for all $n$
		\begin{align*}
			\pnorm{a}{p} &\leq \pnorm{a-a_n}{p} + \pnorm{a_n}{p}.
		\end{align*}
		For $n$ large enough the first term on the right is bounded by $\epsilon$ and the second term is finite since each $a_n$ is in $\ell^p$. Thus, $\ell^p$ is complete, and therefore, a Banach space for $1\leq p<\infty$.\\
		Now let $p = \infty$. If $\{a_n\}$ is a Cauchy sequence in $\ell^\infty$ then for $\epsilon>0$ and $m,n$ sufficiently large we have that $\sup_{k>0}|a_m^{(k)} - a_n^{(k)}|<\epsilon$. Just like in the finite $p$ case, this implies that for any fixed $k$, $\{a_n^{(k)}\}$ is a Cauchy sequence of complex numbers, so we can speak of the entrywise limit $a$. Also similar to the finite $p$ case we have that for $L$ large
		\[
		\sup_{1\leq k\leq L}|a_m^{(k)}-a_n^{(k)}|\leq \pnorm{a_m-a_n}{\infty}<\epsilon.
		\]
		Sending $m$ to infinity gives $\sup_{1\leq k\leq L}|a^{(k)}-a_n^{(k)}|<\epsilon$ and then sending $L$ to infinity gives $\pnorm{a-a_n}{\infty}\to 0$. The argument used in the $p<\infty$ case also shows that $a\in \ell^\infty$.

		\item First we'll show that $(\ell^1)^* = \ell^\infty$ (i.e., they are isometrically isomorphic). 
		% What we're really claiming is that given $T\in (\ell^1)^*$, we can find exactly one $b\in \ell^\infty$ so that for any $a\in \ell^1$ we have $Ta = \sum a^{(k)}\overline{b^{(k)}}$. The uniqueness of $b$ is straightforward to show. Suppose $T$ were generated by $b$ and $c$ in $\ell^\infty$. Then their difference should generate the zero functional:
		% \[
		% 0 = Ta - Ta = \sum_{k=1}^\infty a^{(k)}\overline{(b^{(k)}-c^{(k)})}
		% \]
		% for all $a\in \ell^1$. In particular, if we let $a_n$ be the sequence whose $n$-th entry is 1 and whose other entries are 0, we see that $b^{(k)} = c^{(k)}$ for all $k$.
		Let $\varphi: \ell^\infty\to (\ell^1)^*$ be the map that sends $b\in \ell^\infty$ to $T_b$, where $T_b(a) = \sum_{k=1}^\infty a^{(k)}b^{(k)}$. That $\varphi$ is linear is obvious. By H\"older's inequality we have that
		\[
		|T_b(a)| \leq \sum_{k=1}^\infty \left|a^{(k)}\right|\left|b^{(k)}\right| \leq \pnorm{a}{1}\cdot \pnorm{b}{\infty},
		\]
		This shows that $T_b$ is bounded, and therefore continuous, so the image of $\varphi$ indeed lives in $(\ell^1)^*$. In particular, this shows that $\|\varphi(b)\|\leq \pnorm{b}{\infty}$ (so $\varphi$ is a continuous map of vector spaces). To show that $\varphi$ is an isometry, we need the reverse inequality.\\
		Since $\pnorm{b}{\infty} = \sup_{k\geq 1}|b^{(k)}|$, for any $\epsilon>0$, we can find a natural number $N$ so that $|b^{(N)}|>\pnorm{b}{\infty}-\epsilon$. Consequently, if we let $e_n$ be the sequence in $\ell^1$ whose $n$-th entry is 1 and whose other entries are 0, we have that we can always find $N$ so that $|T_b(e_N)| = |b^{(N)}| >\pnorm{b}{\infty}-\epsilon$. Since $\epsilon$ was arbitrary and $\pnorm{e_n}{1} = 1$, we have that $\pnorm{T_b}{\infty} \geq \pnorm{b}{\infty}$. Thus, $\|\varphi(b)\| = \pnorm{b}{\infty}$ and $\varphi$ is an isometry.\\
		Since isometries are injective, it remains to show that $\varphi$ is surjective. Let $T$ be a functional in $(\ell^1)^*$. For any $a\in \ell^1$ we have that $a= \sum_{k=1}^\infty a^{(k)}e_k$ where $\sum |a^{(k)}|<\infty$ and $e_k$ is as it was above. Since $a = \lim_{N\to \infty}\sum_{k=1}^Na^{(k)}e_k$, continuity of $T$ tells us that
		\[
		T(a) = T\left(\sum_{k=1}^\infty a^{(k)}e_k\right) = \sum_{k=1}^\infty a^{(k)}T(e_k).
		\]
		Since continuity is equivalent to boundedness, we have that $|T(e_k)| <M<\infty$ for some $M$. Thus, $T$ is the image of the bounded sequence sequence $(T(e_1), T(e_2), \ldots )$ under $\varphi$, so $\varphi$ is surjective. $\varphi$ is then a surjective isometry $\ell^\infty \to (\ell^1)^*$.\\\\
		Now let's show that $(\ell^\infty)^* \neq \ell^1$. Let $S$ be the subspace of $\ell^\infty$ consisting of all convergent sequences and let $T:S\to \complex$ be the map that sends a convergent sequence to its limit. $T$ is clearly linear and it's bounded since
		\[
		|T(a)| = |\lim_{k\to \infty} a^{(k)}| \leq \limsup_{k\to \infty}|a^{(k)}| \leq \sup_{k\geq 1}|a^{(k)}| = \pnorm{a}{\infty}.
		\]
		By the Hahn-Banach theorem, $T$ extends to a continuous linear functional $\tilde{T}$ on all of $\ell^\infty$ that agrees with $T$ on $S$.\\
		If $\tilde{T}(a)$ could be written $\tilde{T}(a) = \sum_{k=1}^\infty a^{(k)}b^{(k)}$ for some $b\in \ell^1$, then for all $n$ we would have $b^{(n)} = \tilde{T}(e_n) = T(e_n) = 0$. But then $b$ would be the zero sequence and $\tilde{T}$ is the zero functional, which is nonsense since $\tilde{T}(1, 1, \ldots) = T(1, 1, \ldots) = 1$. We conclude that $\tilde{T}$ does not have the form required for $(\ell^\infty)^* = \ell^1$.
	\end{enumerate}
\end{proof}

\noindent\textbf{Problem 2}
Prove that if $Z$ is a subspace of a normed linear space $X$, and $y\in X$ has distance $d$ from $Z$, then there exists $\Lambda\in X^*$ such that $\|\Lambda\|\leq 1$, $\Lambda(y) = d$ and $\Lambda(z) =0$ for all $z\in Z$.
\begin{proof}
	Consider the subspace $Y = Z\oplus ky$ of $X$, where $k$ the field over which $X$ is defined. This sum is indeed direct since $y$ is not in $Z$. Define the function $f: Y\to \reals$ by $f(z + \alpha y) = \alpha d$. $f$ is linear since
	\begin{align*}
		f[\gamma(z+\alpha y) + (w + \beta y)] &= f[(w+\gamma z) + (\beta+\gamma\alpha)y]\\
		&= (\beta+\gamma\alpha)d\\
		&= \gamma f(z+\alpha y) + f(w+\beta y).
	\end{align*}
	We claim that $|f(z+\alpha y)|\leq \|z+\alpha y\|$. Intuitively, this is because $|f(z+\alpha y)|$ is the distance from $z+\alpha y$ to $Z$, which is at most $\|z+\alpha y\|$, since $0\in Z$. Rigorously, since $0\in Z$ we have
	\begin{align*}
		|f(z+\alpha y)| &= |\alpha\cdot d|\\
		&= |\alpha|\cdot \inf_{w\in Z}\|y - w\|\\
		&= \inf_{w\in Z}\|\alpha y + z - w\|\\
		&\leq \|\alpha y+z - 0\|\\
		&= \|\alpha y + z\|.
	\end{align*}
	By the Hahn-Banach theorem, $f$ extends to a continuous (as $|f(x)|<\|x\|$ on $Y$) linear function $\Lambda$ on all of $X$ that also satisfies $|\Lambda(x)|\leq \|x\|$. This gives $\|\Lambda\|\leq 1$. Furthermore, since $\Lambda$ agrees with $f$ on $Y$, we have that $\Lambda(y) = f(y) = d$ and $\Lambda(z) = f(z) = f(z+0y) = 0$ for all $z\in Z$.
\end{proof}

\noindent\textbf{Problem 3. }
Show that linear combinations of functions of the form
\[
\reals \ni t\mapsto \frac{1}{t-z},\quad \text{Im}(z)\neq 0
\]
are dense in the space of continuous functions on $\reals$ which tend to zero at infinity.
\begin{proof}
	
\end{proof}

\end{document}