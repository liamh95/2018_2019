%% Please change the file name by replacing N with the apporpriate number
%% corresponding to the current homework and XX with your initials.
%% https://www.math.uci.edu/~gpatrick/jsOnline/hw1.html

\documentclass[11pt,letterpaper]{report}
\usepackage{amssymb,amsfonts,color,graphicx,amsmath,enumerate}
\usepackage{tikz} %This package offers the ability to draw pictures
\usepackage{amsthm}

\newcommand{\naturals}{\mathbb{N}}
\newcommand{\integers}{\mathbb{Z}}
\newcommand{\complex}{\mathbb{C}}
\newcommand{\reals}{\mathbb{R}}
\newcommand{\exreals}{\overline{\mathbb{R}}}
\newcommand{\mcal}[1]{\mathcal{#1}}
\newcommand{\mable}{measurable}
\newcommand{\quats}{\mathbb{H}}
\newcommand{\rationals}{\mathbb{Q}}
\newcommand{\norm}{\trianglelefteq}
\newcommand{\Aut}{\text{Aut}}
\newcommand{\disk}{\mathbb{D}}
\newcommand{\halfplane}{\mathbb{H}}
\newcommand{\Lp}[2]{\left\|{#1}\right\|_{L^{#2}}}
\newcommand{\supp}[1]{\text{supp}({#1})}
\newcommand{\Hom}[2]{\text{Hom}_{{#1}}({#2})}
\newcommand{\tr}{\text{tr}}
\newcommand{\field}[1]{\mathbb{F}_{{#1}}}
\newcommand{\Gal}[1]{\text{Gal}\left({#1}\right)}
\newcommand{\esssup}{\text{ess sup }}
\newcommand{\essinf}{\text{ess inf }}
\newcommand{\affine}{\mathbb{A}}
\newcommand{\pnorm}[2]{\left\|{#1}\right\|_{{#2}}}

\newenvironment{solution}
{\begin{proof}[Solution]}
{\end{proof}}

\voffset=-3cm
\hoffset=-2.25cm
\textheight=24cm
\textwidth=17.25cm
\addtolength{\jot}{8pt}
\linespread{1.3}

\begin{document}
\noindent{\em Liam Hardiman\hfill{October 26, 2018} }
% Please give relevant information
\begin{center}
{\bf \Large 260A - Homework 1} %Replace N with the appropriate number
\vspace{0.2cm}
\hrule
\end{center}

\noindent\textbf{Problem 1. }
\begin{enumerate}[(i)]
	\item Show that $\ell^p$, $1\leq p\leq \infty$, is a Banach space.
	\item Prove that $\ell^\infty = (\ell^1)^*$, but $(\ell^\infty)^*\neq \ell^1$.
\end{enumerate}
\begin{proof}
	\begin{enumerate}[(i)]
		\item Let $a = (a^{(n)})$ and $b = (b^{(n)})$ be in $\ell^p$, $1<p<\infty$. We have by H\"older's inequality for any complex $\lambda$
		\begin{align*}
			\pnorm{a+\lambda b}{p}^p &= \sum_{n=1}^\infty|a^{(n)}+\lambda b^{(n)}|^p\\
			&= \sum_{n=1}^\infty|a^{(n)}+\lambda b^{(n)}|\cdot |a^{(n)} + \lambda b^{(n)}|^{p-1}\\
			&\leq \sum_{n=1}^\infty |a^{(n)}|\cdot |a^{(n)}+\lambda b^{(n)}|^{p-1} + |\lambda|\sum_{n=1}^\infty|b^{(n)}|\cdot |a^{(n)}+\lambda b^{(n)}|^{p-1}\\
			&\leq (\pnorm{a}{p} + |\lambda|\pnorm{b}{p})\left(\sum_{n=1}^\infty |a^{(n)}+\lambda b^{(n)}|^{(p-1)\frac{p}{p-1}}\right)^{\frac{p-1}{p}}\\
			&= (\pnorm{a}{p} + |\lambda|\pnorm{b}{p})\pnorm{a+\lambda b}{p}^{p-1},
		\end{align*}
		Which shows that $\pnorm{a+\lambda b}{p}\leq \pnorm{a}{p}+|\lambda|\pnorm{b}{p}<\infty$. This shows both that $\ell^p$, $1<p<\infty$, is a vector space (as linear combinations of elements of $\ell^p$ have finite $p$-norm) and that the $p$-norm satisfies the triangle inequality (take $\lambda = 1$).\\
		$\ell^1$ is a vector space and the $\|\cdot \|_1$ norm satisfies the triangle inequality thanks to the triangle inequality on $\complex$:
		\begin{align*}
			\pnorm{a+\lambda b}{1} &= \sum_{n=1}^\infty |a^{(n)}+\lambda b^{(n)}|\\
			&\leq \sum_{n=1}^\infty|a^{(n)}| + |\lambda|\sum_{n=1}^\infty |b^{(n)}|\\
			&= \pnorm{a}{p} + |\lambda|\pnorm{b}{p}.
		\end{align*}
		Similarly, for $a,b\in \ell^\infty$ and $\lambda\in \complex$ we have
		\begin{align*}
			\pnorm{a+\lambda b}{\infty} = \sup_{n\geq 1}|a^{(n)}+\lambda b^{(n)}| \leq \sup_{n\geq 1}(|a^{(n)}| + |\lambda||b^{(n)}|)\leq \sup_{n\geq 1}|a^{(n)}| + |\lambda|\sup_{n\geq 1}|b^{(n)}| = \pnorm{a}{\infty}+|\lambda|\pnorm{b}{\infty}.
		\end{align*}
		We then have that $\ell^p$ is a normed complex vector space. We now need to show completeness. First let's treat the case of $p<\infty$. Suppose that $\{a_n\}$ is a Cauchy sequence in $\ell^p$ (here $a_i^{(j)}$ is the $j$-th entry in the $i$-th element of the sequence). Since this sequence is Cauchy we have that for any $\epsilon>0$ we can find $N\in \naturals$ so that for all $m,n>N$
		\[
		\pnorm{a_m-a_n}{p}<\epsilon \iff \sum_{k=1}^\infty |a_m^{(k)} - a_n^{(k)}|^p<\epsilon^p.
		\]
		Since each term in the above sum is nonnegative, we must have that $|a_m^{(k)} - a_n^{(k)}|<\epsilon$ for each $k$. In particular, we have that for any fixed $k$, $\{a_n^{(k)}\}$ is a Cauchy sequence of complex numbers. Since $\complex$ is complete, we have that $a_n^{(k)}\to a^{(k)}\in \complex$ as $n\to \infty$.\\
		Let $a$ be the sequence of complex numbers whose $k$-th entry is built from our original Cauchy sequence by $a^{(k)} = \lim_{n\to \infty}a_n^{(k)}$. Our plan is to show that $a_n\to a$ in $\ell^p$ and that $a$ is in $\ell^p$. Fix $\epsilon>0$. Then for some $N$ we have that $\pnorm{a_m - a_n}{p}<\epsilon$ for all $m,n>N$. Our trick is to pass to a finite sum and then take limits in a particular order. For any $L>0$ and $m,n$ sufficiently large we have
		\[
		\sum_{k=0}^L|a_m^{(k)} - a_n^{(k)}|^p \leq \pnorm{a_m-a_n}{p}^p < \epsilon^p.
		\]
		Now the right-hand side does not depend on $m$, so taking $m\to \infty$ gives
		\[
		\sum_{k=0}^L|a^{(k)} - a_n^{(k)}|^p <\epsilon^p.
		\]
		Then we take $L\to \infty$ which gives $\pnorm{a-a_n}{p}<\epsilon$, so $a_n\to a$ in $\ell^p$. We can use this to show that $a$ is in $\ell^p$ since for all $n$
		\begin{align*}
			\pnorm{a}{p} &\leq \pnorm{a-a_n}{p} + \pnorm{a_n}{p}.
		\end{align*}
		For $n$ large enough the first term on the right is bounded by $\epsilon$ and the second term is finite since each $a_n$ is in $\ell^p$. Thus, $\ell^p$ is complete, and therefore, a Banach space for $1\leq p<\infty$.\\
		Now let $p = \infty$. If $\{a_n\}$ is a Cauchy sequence in $\ell^\infty$ then for $\epsilon>0$ and $m,n$ sufficiently large we have that $\sup_{k>0}|a_m^{(k)} - a_n^{(k)}|<\epsilon$. Just like in the finite $p$ case, this implies that for any fixed $k$, $\{a_n^{(k)}\}$ is a Cauchy sequence of complex numbers, so we can speak of the entrywise limit $a$. Also similar to the finite $p$ case we have that for $L$ large
		\[
		\sup_{1\leq k\leq L}|a_m^{(k)}-a_n^{(k)}|\leq \pnorm{a_m-a_n}{\infty}<\epsilon.
		\]
		Sending $m$ to infinity gives $\sup_{1\leq k\leq L}|a^{(k)}-a_n^{(k)}|<\epsilon$ and then sending $L$ to infinity gives $\pnorm{a-a_n}{\infty}\to 0$. The argument used in the $p<\infty$ case also shows that $a\in \ell^\infty$.

		\item First we'll show that $(\ell^1)^* = \ell^\infty$ (i.e., they are isometrically isomorphic). 
		% What we're really claiming is that given $T\in (\ell^1)^*$, we can find exactly one $b\in \ell^\infty$ so that for any $a\in \ell^1$ we have $Ta = \sum a^{(k)}\overline{b^{(k)}}$. The uniqueness of $b$ is straightforward to show. Suppose $T$ were generated by $b$ and $c$ in $\ell^\infty$. Then their difference should generate the zero functional:
		% \[
		% 0 = Ta - Ta = \sum_{k=1}^\infty a^{(k)}\overline{(b^{(k)}-c^{(k)})}
		% \]
		% for all $a\in \ell^1$. In particular, if we let $a_n$ be the sequence whose $n$-th entry is 1 and whose other entries are 0, we see that $b^{(k)} = c^{(k)}$ for all $k$.
		Let $\varphi: \ell^\infty\to (\ell^1)^*$ be the map that sends $b\in \ell^\infty$ to $T_b$, where $T_b(a) = \sum_{k=1}^\infty a^{(k)}b^{(k)}$. That $\varphi$ is linear is obvious. By H\"older's inequality we have that
		\[
		|T_b(a)| \leq \sum_{k=1}^\infty \left|a^{(k)}\right|\left|b^{(k)}\right| \leq \pnorm{a}{1}\cdot \pnorm{b}{\infty},
		\]
		This shows that $T_b$ is bounded, and therefore continuous, so the image of $\varphi$ indeed lives in $(\ell^1)^*$. In particular, this shows that $\|\varphi(b)\|\leq \pnorm{b}{\infty}$ (so $\varphi$ is a continuous map of vector spaces). To show that $\varphi$ is an isometry, we need the reverse inequality.\\
		Since $\pnorm{b}{\infty} = \sup_{k\geq 1}|b^{(k)}|$, for any $\epsilon>0$, we can find a natural number $N$ so that $|b^{(N)}|>\pnorm{b}{\infty}-\epsilon$. Consequently, if we let $e_n$ be the sequence in $\ell^1$ whose $n$-th entry is 1 and whose other entries are 0, we have that we can always find $N$ so that $|T_b(e_N)| = |b^{(N)}| >\pnorm{b}{\infty}-\epsilon$. Since $\epsilon$ was arbitrary and $\pnorm{e_n}{1} = 1$, we have that $\pnorm{T_b}{\infty} \geq \pnorm{b}{\infty}$. Thus, $\|\varphi(b)\| = \pnorm{b}{\infty}$ and $\varphi$ is an isometry.\\
		Since isometries are injective, it remains to show that $\varphi$ is surjective. Let $T$ be a functional in $(\ell^1)^*$. For any $a\in \ell^1$ we have that $a= \sum_{k=1}^\infty a^{(k)}e_k$ where $\sum |a^{(k)}|<\infty$ and $e_k$ is as it was above. Since $a = \lim_{N\to \infty}\sum_{k=1}^Na^{(k)}e_k$, continuity of $T$ tells us that
		\[
		T(a) = T\left(\sum_{k=1}^\infty a^{(k)}e_k\right) = \sum_{k=1}^\infty a^{(k)}T(e_k).
		\]
		Since continuity is equivalent to boundedness, we have that $|T(e_k)| <M<\infty$ for some $M$. Thus, $T$ is the image of the bounded sequence sequence $(T(e_1), T(e_2), \ldots )$ under $\varphi$, so $\varphi$ is surjective. $\varphi$ is then a surjective isometry $\ell^\infty \to (\ell^1)^*$.\\\\
		Now let's show that $(\ell^\infty)^* \neq \ell^1$. Let $S$ be the subspace of $\ell^\infty$ consisting of all convergent sequences and let $T:S\to \complex$ be the map that sends a convergent sequence to its limit. $T$ is clearly linear and it's bounded since
		\[
		|T(a)| = |\lim_{k\to \infty} a^{(k)}| \leq \limsup_{k\to \infty}|a^{(k)}| \leq \sup_{k\geq 1}|a^{(k)}| = \pnorm{a}{\infty}.
		\]
		By the Hahn-Banach theorem, $T$ extends to a continuous linear functional $\tilde{T}$ on all of $\ell^\infty$ that agrees with $T$ on $S$.\\
		If $\tilde{T}(a)$ could be written $\tilde{T}(a) = \sum_{k=1}^\infty a^{(k)}b^{(k)}$ for some $b\in \ell^1$, then for all $n$ we would have $b^{(n)} = \tilde{T}(e_n) = T(e_n) = 0$. But then $b$ would be the zero sequence and $\tilde{T}$ is the zero functional, which is nonsense since $\tilde{T}(1, 1, \ldots) = T(1, 1, \ldots) = 1$. We conclude that $\tilde{T}$ does not have the form required for $(\ell^\infty)^* = \ell^1$.
	\end{enumerate}
\end{proof}

\noindent\textbf{Problem 2}
Prove that if $Z$ is a subspace of a normed linear space $X$, and $y\in X$ has distance $d$ from $Z$, then there exists $\Lambda\in X^*$ such that $\|\Lambda\|\leq 1$, $\Lambda(y) = d$ and $\Lambda(z) =0$ for all $z\in Z$.
\begin{proof}
	Consider the subspace $Y = Z\oplus ky$ of $X$, where $k$ the field over which $X$ is defined. This sum is indeed direct since $y$ is not in $Z$. Define the function $f: Y\to \reals$ by $f(z + \alpha y) = \alpha d$. $f$ is linear since
	\begin{align*}
		f[\gamma(z+\alpha y) + (w + \beta y)] &= f[(w+\gamma z) + (\beta+\gamma\alpha)y]\\
		&= (\beta+\gamma\alpha)d\\
		&= \gamma f(z+\alpha y) + f(w+\beta y).
	\end{align*}
	We claim that $|f(z+\alpha y)|\leq \|z+\alpha y\|$. Intuitively, this is because $|f(z+\alpha y)|$ is the distance from $z+\alpha y$ to $Z$, which is at most $\|z+\alpha y\|$, since $0\in Z$. Rigorously, since $0\in Z$ we have
	\begin{align*}
		|f(z+\alpha y)| &= |\alpha\cdot d|\\
		&= |\alpha|\cdot \inf_{w\in Z}\|y - w\|\\
		&= \inf_{w\in Z}\|\alpha y + z - w\|\\
		&\leq \|\alpha y+z - 0\|\\
		&= \|\alpha y + z\|.
	\end{align*}
	By the Hahn-Banach theorem, $f$ extends to a continuous (as $|f(x)|<\|x\|$ on $Y$) linear function $\Lambda$ on all of $X$ that also satisfies $|\Lambda(x)|\leq \|x\|$. This gives $\|\Lambda\|\leq 1$. Furthermore, since $\Lambda$ agrees with $f$ on $Y$, we have that $\Lambda(y) = f(y) = d$ and $\Lambda(z) = f(z) = f(z+0y) = 0$ for all $z\in Z$.
\end{proof}

\noindent\textbf{Problem 3. }
Show that linear combinations of functions of the form
\[
\reals \ni t\mapsto \frac{1}{t-z},\quad \text{Im}(z)\neq 0
\]
are dense in the space of continuous functions on $\reals$ which tend to zero at infinity.
\begin{proof}
	Let $W$ be the the set of linear combinations of functions of the given form. We'd like to apply Stone-Weierstrass, but unfortunately, $W$ isn't a sub-algebra of $C_{(0)}(\reals)$ since it isn't closed under multiplication. Our plan is to make ourselves a sub-algebra.\\
	By the spanning criterion we have that the closure of $W$ in $C_{(0)}(\reals)$ is given by
	\[
	\overline{W} = \bigcap_{\substack{T\in C_{(0)}(\reals)^*\\T|_{W} = 0}} \ker T.
	\]
	Now by Riesz-Markov-Kakutani, we have that the dual space, $C_{(0)}(\reals)^*$, is the set of all complex Radon measures on $\reals$. It then suffices to show that for any $\mu \in C_{(0)}(\reals)^*$ that satisfies $\int_\reals \varphi\ d\mu = 0$ for all $\varphi\in W$, then $\int_\reals f\ d\mu = 0$ for all $f\in C_{(0)}(\reals)$.\\
	
	Let $\mu$ be a measure such that $\int \varphi\ d\mu$ for all $\varphi$ in $W$ and let $f(z) = \int_\reals \frac{1}{t+z}d\mu(t)$ for Im$(z)\neq 0$. By hypothesis, $f$ is identically zero. By dominated convergence, $f$ is infinitely differentiable with $f^{(n)}(z) = C_n\int_\reals \frac{1}{(t+z)^{n+1}}d\mu(t) = 0$ for some constant $C_n$ dependent on $n$.\\

	Now the set, $\mcal{A}$, of all linear combinations of functions of the form $t\mapsto \frac{1}{(t+z)^n}$ is an algebra of continuous functions that separates points and vanishes nowhere. By Stone-Weierstrass, their uniform closure is all of $C_{(0)}(\reals)$.
	Since any function in $C_{(0)}(\reals)$ can be uniformly approximated by an element of $\mcal{A}$ and $\mu(\reals)$ is finite, we have that $\int \psi d\mu = 0$ for any continuous function $\psi$. By the spanning criterion, the closure of $W$ is all of $C_{(0)}(\reals)$.
\end{proof}

\noindent\textbf{Problem 4. }
Let $V$ be a complex vector space and let $f_j$, $0\leq j\leq N$, be linear forms on $V$ such that
\[
\bigcap_{j=1}^N\ker f_j \subseteq \ker f_0.
\]
Show that $f_0$ is a linear combination of the $f_j$'s, $1\leq j\leq N$.
\begin{proof}
	(This is lemma 3.9 in Rudin's \textit{Functional Analysis}.) In order to apply any result related to Hahn-Banach, we need to be working with a normed vector space, which $V$ needn't be.
	Our plan is to map into $\complex^n$, which clearly is a normed space. We'll apply Hahn-Banach there and use that to help us back in $V$. Define $f: V\to \complex^n$ by $f(x) = (f_1(x), \ldots, f_N(x))$. Now define the linear functional $T:f(V)\to \complex$ by $T(f(x)) = f_0(x)$.\\

	First we need to show that $T$ is well-defined. Suppose $f(x) = f(y)$. Then $f_j(x) = f_j(y)$ for $j = 1, \ldots, N$. In this case, $x-y$ is in the kernel of each $f_j$, so by hypothesis, it's in the kernel of $f_0$ too, so $T(f(x)) = T(f(y))$. Any linear functional on the finite dimensional space $\complex^N$ is continuous, so $T$ is a linear continuous functional on $f(V)$. By Hahn-Banach, we can extend $T$ to a linear functional, $\tilde{T}$, on all of $\complex^N$.\\

	Now any continuous linear functional on $\complex^N$ has the form
	\[
	\tilde{T}(z_1, \ldots, z_N) = \alpha_1z_1 + \cdots + \alpha_Nz_N
	\]
	for some complex numbers $\alpha_1, \ldots, \alpha_N$. This representation gives us exactly what we need. For any $x\in V$ we have
	\begin{align*}
		f_0(x) &= \tilde{T}(f(x))\\
		&= \tilde{T}(f_1(x), \ldots, f_N(x))\\
		&= \alpha_1f_1(x) + \cdots \alpha_Nf_N(x),
	\end{align*}
	so $f_0$ is a linear combination of the $f_j$'s.
\end{proof}

\noindent\textbf{Problem 5. }
Let $X$ be a Banach space such that $X^*$ is separable. Prove that $X$ is separable.
\begin{proof}
	Let $T_n$ be a countable and dense subset of $X^*$. For each $n$ we can find an $x_n$ in $X$ so that $\frac{1}{2}\|T_n\| \leq |T_n x_n| \leq \|T_n\|$ and $\|x_n\| = 1$. We claim that the rational span of the $x_n$'s, $Y$, is a countable dense subset of $X$.\\

	\noindent Suppose not. Then we can find an open neighborhood in $X$ disjoint from $\overline{Y}$. By the geometric form of Hahn-Banach, we can find a closed affine hyperplane separating $\overline{Y}$ and this neighborhood (since linear subspaces and their complements are convex). That is, we can find $T\in X^*$ that vanishes on $\overline{Y}$ but is not identically zero. Now by the density of the $T_n$'s, we can find a sequence $T_{n_j}$ that limits to $T$ in $X^*$. Now let's look at the norms of the $T_{n_j}$'s
	\begin{align*}
		\frac{1}{2}\|T_{n_j}\| &\leq |T_{n_j}x_{n_j}|\\
		&\leq |T_{n_j}x_{n_j} - Tx_{n_j}| + |Tx_{n_j}|\\
		&= |T_{n_j}x_{n_j}-Tx_{n_j}|\\
		&\leq \|T_{n_j}-T\|,
	\end{align*}
	which goes to zero by construction. But then $T$ would be the zero functional - a contradiction. We conclude that $\overline{Y} = X$ and $X$ is separable.
\end{proof}

\noindent\textbf{Problem 6. }
Show that the closure in $L^2(\reals)$ of the set of functions of the form
\[
p(x)e^{-x^2},\quad x\in \reals,
\]
where $p$ is a complex polynomial on $\reals$, is equal to all of $L^2(\reals)$.
\begin{proof}
	Let $W$ be the set of all polynomials of the form $p(x)e^{-x^2}$. $W$ is a linear subspace of $L^2(\reals)$, and since $L^2(\reals)$ is a Hilbert space, we have that $L^2(\reals) = \overline{W}\oplus \overline{W}^\perp$. It then suffices to show that $W^\perp= \{0\}$.\\

	\noindent Rather than dealing with general polynomials, we can consider $W$ to be the span of functions of the form $f_n(x) = x^ne^{-x^2}$. Our plan is to show that if $f\in L^2(\reals)$ is orthogonal to each $f_n$, then its Fourier transform vanishes identically. Since the map that sends $\varphi$ to its Fourier transform is an isometric isomorphism on $L^2(\reals)$, this will show that $f$ itself is identically zero.
	\begin{align*}
		(\widehat{\overline{f}(x)e^{-x^2}})(t)&= \int_\reals \overline{f}(x)e^{-x^2}e^{-itx}\ dx\\
		&= \int_\reals \overline{f}(x)e^{-x^2}\sum_{n=0}^\infty \frac{(-itx)^n}{n!}\ dx\\
		&= \sum_{n=0}^\infty \frac{(-it)^n}{n!}\int_\reals\overline{f}(x)x^ne^{-x^2}\ dx\\
		&= \sum_{n=0}^\infty \frac{(-it)^n}{n!}\langle f_n, f\rangle\\
		&= 0.
	\end{align*}
	We used Fubini's theorem to switch the order of integration and summation since
	\[
	\left|\overline{f}(x)e^{-x^2}\frac{(-itx)^n}{n!}\right| = |f(x)|e^{-x^2}\frac{|t|^n|x|^n}{n!}
	\]
	is integrable in $x$ (by H\"older's inequality) and summable in $n$. Now this shows that the Fourier transform of $\overline{f}$ vanishes. But $\widehat{\overline{f}}(t) = \widehat{f}(-t)$, so one vanishes if and only if the other does. 
\end{proof}

\noindent\textbf{Problem 7. }
Let $f\in L^1_{loc}(\reals)$ be $2\pi$-periodic. Show that linear combinations of the translates $f(x-a)$, $a\in \reals$ are dense in $L^1(0, 2\pi)$ if and only if each Fourier coefficient of $f$ is nonzero.
\begin{proof}
	Suppose that the translates of $f$ are dense in $L^1(0, 2\pi)$, but $\widehat{f}(n) = 0$ for some $n$. Let's look at the Fourier coefficients of the translates.
	\begin{align*}
	\widehat{f(x-a)}(n) &= \frac{1}{2\pi}\int_0^{2\pi}f(x-a)e^{-inx}\ dx\\
	&= e^{-ina}\frac{1}{2\pi}\int_0^{2\pi}f(t)e^{-int}\ dt\\
	&= e^{-ina}\widehat{f}(n).
	\end{align*}
	In particular, if $\widehat{f}(n) = 0$, then the $n$-th Fourier coefficient of each translate also vanishes.\\

	\noindent Now $e^{-inx}$ is bounded for all real $t$, so the map $g\mapsto \frac{1}{2\pi}\int_0^{2\pi}g(x)e^{-inx}\ dx$ is a continuous (by H\"older's inequality) linear functional on $L^1(0, 2\pi)$. By the above discussion, we have that this functional vanishes on all translates of $f$. Since these translates are dense in $L^1(0, 2\pi)$ by hypothesis, we must have that this functional is the zero functional, which it clearly isn't. We conclude that none of $f$'s Fourier coefficients vanish.\\

	\noindent Suppose now that $\widehat{f}(n)\neq 0$ for all $n$ and that some linear continuous functional $T$ on $L^1(0,2\pi)$ vanishes on the translates of $f$. More concretely, since the dual space of $L^1(0,2\pi)$ is $L^\infty(0,2\pi)$, there is some $h\in L^\infty(0,2\pi)$ such that
	\[
	T(f(x-a)) = \int_0^{2\pi}f(x-a)h(x)\ dx = 0\quad\text{for all }a\in \reals.
	\]
	This says that the convolution $f*h$ vanishes identically. Since $\widehat{f*h} = \widehat{f}\cdot\widehat{h}$, we have that $\widehat{f}(t)\widehat{h}(t) = 0$ for all $t$. By hypothesis, $\widehat{f}(n)\neq 0$ for all $n$, so $\widehat{h}(n) = 0$ for all $n$. (Note that $\widehat{h}$ exists since the Fourier transform is defined on $L^1(0,2\pi)$ and $L^\infty(0,2\pi)\subseteq L^1(0,2\pi)$, since $(0,2\pi)$ has finite measure.)\\

	\noindent Now by Stone-Weierstrass, the linear space spanned by functions of the form $e^{-ikx}$ is dense in the space of continuous functions on the torus. Since the continuous functions here are dense in $L^1(0,2\pi)$, we have that the functional $T$ vanishes on a dense subset of $L^1(0,2\pi)$, and therefore on all of $L^1(0,2\pi)$. By the spanning criterion, we have that the set of translates of $f$ is then dense in $L^1(0,2\pi)$.
\end{proof}

\noindent\textbf{Problem 8. }
Let $E_1$ be a finite-dimensional subspace of the normed space $E$. Show that there exists a continuous projection $P:E\to E_1$.
\begin{proof}
	Suppose that $E_1$ is spanned by the vectors $e_1, \ldots, e_n\in E$ for some $1\leq n<\infty$. Define the functionals $P_i: E_1\to \complex$ by $\sum_{j=1}^n \alpha_j e_j\mapsto \alpha_i$. We have that for any $x\in E_1$, $|P_i(x)|\leq \|x\|$, so the $P_i$'s are continuous. By Hahn-Banach, we can extend each $P_i$ to $\tilde{P}_i: E\to \complex$. Now define $P: E\to E_1$ by
	\[
	P(x) = \tilde{P}_1(x)e_1 + \cdots \tilde{P}_n(x)e_n.
	\]
	Since the $e_j$'s are in $E_1$, we have that the range of $P$ is contained in $E_1$. Furthermore, $P$ restricts to the identity map on $E_1$, so the range of $P$ is all of $E_1$ and $P^2 = P$.
\end{proof}

\end{document}