%% Please change the file name by replacing N with the apporpriate number
%% corresponding to the current homework and XX with your initials.
%% https://www.math.uci.edu/~gpatrick/jsOnline/hw1.html

\documentclass[11pt,letterpaper]{report}
\usepackage{amssymb,amsfonts,color,graphicx,amsmath,enumerate}
\usepackage{tikz} %This package offers the ability to draw pictures
\usepackage{amsthm}

\newcommand{\naturals}{\mathbb{N}}
\newcommand{\integers}{\mathbb{Z}}
\newcommand{\complex}{\mathbb{C}}
\newcommand{\reals}{\mathbb{R}}
\newcommand{\exreals}{\overline{\mathbb{R}}}
\newcommand{\mcal}[1]{\mathcal{#1}}
\newcommand{\mable}{measurable}
\newcommand{\quats}{\mathbb{H}}
\newcommand{\rationals}{\mathbb{Q}}
\newcommand{\norm}{\trianglelefteq}
\newcommand{\Aut}{\text{Aut}}
\newcommand{\disk}{\mathbb{D}}
\newcommand{\halfplane}{\mathbb{H}}
\newcommand{\Lp}[2]{\left\|{#1}\right\|_{L^{#2}}}
\newcommand{\supp}[1]{\text{supp}({#1})}
\newcommand{\Hom}[2]{\text{Hom}_{{#1}}({#2})}
\newcommand{\tr}{\text{tr}}
\newcommand{\field}[1]{\mathbb{F}_{{#1}}}
\newcommand{\Gal}[1]{\text{Gal}\left({#1}\right)}
\newcommand{\esssup}{\text{ess sup }}
\newcommand{\essinf}{\text{ess inf }}
\newcommand{\affine}{\mathbb{A}}
\newcommand{\ball}{\mathbb{B}}

\newenvironment{solution}
{\begin{proof}[Solution]}
{\end{proof}}

\voffset=-3cm
\hoffset=-2.25cm
\textheight=24cm
\textwidth=17.25cm
\addtolength{\jot}{8pt}
\linespread{1.3}

\begin{document}
\noindent{\em Liam Hardiman\hfill{October 15, 2018} }
% Please give relevant information
\begin{center}
{\bf \Large 218A - Homework 1} %Replace N with the appropriate number
\vspace{0.2cm}
\hrule
\end{center}

%1, 6, 7, 8, 9, 11
\noindent\textbf{1-1}
Let $X$ be the set of all points $(x,y)\in \reals^2$ such that $y = \pm 1$, and let $M$ be the quotient of $X$ by the equivalence relation generated by $(x, -1)\sim (x, 1)$ for all $x\neq 0$. Show that $M$ is locally Euclidean and second-countable, but not Hausdorff.
\begin{proof}
	There is a bijective correspondence between the nonzero real numbers and the elements of $M$ that are not the images of $(0, 1)$ and $(0, -1)$ under the quotient map. Let $0_+$ be the image of $(0, 1)$ under the quotient map and let $0_-$ be the image of $(0, -1)$. Denote by $[r]$ the equivalence class $\{(1, r), (1, -r)\}$, $r\neq 0$.\\\\
	% Consider the point $[r]\in M$ where $r>0$. Let $U_r = \{[x]\in M: r/2<x<3r/2\}$. This set is an open neighborhood of $[r]$ since it contains $[r]$ and
	% \[
	% \bigcup_{[x]\in U_r}\{x\in [x]\} = \{(x, 1): r/2<x<3r/2\}\cup \{(x, -1): r/2<x<3r/2\}\subseteq X
	% \]
	% is an open set in $X$ (viewed as a subspace of $\reals^2$). Let $V_r\subseteq \reals$ be the open interval $(r/2, 3r/2)$. Now let $f_r: U_r\to V_r$ be the map that sends $[x]\in U_r$ to $x$. It's clear that $f_r$ is a bijection of sets and continuity (of both $f_r$ and its inverse) follows from the fact that the preimage of an open interval in $V_r$ is a set of the form $\{[y]: s<y<t\}$, which is open in $U_r$. We can analogously define local homeomorphisms for points $[r]\in M$ where $r<0$.\\\\
	% Now for our two ``origins''. $U_+ = \{[x]\in M: x\in (-1, 0)\cup (0, 1)\}\cup \{0_+\}$ is an open neighborhood of $0_+$ in $M$ since the union of the elements in the equivalence classes in $U_+$ is the interval $(-1, 1)$.
	To show that $M$ is locally Euclidean it suffices to show that each of the subspaces $M\setminus \{0_+\}$ and $M\setminus \{0_-\}$ is homeomorphic to $\reals$. Let $f_+: M\setminus \{0_-\}$ be defined by
	\[
	f_+([x]) = \begin{cases}
		x&\text{, if }x\neq 0\\
		0&\text{, if }[x] = 0_+
	\end{cases}
	\]
	$f_+$ is clearly a bijection of sets and continuity follows from the fact that the inverse image of an interval $(a,b)$ is given by
	\[
		(f_+)^-[(a,b)] = \begin{cases}
			\{[x]: x\in (a,b)\}&\text{, if }0\notin (a,b)\\
			\{[x]: x\in (a, 0)\}\cup 0_+\cup \{[x]: x\in (0,b)\}&\text{, otherwise}
		\end{cases}
	\]
	which is open in $M$ in either case. Continuity of the inverse follows from the same reasoning, where we consider a basis for the topology on $M$ given by sets of the form $\{[x]: x\in (a,b),\ 0\notin (a,b)\}$ and $\{[x]: x\in (a, 0)\cup (0,a)\}\cup 0_+$.\\\\
	The basis for the topology on $M$ considered above can be made countable by taking $a$ and $b$ to be rational, so $M$ is second-countable. Finally $M$ is not Hausdorff because $0_+$ and $0_-$ are not separable by open sets. Specifically, any open neighborhood of $0_+$ would contain a set of the form $\{[x]: x\in (-a, 0)\cup (a, 0)\}$ for some $a$, which intersects the neighborhood of $0_-$ given by $\{[x]: x\in (-a, 0)\cup (a, 0)\}\cup \{0_-\}$.
\end{proof}

\noindent\textbf{1-6}
Let $M$ be a nonempty topological manifold of dimension $n\geq 1$. If $M$ has a smooth structure, show that it has uncountably many distinct ones.
\begin{proof}
	Let $\mcal{A}$ be a smooth structure on $M$. Since $M$ is a manifold, for any $x\in M$ we can find an open neighborhood $U$ of $x$ and a homeomorphism $\phi: U\to \ball^n$ and $\phi(x) = 0$.
\end{proof}

\end{document}