%% Please change the file name by replacing N with the apporpriate number
%% corresponding to the current homework and XX with your initials.
%% https://www.math.uci.edu/~gpatrick/jsOnline/hw1.html

\documentclass[11pt,letterpaper]{report}
\usepackage{amssymb,amsfonts,color,graphicx,amsmath,enumerate}
\usepackage{tikz} %This package offers the ability to draw pictures
\usepackage{amsthm}
\usepackage{bookmark}

\newcommand{\naturals}{\mathbb{N}}
\newcommand{\integers}{\mathbb{Z}}
\newcommand{\complex}{\mathbb{C}}
\newcommand{\reals}{\mathbb{R}}
\newcommand{\exreals}{\overline{\mathbb{R}}}
\newcommand{\mcal}[1]{\mathcal{#1}}
\newcommand{\mable}{measurable}
\newcommand{\quats}{\mathbb{H}}
\newcommand{\rationals}{\mathbb{Q}}
\newcommand{\norm}{\trianglelefteq}
\newcommand{\Aut}{\text{Aut}}
\newcommand{\disk}{\mathbb{D}}
\newcommand{\halfplane}{\mathbb{H}}
\newcommand{\Lp}[2]{\left\|{#1}\right\|_{L^{#2}}}
\newcommand{\supp}[1]{\text{supp}({#1})}
\newcommand{\Hom}[2]{\text{Hom}_{{#1}}({#2})}
\newcommand{\tr}{\text{tr}}
\newcommand{\field}[1]{\mathbb{F}_{{#1}}}
\newcommand{\Gal}[1]{\text{Gal}\left({#1}\right)}
\newcommand{\esssup}{\text{ess sup }}
\newcommand{\essinf}{\text{ess inf }}
\newcommand{\affine}{\mathbb{A}}

\newenvironment{solution}
{\begin{proof}[Solution]}
{\end{proof}}

\voffset=-3cm
\hoffset=-2.25cm
\textheight=24cm
\textwidth=17.25cm
\addtolength{\jot}{8pt}
\linespread{1.3}

\begin{document}
% Please give relevant information
\begin{center}
{\bf \Large Real Analysis Qualifying Exams} %Replace N with the appropriate number
\vspace{0.2cm}
\hrule
\end{center}

\section{Spring 2016}
\begin{enumerate}
	\item Assume $f\in L^1[0,1]$. Compute
	\[
	\lim_{k\to \infty}\int_{[0,1]}|f|^{1/k}\ dx.
	\]
	\begin{solution}
		Let's split this integral into three regions.
		\[
		\int_{[0,1]}|f|^{1/k}\ dx = \int_{f=0}|f|^{1/k}\ dx + \int_{0<|f|\leq 1}|f|^{1/k}\ dx + \int_{|f|>1}|f|^{1/k}\ dx.
		\]
		The integral over the first region is clearly zero for all $k$. On the second region we have that $|f|^{1/k} \leq 1$ for all $k$. Since the interval $[0,1]$ has finite measure, we have that the constant function 1 is in $L^1(\{x: 0<|f|\leq 1\})$, so the dominated convergence theorem says that the integral over the second region goes to $m(\{0<|f|\leq 1\})$. Similarly, on the third region we have that $|f|^{1/k}\leq |f|$, which is in $L^1$, so the dominated convergence theorem says that the third integral goes to $m(\{|f|>1\})$. Combining these, we have that
		\[
		\lim_{k\to \infty}\int_{[0,1]}|f|^{1/k}\ dx = m(\{|f|>0\}).
		\]
	\end{solution}

	\item Let $\{f_n\}$ be a sequence of measurable functions on $[0,1]$ and $0\leq f_n\leq 1$ a.e. Assume that
	\[
	\lim_{n\to \infty}\int_{[0,1]}f_ng\ dx = \int_{[0,1]}fg\ dx
	\]
	for some $f\in L^1[0,1]$ and any $g\in C[0,1]$. Prove that $0\leq f\leq 1$ a.e.
	\begin{solution}
		Since $f\in L^1[0,1]$, by the Lebesgue differentiation theorem we have that
		\begin{equation}\label{s16_2_ldt}
		\frac{1}{m(E)}\int_Ef(t)\ dt \to f(x)
		\end{equation}
		as $E$ shrinks to $x$ for almost all $x$. Furthermore, since $0\leq f_n\leq 1$ we also have that
		\[
		\frac{1}{m(E)}\int_Ef_n(t)\ dt\to f_n(x)\in [0,1]
		\]
		as $E$ shrink to $x$ for almost all $x$.  Intuitively, we'd like to replace the integral of $f$ in (\ref{s16_2_ldt}) with a limit of integrals of $f_n$.\\

		\noindent We claim that the function $g$ in the given hypothesis can be replaced with the indicator function of an interval $\chi_I$. To see this, let $g_m$ be a sequence of continuous functions with $g_m\to \chi_I$ in $L^1$ and $0\leq \chi_I\leq 1$. By extracting a subsequence we can assume that $g_m\to \chi_I$ a.e. as well. We then have
		\[
		\int_0^1|f_n\chi_I - f\chi_i| \leq \int_0^1|f_n\chi_I-f_ng_m| + \int_0^1|f_ng_m-fg_m| + \int_0^1|fg_m-f\chi_i|.
		\]
		Since $\|f_n\|_{L^\infty}\leq 1$, we have that the first integral on the RHS can be made small uniformly in $n$ by picking $m$ large. The second integral goes to zero as $n\to \infty$ by hypothesis since $g_m$ is continuous. The third integral can be made small for $m$ large by dominated convergence since $|fg_m| \leq |f|\in L^1$.\\

		For almost all $x$, if $I_k$ is a sequence of intervals shrinking to $x$ then
		\begin{align*}
		\frac{1}{m(I_k)}\int_{I_k}f\ dx &= \frac{1}{m(I_k)}\int f\chi_{I_k}\ dx\\
		&= \lim_{n\to \infty}\frac{1}{m(I_k)}\int f_n\chi_{I_k}\ dx.
		\end{align*}
		Since $0\leq f_n\leq 1$, the RHS is in $[0, 1]$ for almost all $x$. By the Lebesgue differentiation theorem we then have that $0\leq f\leq 1$ a.e.
	\end{solution}

	\item Let $f,g\in L^2(\reals, \mcal{M}_L, \mu_L)$. Show that $f*g$ is a continuous function on $\reals$ vanishing at infinity, that is, $f*g\in C(R)$ and $\lim_{|x|\to \infty}(f*g)(x) = 0$.
	\begin{proof}
		For any $h$ we have by H\"older's inequality
		\begin{align}\label{s16_3_main}
			|(f*g)(x+h)-(f*g)(x)| &= \left|\int f(t)[g(x+h-t) - g(x-t)]\ dt\right|\\
			&\leq \|f\|_{L^2}\cdot \|g_h-g\|_{L^2},
		\end{align}
		where $F_h(x) = F(x+h)$ for any function $F$. Now for any $\epsilon>0$ we can find $\varphi\in C_0(\reals)$ with $\|g-\varphi\|_{L^2} = \|g_h-\varphi_h\|_{L^2} <\epsilon$. By the triangle inequality we then have
		\begin{align*}
		\Lp{g_h-g}{2} &\leq \Lp{g_h-\varphi_h}{2} + \Lp{\varphi_h-\varphi}{2} +\Lp{\varphi-g}{2}\\
		&< \Lp{\varphi_h-\varphi}{2} + 2\epsilon.
		\end{align*}
		Suppose that $\varphi$ has support contained in the compact set $K$. If we pick $h$ small enough then we can guarantee that $\varphi_h-\varphi$ is supported on a set with measure at most $2\cdot m(K)$. Now since $\varphi$ is continuous with compact support, it is uniformly continuous, so we can choose $h$ small enough that $|\varphi_h(x)-\varphi(x)| = |\varphi(x+h)-\varphi(x)|<\epsilon$ for all $x$ in the support of $\varphi_h-\varphi$. For such $h$ we have
		\[
		\Lp{\varphi_h-\varphi}{2} \leq \epsilon \cdot (2\cdot m(K))^{1/2},
		\]
		so (\ref{s16_3_main}) can be made arbitrarily small, which shows that $f*g$ is continuous.\\

		\noindent First we claim that if $\varphi$ and $\psi$ are continuous with compact support then $\varphi*\psi$ vanishes at infinity. By definition we have that
		\[
		(\varphi*\psi)(x) = \int\varphi(t)\psi(x-t)\ dt.
		\]
		The product $\varphi(t)\psi(x-t)$ is nonzero only if $t$ is in the support of $\varphi$ and $x-t$ is in the support of $\varphi$. If pick $x$ large enough then supports of $t\mapsto \varphi(t)$ and $t\mapsto \psi(x-t)$ are disjoint, so this integral is zero.\\

		\noindent Let $f_n$ and $g_n$ be sequences in $C_0(\reals)$ converging in $L^2$ to $f$ and $g$, respectively. We then have
		\begin{align*}
			|(f*g)(x)-(f_n*g_n)(x)| &\leq |(f*g)(x)-(f_n*g)(x)| + |(f_n*g)(x)-(f_n*g_n)(x)|\\
			&\leq \Lp{g}{2}\cdot \Lp{f-f_n}{2} + \Lp{f_n}{2}\cdot \Lp{g-g_n}{2}.
		\end{align*}
		Since $f_n\to f$ and $g_n\to g$ in $L^2$, we have that $f_n*g_n$ converges uniformly to $f*g$. Since $f_n*g_n$ vanishes at infinity, we must then have that $f*g$ vanishes at infinity.
	\end{proof}

	\item Let $(X, \mcal{A}, \mu)$ be a finite measure space, and let $p_1\in (1, \infty]$. Let $\{f_n\}$ be a uniformly bounded sequence in $L^{p_1}(X, \mcal{A}, \mu)$. Suppose $f = \lim_{n\to \infty}f_n$ exists $\mu$-a.e. Prove that $f\in L^p(X, \mcal{A}, \mu)$ for all $p\in [1, p_1]$ and $f_n\to f$ in $L^p(X, \mcal{A}, \mu)$ for all $p\in [1, p_1)$.
	\begin{proof}
		Suppose that $\Lp{f_n}{p_1}\leq M$ for all $n$. First we claim that the $f_n$ are in $L^p(X, \mcal{A}, \mu)$ for all $p\in [1, p_1]$. In fact, they are uniformly bounded:
		\begin{equation}\label{s16_4_main}
		\begin{split}
			\int_X|f_n|^p &= \int_{|f_n|<1}|f_n|^p + \int_{|f_n|\geq 1}|f_n|^p\\
			&\leq \int_{|f_n|<1}1 + \int_{|f_n|\geq 1}|f_n|^{p_1}\\
			&\leq \mu(\{f\leq 1\}) + M^{p_1}.
		\end{split}
		\end{equation}
		Since $(X, \mcal{A}, \mu)$ is a finite measure space, this quantity is finite, so $f_n\in L^p(X, \mcal{A}, \mu)$ for all $n$ and $p\in [1, p_1]$. We can then use the fact that $f_n\to f$ a.e. and Fatou's lemma to show that $f\in L^p(X, \mcal{A}, \mu)$ for $p\in [1, p_1]$:
		\[
		\int_X|f|^p \leq \liminf_{n\to \infty}\int_X|f_n|^p<\infty,
		\]
		where the finiteness follows from the $L^p$ uniform-boundedness of the $f_n$.\\

		\noindent To establish convergence in $L^p$, $p\in [1, p_1)$ our plan is to use the Vitali convergence theorem. The family $f_n$ is tight over $X$ since $X$ is a finite measure space and we're given that $f_n\to f$ a.e., so it only remains to show that the $f_n$'s are uniformly integrable.
		To this end, let $E$ be any measurable subset of $X$. Since $f_n$ is in $L^{p_1}$, we have that $|f_n|^p\in L^{p_1/p}$. If we let $q$ be the H\"older conjugate to $p_1/p$ then we have
		\begin{align*}
			\int_E|f_n|^p &= \int_X |f_n|^p \cdot \chi_E\\
			&\leq \Lp{|f_n|^p}{p_1/p}\cdot \Lp{\chi_E}{q}\\
			&\leq M^{p_1^2/p}\cdot \mu(E)^{1/q}.
		\end{align*}
		% Intuitively, since the $f_n$'s are in $L^p$, the measure of the set $\{f_n \geq N\}$ should shrink as $N$ grows.  Now since $p<p_1$, if $N>1$ then
		% \[
		% |f_n|^p\chi_{\{|f_n|\geq N\}}N^{p_1-p}\leq |f_n|^{p_1}.
		% \]
		% If we integrate both sides over any measurable set $E$ we have
		% \[
		% \int_{E\cap \{|f_n|\geq N\}}|f_n|^p \leq \frac{M}{N^{p_1-p}}.
		% \]
		% On the complement we have
		% \[
		% \int_{E\cap \{|f_n|<N\}}|f_n|^p \leq N^p\cdot \mu(E).
		% \]

		% Putting these together, we have that
		% \begin{align*}
		% 	\int_E|f_n|^p &= \int_{E\cap \{|f_n|\geq N\}}|f_n|^p + \int_{E\cap \{|f_n|<N\}}|f_n|^p\\
		% 	&\leq \frac{M}{R^{p_1-p}} + R^p\cdot\mu(E).
		% \end{align*}
		% If we choose $R$ so that $M/R^{p_1-p}<\epsilon/2$ and $E$ so that $R^p\cdot \mu(E)<\epsilon/2$ then we'll have that $\int_E |f_n|^p<\epsilon$ for any $E$ of sufficiently small measure, so the $f_n$'s are uniformly integrable. By the Vitali convergence theorem we have that $f_n\to f$ in $L^p$ for $p\in [1, p_1)$.
		If we choose $E$ so that $\mu(E)^{1/q} < \epsilon\cdot M^{-p_1^2/p}$, then we'll have that $\int_E|f_n|^p <\epsilon$, so the $f_n$'s are uniformly integrable. By the Vitali convergence theorem we have that $f_n\to f$ in $L^p$ for $p\in [1, p_1]$.
	\end{proof}

	\item Let $(X, \mcal{A}, \mu)$ be a measure space, and let $f: X\to [0, \infty)$ be $\mcal{A}$-measurable. Consider the measure space $(\reals, \mcal{B}_\reals, \mu_L)$, where $\mcal{B}_\reals$ is the Borel $\sigma$-algebra on $\reals$ and $\mu_L$ is the Lebesgue measure, and form the product measure space $(X\times \reals, \sigma(\mcal{A}\times \mcal{B}_\reals), \mu\times \mu_L)$. Define $E\subset X\times R$ by $(x,y)\in E\iff y\in [0, f(x))$. Prove that $E\in \sigma(\mcal{A}\times \mcal{B}_\reals)$ and $(\mu\times \mu_L)(E) = \int_Xf\ d\mu$.
	\begin{proof}
		A function is measurable if it pulls measurable sets back to measurable sets. The plan is then to write $E$ is a union and/or intersection of preimages of measurable sets under measurable functions. The function $F(x,y) = f(x)$ is measurable since
		\[
		F^{-1}[(-\infty, \alpha]) = \{(x,y): f(x)\leq \alpha\} = \{x: f(x)\leq \alpha\} \times \reals \in \sigma(\mcal{A}\times \mcal{B}_\reals),
		\]
		as $f$ is $\mu$-measurable. We also clearly have that the function $G(x,y) = y$ is measurable. Now consider the function $H(x,y) = y-f(x)$. $H$ is measurable as it is the difference of the measurable functions $G$ and $F$. We then have that $E$ is measurable through the following decomposition
		\begin{align*}
			E &= \{(x,y): 0\leq y<f(x)\}\\
			&= \{(x,y): y\geq 0\} \cap \{(x,y): y<f(x)\}\\
			&= G^{-1}[[0, \infty)]\cap H^{-1}[(-\infty, 0)].
		\end{align*}
		If $\{f>0\}$ is $\sigma$-finite we can use Tonelli's theorem to say
		\begin{align*}
			(\mu\times \mu_L)(E) &= \int_{X\times \reals}\chi_E(x,y)\ d(\mu\times \mu_L)\\
			&= \int_X\int_\reals\chi_E(x,y)\ d\mu_Ld\mu\\
			&= \int_X\int_\reals \chi_{[0, f(x))}(y)\ dyd\mu\\
			&= \int_Xf(x)\ d\mu.
		\end{align*}
		On the other hand, suppose that $\{f>0\}$ is note $\sigma$-finite. We claim that $\int_Xf\ d\mu = +\infty$. Indeed, since we can decompose this set into a countable union,
		\begin{equation}\label{s16_5_main}
		\{f>0\} = \bigcup_{m=1}^\infty \{\frac{1}{m+1}<f\leq \frac{1}{m}\} \cup \bigcup_{n=1}^\infty\{n<f\leq n+1\},
		\end{equation}
		we must have that one of these sets has infinite measure. We need to show that $(\mu\times \mu_L)(E) = +\infty$ too. For any $\alpha,\beta>0$ we have that if $\alpha\leq f(x)<\beta$ then the product set
		\[
		\{x: \alpha\leq f(x)<\beta\} \times \{y: 0\leq \alpha\}
		\]
		is contained in $E$. This product set has measure $\alpha\cdot \mu_L\{\alpha\leq f<\beta\}$, so by monotonicity we have that
		\[
		\alpha\cdot \mu_L\{\alpha\leq f<\beta\} \leq (\mu\times \mu_L)(E)
		\]
		for all $\alpha,\beta>0$. But by the decomposition (\ref{s16_5_main}), we have that some set of the form $\{\alpha\leq f(x)<\beta\}$ must have infinite measure, so we must have $(\mu\times \mu_L)(E) = +\infty$.
	\end{proof}

	\item Let $f\in L^1(\reals)$ and let $a_1, \ldots, a_k\in \reals$ and $b_1, \ldots, b_k\in \reals\setminus\{0\}$. Assume that the quotients $\frac{a_j}{b_j}$ are all distinct. Determine
	\[
	\lim_{t\to \infty}\int\left|\sum_{j=1}^kf(b_jx+ta_j)\right| dx.
	\]
	\begin{solution}
		Let $\varphi\in C_0(\reals)$ be such that $\Lp{f-\varphi}{1}<\epsilon$. Our plan is to compute the desired limit with $\varphi$ in place of $f$ and then argue that the difference can be made small. We have
		\[
		\int\left|\sum_{j=1}^k\varphi(b_jx+ta_j)\right| dx = \int\left|\sum_{j=1}^k\varphi\left[b_j\left(x+\frac{a_j}{b_j}t\right)\right]\right|dx
		\]
		Now $\varphi(b_jx+ta_j)$ is $\varphi$ stretched horizontally by a factor of $b_j$ and shifted over $a_j/b_j$. Since the support of $\varphi$ is compact and the $a_j/b_j$ are distinct, the supports of these transformations are disjoint for sufficiently large $t$. When these supports are disjoint we then have
		\begin{align*}
		\int\left|\sum_{j=1}^k\varphi(b_jx+ta_j)\right| dx &= \int\sum_{j=1}^k|\varphi(b_jx+ta_j)|\ dx\\
		&= \Lp{\varphi}{1}\cdot \sum_{j=1}^k\frac{1}{b_j}.
		\end{align*}
		That we can approximate the desired sum for $f\in L^1$ follows from the reverse triangle inequality.
		\begin{align*}
			\left|\int\left|\sum_{j=1}^kf(b_jx+ta_j)\right|dx - \int\left|\sum_{j=1}^k\varphi(b_jx+ta_j)\right|dx\right| &\leq \sum_{j=1}^k\int|f(b_jx+ta_j)-\varphi(b_jx+ta_j)|\ dx\\
			&= \epsilon\cdot \sum_{j=1}^k\frac{1}{b_k}.
		\end{align*}
	\end{solution}
\end{enumerate}


\section{Fall 2015}
\begin{enumerate}
	\item Let $E$ be a measurable subset of $[0, 2\pi]$. Assume that $f\in C(\reals)$ is 1-periodic, i.e. $f(x+1) = f(x)$. Compute
	\[
	\lim_{n\to \infty}\int_Ef(nx)\ dx.
	\]
	\begin{solution}
		We rewrite the integral over $E$ as an integral over $\reals$ against the indicator function of $E$:
		\[
		\int_Ef(nx)\ dx = \int f(nx)\chi_E(x)\ dx.
		\]
		Now let $\varphi\in C^\infty_0(\reals)$. Since $f\in C(\reals)$ is 1-periodic, it has a 1-periodic continuous primitive $F$ with $F' = f$. By the chain rule we have $[\frac{1}{n}F(nx)]' = f(nx)$. Integration by parts gives
		\begin{align*}
		\int f(nx)\varphi(x)\ dx&= -\frac{1}{n}\int F(nx)\varphi'(x)\ dx.
		\end{align*}
		$F(nx)$ is bounded since $F$ is 1-periodic and $\varphi\in C^\infty_0(\reals)$, so it's integrable. We then have
		\begin{align*}
			\left|\int f(nx)\varphi(x)\ dx\right|& \leq \frac{1}{n}\|F\|_\infty\cdot \Lp{\varphi'}{1}\\
			&\to 0.
		\end{align*}
		Since $E$ is a measurable subset of $[0,2\pi]$, it has finite measure and $\chi_E\in L^1(\reals)$. We can then find $\varphi\in C_0^\infty(\reals)$ with $\Lp{\chi_E-\varphi}{1}<\epsilon$. Since $f$ is continuous and 1-periodic, it is bounded and we have
		\begin{align*}
			\left|\int f(nx)\chi_E(x)\ dx - \int f(nx)\varphi(x)\ dx\right| &\leq \|f\|_\infty\cdot \Lp{\chi_E-\varphi}{1}\\
			&\leq \|f\|_\infty\cdot \epsilon.
		\end{align*}
		Since $\int f(nx)\varphi(x)\ dx\to 0$, we must have $\int_Ef(nx)\to 0$.
	\end{solution}

	\item Suppose $f\in L^1[0,1]$ and assume that there exists $C>0$ such that for all measurable subsets $E\subset [0,1]$ we have
	\[
	\int_E|f(x)|\ dx \leq C\mu(E)^{1/2}.
	\]
	Show that $f\in L^p[0,1]$ for $1\leq p<2$. Show that the statement fails for $p=2$ by giving a counterexample.
	\begin{proof}
		% Since $[0,1]$ is a $\sigma$-finite measure space, we can use Tonelli's theorem to write
		% \begin{align*}
		% \int_{[0,1]}|f|^p\ dx &= \int_0^\infty \mu\{x: |f(x)|^p\geq t\}\ dt\\
		% &= \int_0^1\mu\{x: |f(x)|^p\geq t\}\ dt + \int_1^\infty \mu\{x: |f(x)|^p\geq t\}\ dt.
		% \end{align*}
		We have that
		\[
		|f(x)|^p-1 \leq \sum_{n=1}^\infty \chi_{\{|f|^p\geq n\}}(x) \leq |f(x)|^p.
		\]
		Since $[0,1]$ is a finite measure space, integrating through this inequality shows that $f\in L^p[0,1]$ if and only if the series
		\[
		\sum_{n=1}^\infty \mu\{|f(x)|^p\geq n\} = \sum_{n=1}^\infty \mu\{|f(x)|\geq n^{1/p}\}.
		\]
		converges. By Chebyshev's inequality and the given hypotheses we have
		\[
		n^{1/p}\mu\{|f|\geq n^{1/p}\} \leq \int_{\{|f|\geq n^{1/p}\}}|f|\ dx \leq C\mu\{|f|\geq n^{1/p}\}^{1/2}.
		\]
		Dividing through by $n^{1/p}\mu\{|f|\geq n^{1/p}\}^{1/2}$ and squaring gives
		\begin{align*}
		\sum_{n=1}^\infty \mu\{|f(x)|\geq n^{1/p}\} &\leq \sum_{n=1}^\infty \frac{C^2}{n^{2/p}},
		\end{align*}
		which converges for all $p\in [1, 2)$.\\

		\noindent 
	\end{proof}

	\item Show that a function $f: \reals^n\to \reals^+$ is measurable if and only if $E = \{(x,y): 0\leq y\leq f(x)\}$ is a measurable set of $\reals^{n+1}$.
	\begin{proof}
		Suppose $f$ is measurable. Then the function $F(x,y) = f(x)$ is a measurable function $\reals^{n+1}\to \reals$. Since $G(x,y)=y$ is also measurable, $H(x,y) = y-f(x)$ is measurable as the difference of measurable functions. We can then write $E$ as the intersection of two measurable sets:
		\[
		E = G^{-1}[[0,\infty)]\cap H^{-1}[(-\infty,0]].
		\]
		Thus, $E$ is measurable if $f$ is measurable.\\

		\noindent Conversely, suppose that $E$ is a measurable set. Then for any $\alpha\geq 0$ the set $A\cap G^{-1}(\alpha) = F^{-1}[[\alpha, \infty)]$. This shows that $F$, and therefore $f$, is measurable.
	\end{proof}

	\item Let $f\in L^1(\reals)$ and set
	\[
	f_h(x) = \frac{1}{2h}\int_{x-h}^{x+h}f(t)\ dt,\quad h>0.
	\]
	Show that $f_h\in L^1(\reals)$ and $f_h\to f$ in $L^1(\reals)$.
	\begin{proof}
		Let's integrate $f_h$. By Tonelli we have
		\begin{align}\label{f15_4_main}
		\begin{split}
			\int |f_h(x)|\ dx &= \frac{1}{2h}\int\left|\int f(t)\chi_{[x-h, x+h]}(t)\ dt\right|dx\\
			&\leq \frac{1}{2h}\int\int |f(t)|\chi_{[t-h, t+h]}(x)\ dxdt\\
			&= \Lp{f}{1}.
		\end{split}
		\end{align}
		Since $f\in L^1(\reals)$, we have that this quantity is finite and $f_h\in L^1(\reals)$.\\

		\noindent Now since $f\in L^1(\reals)$, $f_h\to f$ a.e. by the Lebesgue differentiation theorem. By Fatou's lemma and (\ref{f15_4_main}), we have for any sequence $h_n\to 0$
		\begin{align*}
			\int |f|\ dx &\leq \liminf_{n\to \infty}\int |f_{h_n}|\ dx\\
			&\leq \int |f|\ dx,
		\end{align*}
		so $\liminf_{n\to \infty} \int|f_{h_n}| = \int|f|$. By the triangle inequality we have $|f_{h_n}| + |f| - |f-f_{h_n}|\geq 0$. Since $|f_{h_n}|+|f| - |f-f_{h_n}|$ converges to $2|f|$ a.e., another application of Fatou's lemma gives
		\begin{align*}
		&2\int|f|\ dx \leq \liminf_{n\to \infty}\int(|f_{h_n}| + |f| - |f-f_{h_n}|)\ dx\\
		\iff&\limsup_{n\to \infty} \int |f-f_{h_n}|\ dx\leq 0.
		\end{align*}
		We then have $\int |f-f_{h_n}|\to 0$, so $f_{h_n}\to f$ in $L^1$ for any $h_n\to 0$.
	\end{proof}

	\item Let $(X, \mcal{A}, \mu)$ be a measure space and let $f_k: X\to \reals$ be a sequence of measurable functions satisfying the following:
	\[
	\int_X |f_k|^2\ d\mu\leq 2015,\quad \text{for all }k,
	\]
	and
	\[
	\int_X f_jf_k\ d\mu=0,\quad \text{for all }j\neq k.
	\]
	Prove that for all $\beta>3/2$,
	\[
	\lim_{n\to \infty}\frac{1}{n^\beta}\sum_{k=1}^{n^2}f_k(x) = 0,\quad \text{for a.a. } x\in X.
	\]
	\begin{proof}
		Let's compute the $L^2$ norm of the sum
		\begin{align*}
			\left\|\frac{1}{n^\beta}\sum_{j=1}^{n^2}f_j\right\|_{L^2}^2 &= \frac{1}{n^{2\beta}}\left(\sum_{j=1}^{n^2}f_j, \sum_{k=1}^{n^2}f_k\right)\\
			&= \frac{1}{n^{2\beta}}\sum_{j=1}^{n^2}\sum_{k=1}^{n^2}(f_j, f_k)\\
			&= \frac{1}{n^{2\beta}}\sum_{j=1}^{n^2}\Lp{f_j}{2}^2\\
			&\leq \frac{2015}{n^{2\beta-2}}.
		\end{align*}
		Now if $\beta>3/2$, $2\beta-2>1$, so the above quantity is summable in $n$. Summability and wanting to show that something holds for almost all $x$ leads us to think Borel-Cantelli might be useful. For any fixed $\epsilon>0$, Chebyshev gives us
		\begin{align*}
			\mu\left\{x: \left|\frac{1}{n^\beta}\sum_{j=1}^{n^2}f_j\right|^2\geq\epsilon\right\} &\leq \frac{1}{\epsilon^2}\int_X \left(\frac{1}{n^\beta}\sum_{j=1}^{n^2}f_j\right)^2dx\\
			&\leq \frac{2015}{\epsilon^2n^{2\beta-2}}.
		\end{align*}
		If we call the set on the LHS $A_n$, then we have $\sum \mu(A_n)<\infty$. By Borel-Cantelli we have $\mu(\limsup_{n\to \infty}A_n) = 0$, i.e., the set of $x$ that belong to infinitely many $A_n$ has measure zero, so the sum is zero for almost all $x$.
	\end{proof}

	\item Let $A, B\subseteq \reals^n$ be Lebesgue measurable sets and assume that for every $x\in \rationals^n$ there exists a null set $N_x$ such that
	\[
	A+x\subset B\cup N_x.
	\]
	Show that if $A$ is not a null set then the complement of $B$ in $\reals^n$ is a null set.
	\begin{proof}
		Suppose $A$ has positive measure. Since $\rationals$ is countable and the countable union of null sets is null, we have that $A+\rationals \subset B \cup N$ for some null set $N$. If $A+\rationals$ missed a set of positive measure, then the complement of $B$ would contain a set of positive measure. Let's show that this cannot happen.\\

		\noindent Suppose $E$ is a set of positive measure with $E\cap (A+\rationals) = \emptyset$. Define the function $f$ by the convolution
		\[
		f(x) = \int_{\reals^n}\chi_A(x-y)\chi_E(y)\ dy.
		\]
		If we choose $x=q\in \rationals^n$, then the integrand is nonzero if and only if $y\in E\cap (A+q) = \emptyset$, so $f(q) = 0$. But the convolution is continuous if we take $E$ to have finite measure and $\rationals^n$ is dense, so we must have $f \equiv 0$. But by Tonelli we have
		\begin{align*}
		\int_{\reals^n\times \reals^n}\chi_A(x-y)\chi_E(y)\ d(\mu_x\times \mu_y) &= \int \int \chi_A(x-y)\chi_E(y)\ dxdy\\
		&= m(A)m(E).
		\end{align*}
		Since $A$ is not null and $E$ is assumed to have positive measure, this must be positive, contradicting $f\equiv 0$. We conclude that $A+\rationals$ is null.
	\end{proof}
\end{enumerate}

\section{Spring 2015}
\begin{enumerate}
	\item Show that if $f\in L^4(\reals)$ then
	\[
	\lim_{c\to 1}\int_\reals|f(cx)-f(x)|^4\ dx = 0.
	\]
	\begin{proof}
		Suppose $\varphi$ is continuous with compact support. Then $\varphi(cx)$ converges to $\varphi(x)$ uniformly, and since the support of $\varphi$ is compact, we have that the desired limit holds with $\varphi$ in place of $f$.\\

		\noindent Now let $\varphi\in C_0(\reals)$ be such that $\Lp{f-\varphi}{4}<\epsilon$. Since $|a+b|^p \leq 2^p(|a|^p+|b|^p)$ for all $p>0$ we have
		\begin{align*}
			\int|f(cx)-f(x)|^4\ dx &= \int|f(cx)-\varphi(cx)+\varphi(cx)-\varphi(x)+\varphi(x)-f(x)|^4\ dx\\
			&\leq 2^4\int|f(cx)-\varphi(cx)|^4\ dx\\
			&+ 2^8\int|\varphi(cx)-\varphi(x)|^4\ dx + 2^8\int|\varphi(x)-f(x)|^4\ dx.
		\end{align*}
		The first and third integrals are small since $\Lp{f-\varphi}{4}<\epsilon$ and the second integral can be made small as $c\to 1$ since $\varphi(cx)\to \varphi(x)$ uniformly on a compact set.
	\end{proof}

	\item Let $f_n: (0, \infty)\to \reals$, be a sequence of Lebesgue measurable functions such that $f_n\to f$ a.e. as $n\to \infty$. Assume that there exists $g: (0, \infty)\to \reals$ such that $|f_n|\leq g$ for all $n$ and $g\in L^1(0, a)$ for all $0<a<\infty$. Assume furthermore that
	\[
	\int_1^\infty |f_n(\sqrt{x})|\ dx\leq C,
	\]
	for all $n$ and for some constant $C>0$. Show that $f_n\in L^1(0, \infty)$, $f\in L^1(0, \infty)$ and $f_n\to f$ in $L^1(0, \infty)$ as $n\to \infty$.
	\begin{proof}
		First let's show that $f_n\in L^1(0, \infty)$ for all $n$. Write
		\begin{equation}\label{s15_2_main}
		\int_0^\infty |f_n|\ dx = \int_0^1|f_n|\ dx + \int_1^\infty |f_n|\ dx.
		\end{equation}
		For the first integral, since $|f_n| \leq g$ and $g\in L^1(0, 1)$ we have
		\[
		\int_0^1|f_n|\ dx \leq \int_0^1 g\ dx <\infty.
		\]
		For the second integral in (\ref{s15_2_main}) we use the hypothesis about $f_n(\sqrt{x})$.
		\begin{align*}
			C &\geq \int_1^\infty |f_n(\sqrt{x})|\ dx\\
			&= 2\int_1^\infty t|f_n(t)|\ dt\\
			&\geq \int_1^\infty |f_n(t)|\ dt.
		\end{align*}
		Both integrals in (\ref{s15_2_main}) are then finite, so $f_n\in L^1(0, \infty)$. In fact, we actually have that the $f_n$ are uniformly bounded in $L^1(0, \infty)$ by $\int_0^1g\ dx + C$. Since $f_n\to f$ a.e. we can apply Fatou's lemma to show that $f\in L^1(0, \infty)$:
		\begin{align*}
			\int_0^\infty |f|\ dx &\leq \liminf_{n\to \infty}\int_0^\infty |f_n|\ dx\\
			&\leq \int_0^1 g\ dx + C\\
			&<\infty.
		\end{align*}
		Our plan is to use the Vitali convergence theorem to show that $f_n\to f$ in $L^1(0, \infty)$. We are given that $f_n\to f$ a.e., which implies that $f_n\to f$ in measure. Since $|f-f_n| \leq |f| + g$, we have that $f_n\to f$ in $L^1(0,a)$ for any $a$ by the dominated convergence theorem, so the $f_n$'s are uniformly integrable. To establish tightness, note that for any $t>1$ we have
		\begin{align*}
			\int_t^\infty |f_n(x)|\ dx &= \int_{t^2}^\infty\frac{|f_n(\sqrt{x})|}{2\sqrt{x}}\ dx\\
			&\leq \frac{C}{2t},
		\end{align*}
		which goes to zero as $t\to \infty$. By the Vitali convergence theorem we have that $f_n\to f$ in $L^1(0, \infty)$.
	\end{proof}

	\item Assume that $f\in C^1(0, 1)$ and
	\[
	\int_0^1x|f'|^p\ dx <+\infty
	\]
	for some $p>2$. Show that $\lim_{x\to 0^+}f(x)$ exists.
	\begin{proof}
		Let $x_n\to 0$ and say the integral in the problem statement has value $C<\infty$. If $q$ is such that $\frac{1}{p}+\frac{1}{q} = 1$, we have by H\"older's inequality
		\begin{align*}
			|f(x_n)-f(x_m)| &=\left| \int_{x_m}^{x_n}f'(x)\ dx\right|\\
			&\leq \int_{x_m}^{x_n}|f'(x)|\ dx\\
			&= \int_0^1x^{1/p}|f'(x)|x^{-1/p}\chi_{[x_m, x_n]}(x)\ dx\\
			&\leq \left(\int_0^1x|f'(x)|^p\ dx\right)^{1/p}\cdot \left(\int_{x_m}^{x_n}x^{-q/p}\ dx\right)^{1/q}.
		\end{align*}
		Since $p>2$, we have that $q<2$, so the last line above becomes
		\begin{align*}
			|f(x_n)-f(x_m)| \leq C\cdot \left.\frac{x^{1-q/p}}{1-q/p}\right|_{x_m}^{x_n}.
		\end{align*}
		Since $q<2$, we have that $1-\frac{q}{p}>0$, so as $x_m,x_n\to 0$, this expression goes to zero. Thus, the sequence $f(x_n)$ is Cauchy, so $\lim_{x\to 0}f(x)$ exists.
	\end{proof}

	\item Suppose that $E\subset [0,1]^2$ is measurable. Denote
	\[
	E_x = \{y\in [0,1]: (x,y)\in E\},\quad E_y = \{x\in [0,1]: (x,y)\in E\}.
	\]
	Show that if $m(E_x) = 0$ for almost all $x\in [0, \frac{1}{2}]$, then
	\[
	m(\{y\in [0,1]: m(E_y) = 1\})\leq \frac{1}{2}.
	\]
	\begin{proof}
		$E$ is contained in the unit square, which has finite measure. By Tonelli's theorem we then have
		\begin{align*}
			m(E) &= \int \chi_E(x,y)\ d(\mu_x\times \mu_y)\\
			&= \int_0^1\int_0^1\chi_E(x,y)\ dydx\\
			&= \int_0^1 m(E_y)\ dy =  \int_0^1m(E_x)\ dx\\
			&= \int_{1/2}^1m(E_x)\ dx\\
			&\leq \frac{1}{2}.
		\end{align*}
		This gives us
		\begin{align*}
		m(\{y\in [0,1]: m(E_y) = 1\}) &= \int_{\{y\in [0,1]: m(E_y) = 1\}}m(E_y)\ dy\\
		&\leq \int_0^1 m(E_y)\ dy\\
		&\leq \frac{1}{2}.
		\end{align*}
	\end{proof}

	\item Let $f\in L^p(\reals)$, $1<p<\infty$, and let $\alpha>1-\frac{1}{p}$. Show that the series
	\[
	\sum_{n=1}^\infty \int_n^{n+n^{-\alpha}}|f(x+y)|\ dy
	\]
	converges for a.e. $x\in \reals$.
	\begin{proof}
		Our strategy is to show that the sum, as a function of $x$, is locally integrable, and therefore finite almost everywhere. To this end, let $k$ be an arbitrary integer. Since the above integrands are nonnegative, the monotone convergence theorem will let us interchange the sum with integrals. By Tonelli we will interchange the integrals.
		\begin{align*}
			\int_k^{k+1}\sum_{n=1}^\infty \int_n^{n+n^{-\alpha}}|f(x+y)|\ dydx &= \sum_{n=1}^\infty \int_k^{k+1}\int_n^{n+n^{-\alpha}}|f(x+y)|\ dydx\\
			&= \sum_{n=1}^\infty \int_k^{k+1}\int|f(y)|\cdot \chi_{[n+x, n+n^{-\alpha}+x]}(y)\ dydx\\
			&= \sum_{n=1}^\infty \int\int_k^{k+1}|f(y)|\cdot \chi_{[y-n-n^{-\alpha}, y-n]}(x)\ dxdy.
		\end{align*}
		Let's compute the values of $y$ for which $[y-n-n^{-\alpha}, y-n]\cap [k, k+1]$ is nonzero. We need $k<y-n$, so $k+n<y$. We also need $y-n-n^{-\alpha}<k+1$, so $y<k+n+n^{-\alpha}+1$. This gives us
		\begin{align*}
			\int_k^{k+1}\sum_{n=1}^\infty \int_n^{n+n^{-\alpha}}|f(x+y)|\ dydx &= \sum_{n=1}^\infty \int_{k+n}^{k+n+n^{-\alpha}+1}\int|f(y)|\chi_{[y-n-n^{-\alpha}, y-n]}(x)\ dxdy\\
			&= \sum_{n=1}^\infty n^{-\alpha}\int_{k+n}^{k+n+n^{-\alpha}+1}|f(y)|\ dy.
		\end{align*}
		Our plan is to use H\"older's inequality with respect to the counting measure on the sequences $n^{-\alpha}$ and $\int_{k+n}^{k+n+n^{-\alpha}+1}|f(y)|\ dy$. Since $\alpha$ is given to be larger than the H\"older conjugate of $p$, we have that $n^{-\alpha}$ is in $\ell^q$. We also have
		\[
		\sum_{n=1}^\infty\left(\int_{k+n}^{k+n+n^{-\alpha}+1}|f(y)|\ dy\right)^p
		\]
	\end{proof}

	\item Suppose $E\subset \reals$ is measurable and $E = E+\frac{1}{n}$ for every natural number $n\geq 1$. Show that either $m(E) = 0$ or $m(E^c) = 0$.
	\begin{proof}
		By induction we can see that $E = E+\rationals$. Suppose $E$ isn't null and $E=E+\rationals$ misses a set $A$ of positive finite measure. Consider the consider the convolution
		\[
		f(x) = \int_\reals \chi_E(x-y)\chi_A(y)\ dy.
		\]
		Since $E+\rationals\cap A$ is empty, if $x\in \rationals$ then $f(x) = 0$. Furthermore, since $A$ has finite measure and $E$ is in $L^\infty(\reals)$, the convolution is continuous. Since $\rationals$ is dense and $f$, a continuous function vanishes on $\rationals$, we must have $f\equiv 0$. But by Tonelli we have that $\int f(x)\ dx = m(E)m(A)$, which is positie. We conclude that $E$ cannot miss a set of positive measure if it isn't null.
	\end{proof}
\end{enumerate}

\section{Fall 2014}
\begin{enumerate}
	\item Let $A$ be the collection of all subsets of $\reals$ that consist of exactly 5 points. Find the $\sigma$-algebra of sets generated by $A$.
	\begin{solution}
		By intersecting five element sets with exactly one point in common we can obtain all singleton subsets of $\mathbb{R}$. We claim that the $\sigma$-algebra generated by the singleton sets, which will be the $\sigma$-algebra generated by $A$, consists of all countable or co-countable subsets of $\mathbb{R}$.\\

		\noindent Call the $\sigma$-algebra consisting of all countable or co-countable sets $\mcal{A}$. Since $\mcal{A}$ contains all singletons, we clearly have $\sigma(A)\subseteq \mcal{A}$. Conversely, let $S\in \mcal{A}$. If $S$ is countable, then it is a countable union of singletons, so $S\in \sigma(A)$. On the other hand, if $S$ is co-countable, then its complement is in $\sigma(A)$. Since $\sigma(A)$ is closed under taking complements, this puts $S$ in $\sigma(A)$ as well. We conclude that $\sigma(A) = \mcal{A}$.
	\end{solution}

	\item Assume that $f\in L^1(0, 1)$ is a non-negative real-valued function satisfying $\int_{[0,1]}f(x)\ dx = 1$. Show that
	\[
	\int_{[0,1]}\frac{1}{f(x)}\ dx \geq 1.
	\]
	\begin{proof}
		Since $f\in L^1$ and $f\geq 0$, we have that $\sqrt{f}\in L^2$. We then have by H\"older's inequality
		\begin{align*}
			1 &= \int 1\ dx\\
			&= \int \frac{\sqrt{f}}{\sqrt{f}}\ dx\\
			&\leq \Lp{\sqrt{f}}{2} \cdot \Lp{1/\sqrt{f}}{2}\\
			&= \sqrt{\Lp{f}{1}}\cdot \sqrt{\Lp{1/f}{1}}\\
			&= \sqrt{\Lp{1/f}{1}}.
		\end{align*}
	\end{proof}

	\item Denote
	\[
	E = \left\{x\in [0,1]: \text{there exist infinitely many $p, q\in \naturals$ such that }|x-\frac{p}{q}| \leq \frac{1}{q^3}\right\}.
	\]
	Show that $m(E) = 0$.
	\begin{proof}
		Let $E_{p,q} = \{x\in [0,1]: |x-p/q|\leq 1/q^3\}$ where $p,q$ range over $\naturals$. Note that since we're confined to $[0,1]$, these sets are empty for $p>q$ for any fixed $q$. We also have that $m(E_{p,q}) \leq \frac{2}{q^3}$. We can then sum (using Tonelli to sum over $p$ and $q$ individually)
		\begin{align*}
			\sum_{p,q\in \naturals}m(E_{p,q}) &= \sum_{q\in \naturals}\sum_{0\leq p< q}m(E_{p,q})\\
			&\leq \sum_{q\in \naturals}q\cdot \frac{2}{q^3}\\
			&= \frac{\pi^2}{3}.
		\end{align*}
		Since this sum is finite, by Borel-Cantelli we must have that $m(\limsup E_{p,q}) = 0$. $\limsup E_{p,q}$ is the set of $x\in [0,1]$ belonging to infinitely many $E_{p,q}$, which is exactly the definition of $E$.
	\end{proof}

	\item Assume that $\eta \in L^1(\reals)$ is a non-negative function satisfying $\int_\reals\eta\ dx = 1$. Show that for any $f\in L^1(\reals)$,
	\[
	\Lp{f*\eta}{1} \leq \Lp{f}{1}.
	\]
	\begin{proof}
		We use Tonelli's theorem
		\begin{align*}
			\int |(f*\eta)(x)|\ dx &\leq \int\int |f(x-y)|\eta(y)\ dydx\\
			&= \int\int |f(x-y)|\eta(y)\ dxdy\\
			&= \Lp{f}{1}\int\eta(y)\ dy\\
			&= \Lp{f}{1}.
		\end{align*}
	\end{proof}

	\item Let $f:\reals\to \reals$ be continuous and periodic with period one. Prove that
	\[
	\lim_{n\to \infty}\int_0^1f(nx)\cos^2(2\pi x)\ dx = \frac{1}{2}\int_0^1f(x)\ dx.
	\]
	\begin{proof}
		The idea is to replace the cosine with the characteristic function of an interval $[a,b]$ and show that
		\[
		\lim_{n\to \infty}\int_0^1f(nx)\chi_{[a,b]}(x)\ dx = (b-a)\int_0^1f(x)\ dx.
		\]
		Since the step functions are dense in $L^1$, we can then apply an approximation argument to show that
		\begin{align*}
			\lim_{n\to \infty}\int_0^1f(nx)\cos^2(2\pi x)\ dx = \left(\right)
		\end{align*}
	\end{proof}
\end{enumerate}

\section{Spring 2014}
\begin{enumerate}
	\item Let $A$ be a subset of $\reals$ of positive Lebesgue measure. Prove that there exist $k,n\in\naturals$ and $x,y\in A$ with $|x-y| = k/2^n$.
	\begin{proof}
		The main idea is to show that the difference set $A-A$ contains a neighborhood of the origin. Since the set of dyads, $D = \{k/2^n: k\in \integers, n\in \naturals\}$, is dense in $\reals$, it must intersect the interval inside $A-A$.\\

		Let's show that $A-A$ contains an interval. If we assume that $A$ has positive \textit{finite} measure (just intersect $A$ with $[-N, N]$ for sufficiently large $N$), then the function $\varphi = \chi_A *\chi_{-A}$ is continuous as the convolution of an $L^\infty$ function with an $L^1$ function. We see that
		\[
		\varphi(0) = \int_\reals \chi_A(t)\chi_{-A}(0-t)\ dt = m(A).
		\]
		Since $m(A)>0$ and $\varphi$ is continuous, we have that $\varphi$ is positive on some neighborhood of the origin, say $(-\delta, \delta)$. We claim that $(-\delta, \delta)$ is in $A-A$. If $\varphi(x)>0$, then the integrand $\chi_A(t)\chi_{-A}(x-t)$ must be nonzero for some $t$. Then $t\in A$ and $x-t\in -A$, so $x=t +(x-t)\in A-A$.\\

		Now let's show that $D$ is dense in $\reals$. Given any $x\in \reals$ and any $\epsilon>0$ we'll show that there is a dyad in the $\epsilon$-neighborhood of $x$. Choose $n$ such that $\frac{1}{2^n}<\epsilon$ and $k$ such that $k \leq x\cdot 2^{n}\leq k+1$. Then $k/2^n$ is in the $\epsilon$-neighborhood of $x$.
	\end{proof}

	\item Either prove or give a counterexample: If a sequence of functions $f_n$ on a measure space $(X, \mu)$ satisfies $\int_X|f_n|\ d\mu \leq \frac{1}{n^2}$, then $f_n\to 0$ $\mu$-a.e.
	\begin{solution}
		This is true. By the monotone convergence theorem we have that $\int \sum |f_n| = \sum \int |f_n|$. By hypothesis, the second sum is finite, so $\sum |f_n|$ is integrable, and therefore finite a.e.. If this sum is finite a.e. then $|f_n(x)| \to 0$ for almost all $x$.
	\end{solution}

	\item Let $f\in L^4[a,b]$ and let $F(x) = \int_a^xf(x)\ dx$. Show that $\lim_{h\to 0}\frac{F(x+h)-F(x)}{h^{3/4}} = 0$ for all $x\in (a,b)$.
	\begin{proof}
		We have that
		\[
		|F(x+h)-F(x)| \leq \int_a^b|f(t)|\chi_{[x,x+h]}(t)\ dt.
		\]
		The trick here is that we can square the indicator function at no cost. By H\"older we then have
		\begin{align*}
			|F(x+h)-F(x)| &\leq \Lp{f\cdot \chi_{[x,x+h]}}{4}\cdot \Lp{\chi_{[x,x+h]}}{4/3}\\
			&= \Lp{f\cdot \chi_{[x,x+h]}}{4}\cdot h^{3/4}.
		\end{align*}
		Now by the absolute continuity of the integral, we can choose $h$ small enough that the first factor on the last line above is small. The result then follows.
	\end{proof}

	\item Assume $f,g\in L^2(\reals)$. Define
	\[
	A(x) = \int_\reals f(x-y)g(y)\ dy.
	\]
	Show that $A(x)\in C(\reals)$ and
	\[
	\lim_{|x|\to \infty}A(x) = 0.
	\]
	\begin{proof}
		By H\"older's inequality we can see that $|A(x)| \leq \Lp{f}{2}\Lp{g}{2}$ for all $x$. As for continuity, we have
		\begin{align*}
			|A(x+h)-A(x)| \leq \int |[f(x+h-y)-f(x-y)]g(y)|\ dy.
		\end{align*}
		The idea is to approximate $f$ by a continuous function with compact support. 
	\end{proof}

	\item Is it possible for a continuous function $f: [0,1]\to \reals$ to have
	\begin{enumerate}
		\item Infinitely many strict local minima?
		\begin{solution}
			Yes. For example, let $f(x) = x\sin\frac{1}{x}$. $f$ is continuous as $\lim_{x\to 0}f(x) = 0$ since $\sin \frac{1}{x}$ is bounded near the origin. Any local minimum of $\sin \frac{1}{x}$, of which there are infinitely many accumulating at the origin, is a local minimum of $f$ as well. 
		\end{solution}
		\item Uncountably many strict local minima?
	\end{enumerate}

	\item Let $A$ be the collection of functions $f\in L^1(X, \mu)$ such that $\Lp{f}{1}=1$ and $\int_X f\ d\mu = 0$. Prove that for every $g\in L^\infty(X, \mu)$,
	\[
	\sup_{f\in A}\int_Xfg\ d\mu = \frac{1}{2}(\esssup g - \essinf g).
	\]
	\begin{proof}
		
	\end{proof}
\end{enumerate}

\end{document}