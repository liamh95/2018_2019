%% Please change the file name by replacing N with the apporpriate number
%% corresponding to the current homework and XX with your initials.
%% https://www.math.uci.edu/~gpatrick/jsOnline/hw1.html

\documentclass[11pt,letterpaper]{report}
\usepackage{amssymb,amsfonts,color,graphicx,amsmath,enumerate}
\usepackage{tikz} %This package offers the ability to draw pictures
\usepackage{amsthm}

\newcommand{\naturals}{\mathbb{N}}
\newcommand{\integers}{\mathbb{Z}}
\newcommand{\complex}{\mathbb{C}}
\newcommand{\reals}{\mathbb{R}}
\newcommand{\exreals}{\overline{\mathbb{R}}}
\newcommand{\mcal}[1]{\mathcal{#1}}
\newcommand{\mable}{measurable}
\newcommand{\quats}{\mathbb{H}}
\newcommand{\rationals}{\mathbb{Q}}
\newcommand{\norm}{\trianglelefteq}
\newcommand{\Aut}{\text{Aut}}
\newcommand{\disk}{\mathbb{D}}
\newcommand{\halfplane}{\mathbb{H}}
\newcommand{\Lp}[2]{\left\|{#1}\right\|_{L^{#2}}}
\newcommand{\supp}[1]{\text{supp}({#1})}
\newcommand{\Hom}[2]{\text{Hom}_{{#1}}({#2})}
\newcommand{\tr}{\text{tr}}
\newcommand{\field}[1]{\mathbb{F}_{{#1}}}
\newcommand{\Gal}[1]{\text{Gal}\left({#1}\right)}
\newcommand{\esssup}{\text{ess sup }}
\newcommand{\essinf}{\text{ess inf }}
\newcommand{\affine}{\mathbb{A}}

\newenvironment{solution}
{\begin{proof}[Solution]}
{\end{proof}}

\voffset=-3cm
\hoffset=-2.25cm
\textheight=24cm
\textwidth=17.25cm
\addtolength{\jot}{8pt}
\linespread{1.3}

\begin{document}
% Please give relevant information
\begin{center}
{\bf \Large Real Analysis Qualifying Exams} %Replace N with the appropriate number
\vspace{0.2cm}
\hrule
\end{center}

\section*{Spring 2016}
\begin{enumerate}
	\item Assume $f\in L^1[0,1]$. Compute
	\[
	\lim_{k\to \infty}\int_{[0,1]}|f|^{1/k}\ dx.
	\]
	\begin{solution}
		Let's split this integral into three regions.
		\[
		\int_{[0,1]}|f|^{1/k}\ dx = \int_{f=0}|f|^{1/k}\ dx + \int_{0<|f|\leq 1}|f|^{1/k}\ dx + \int_{|f|>1}|f|^{1/k}\ dx.
		\]
		The integral over the first region is clearly zero for all $k$. On the second region we have that $|f|^{1/k} \leq 1$ for all $k$. Since the interval $[0,1]$ has finite measure, we have that the constant function 1 is in $L^1(\{x: 0<|f|\leq 1\})$, so the dominated convergence theorem says that the integral over the second region goes to $m(\{0<|f|\leq 1\})$. Similarly, on the third region we have that $|f|^{1/k}\leq |f|$, which is in $L^1$, so the dominated convergence theorem says that the third integral goes to $m(\{|f|>1\})$. Combining these, we have that
		\[
		\lim_{k\to \infty}\int_{[0,1]}|f|^{1/k}\ dx = m(\{|f|>0\}).
		\]
	\end{solution}

	\item Let $\{f_n\}$ be a sequence of measurable functions on $[0,1]$ and $0\leq f_n\leq 1$ a.e. Assume that
	\[
	\lim_{n\to \infty}\int_{[0,1]}f_ng\ dx = \int_{[0,1]}fg\ dx
	\]
	for some $f\in L^1[0,1]$ and any $g\in C[0,1]$. Prove that $0\leq f\leq 1$ a.e.
	\begin{solution}
		Since $f\in L^1[0,1]$, by the Lebesgue differentiation theorem we have that
		\begin{equation}\label{s16_2_ldt}
		\frac{1}{m(E)}\int_Ef(t)\ dt \to f(x)
		\end{equation}
		as $E$ shrinks to $x$ for almost all $x$. Furthermore, since $0\leq f_n\leq 1$ we also have that
		\[
		\frac{1}{m(E)}\int_Ef_n(t)\ dt\to f_n(x)\in [0,1]
		\]
		as $E$ shrink to $x$ for almost all $x$.  Intuitively, we'd like to replace the integral of $f$ in (\ref{s16_2_ldt}) with a limit of integrals of $f_n$.\\

		\noindent We claim that the function $g$ in the given hypothesis can be replaced with the indicator function of an interval $\chi_I$. To see this, let $g_m$ be a sequence of continuous functions with $g_m\to \chi_I$ in $L^1$ and $0\leq \chi_I\leq 1$. By extracting a subsequence we can assume that $g_m\to \chi_I$ a.e. as well. We then have
		\[
		\int_0^1|f_n\chi_I - f\chi_i| \leq \int_0^1|f_n\chi_I-f_ng_m| + \int_0^1|f_ng_m-fg_m| + \int_0^1|fg_m-f\chi_i|.
		\]
		Since $\|f_n\|_{L^\infty}\leq 1$, we have that the first integral on the RHS can be made small uniformly in $n$ by picking $m$ large. The second integral goes to zero as $n\to \infty$ by hypothesis since $g_m$ is continuous. The third integral can be made small for $m$ large by dominated convergence since $|fg_m| \leq |f|\in L^1$.\\

		For almost all $x$, if $I_k$ is a sequence of intervals shrinking to $x$ then
		\begin{align*}
		\frac{1}{m(I_k)}\int_{I_k}f\ dx &= \frac{1}{m(I_k)}\int f\chi_{I_k}\ dx\\
		&= \lim_{n\to \infty}\frac{1}{m(I_k)}\int f_n\chi_{I_k}\ dx.
		\end{align*}
		Since $0\leq f_n\leq 1$, the RHS is in $[0, 1]$ for almost all $x$. By the Lebesgue differentiation theorem we then have that $0\leq f\leq 1$ a.e.
	\end{solution}

	\item Let $f,g\in L^2(\reals, \mcal{M}_L, \mu_L)$. Show that $f*g$ is a continuous function on $\reals$ vanishing at infinity, that is, $f*g\in C(R)$ and $\lim_{|x|\to \infty}(f*g)(x) = 0$.
	\begin{proof}
		For any $h$ we have by H\"older's inequality
		\begin{align}\label{s16_3_main}
			|(f*g)(x+h)-(f*g)(x)| &= \left|\int f(t)[g(x+h-t) - g(x-t)]\ dt\right|\\
			&\leq \|f\|_{L^2}\cdot \|g_h-g\|_{L^2},
		\end{align}
		where $F_h(x) = F(x+h)$ for any function $F$. Now for any $\epsilon>0$ we can find $\varphi\in C_0(\reals)$ with $\|g-\varphi\|_{L^2} = \|g_h-\varphi_h\|_{L^2} <\epsilon$. By the triangle inequality we then have
		\begin{align*}
		\Lp{g_h-g}{2} &\leq \Lp{g_h-\varphi_h}{2} + \Lp{\varphi_h-\varphi}{2} +\Lp{\varphi-g}{2}\\
		&< \Lp{\varphi_h-\varphi}{2} + 2\epsilon.
		\end{align*}
		Suppose that $\varphi$ has support contained in the compact set $K$. If we pick $h$ small enough then we can guarantee that $\varphi_h-\varphi$ is supported on a set with measure at most $2\cdot m(K)$. Now since $\varphi$ is continuous with compact support, it is uniformly continuous, so we can choose $h$ small enough that $|\varphi_h(x)-\varphi(x)| = |\varphi(x+h)-\varphi(x)|<\epsilon$ for all $x$ in the support of $\varphi_h-\varphi$. For such $h$ we have
		\[
		\Lp{\varphi_h-\varphi}{2} \leq \epsilon \cdot (2\cdot m(K))^{1/2},
		\]
		so (\ref{s16_3_main}) can be made arbitrarily small, which shows that $f*g$ is continuous.\\

		\noindent First we claim that if $\varphi$ and $\psi$ are continuous with compact support then $\varphi*\psi$ vanishes at infinity. By definition we have that
		\[
		(\varphi*\psi)(x) = \int\varphi(t)\psi(x-t)\ dt.
		\]
		The product $\varphi(t)\psi(x-t)$ is nonzero only if $t$ is in the support of $\varphi$ and $x-t$ is in the support of $\varphi$. If pick $x$ large enough then supports of $t\mapsto \varphi(t)$ and $t\mapsto \psi(x-t)$ are disjoint, so this integral is zero.\\

		\noindent Let $f_n$ and $g_n$ be sequences in $C_0(\reals)$ converging in $L^2$ to $f$ and $g$, respectively. We then have
		\begin{align*}
			|(f*g)(x)-(f_n*g_n)(x)| &\leq |(f*g)(x)-(f_n*g)(x)| + |(f_n*g)(x)-(f_n*g_n)(x)|\\
			&\leq \Lp{g}{2}\cdot \Lp{f-f_n}{2} + \Lp{f_n}{2}\cdot \Lp{g-g_n}{2}.
		\end{align*}
		Since $f_n\to f$ and $g_n\to g$ in $L^2$, we have that $f_n*g_n$ converges uniformly to $f*g$. Since $f_n*g_n$ vanishes at infinity, we must then have that $f*g$ vanishes at infinity.
	\end{proof}

	\item Let $(X, \mcal{A}, \mu)$ be a finite measure space, and let $p_1\in (1, \infty]$. Let $\{f_n\}$ be a uniformly bounded sequence in $L^{p_1}(X, \mcal{A}, \mu)$. Suppose $f = \lim_{n\to \infty}f_n$ exists $\mu$-a.e. Prove that $f\in L^p(X, \mcal{A}, \mu)$ for all $p\in [1, p_1]$ and $f_n\to f$ in $L^p(X, \mcal{A}, \mu)$ for all $p\in [1, p_1)$.
	\begin{proof}
		Suppose that $\Lp{f_n}{p_1}\leq M$ for all $n$. First we claim that the $f_n$ are in $L^p(X, \mcal{A}, \mu)$ for all $p\in [1, p_1]$. In fact, they are uniformly bounded:
		\begin{align*}
			\int_X|f_n|^p &= \int_{|f_n|<1}|f_n|^p + \int_{|f_n|\geq 1}|f_n|^p\\
			&\leq \int_{|f_n|<1}1 + \int_{|f_n|\geq 1}|f_n|^{p_1}\\
			&\leq \mu(\{f\leq 1\}) + M^{1/p_1}.
		\end{align*}
		Since $(X, \mcal{A}, \mu)$ is a finite measure space, this quantity is finite, so $f_n\in L^p(X, \mcal{A}, \mu)$ for all $n$ and $p\in [1, p_1]$. We can then use the fact that $f_n\to f$ a.e. and Fatou's lemma to show that $f\in L^p(X, \mcal{A}, \mu)$ for $p\in [1, p_1]$:
		\[
		\int_X|f|^p \leq \liminf_{n\to \infty}\int_X|f_n|^p<\infty,
		\]
		where the finiteness follows from the $L^p$ uniform-boundedness of the $f_n$.\\

		\noindent To establish convergence in $L^p$, $p\in [1, p_1)$ our plan is to use the Vitali convergence theorem. The family $f_n$ is tight over $X$ since $X$ is a finite measure space and we're given that $f_n\to f$ a.e., so it only remains to show that the $f_n$'s are uniformly integrable. Intuitively, since the $f_n$'s are in $L^p$, the measure of the set $\{f_n \geq N\}$ should shrink as $N$ grows.  Now since $p<p_1$, if $N>1$ then
		\[
		|f_n|^p\chi_{\{|f_n|\geq N\}}N^{p_1-p}\leq |f_n|^{p_1}.
		\]
		If we integrate both sides over any measurable set $E$ we have
		\[
		\int_{E\cap \{|f_n|\geq N\}}|f_n|^p \leq \frac{M}{N^{p_1-p}}.
		\]
		On the complement we have
		\[
		\int_{E\cap \{|f_n|<N\}}|f_n|^p \leq N^p\cdot \mu(E).
		\]
		Putting these together, we have that
		\begin{align*}
			\int_E|f_n|^p &= \int_{E\cap \{|f_n|\geq N\}}|f_n|^p + \int_{E\cap \{|f_n|<N\}}|f_n|^p\\
			&\leq \frac{M}{R^{p_1-p}} + R^p\cdot\mu(E).
		\end{align*}
		If we choose $R$ so that $M/R^{p_1-p}<\epsilon/2$ and $E$ so that $R^p\cdot \mu(E)<\epsilon/2$ then we'll have that $\int_E |f_n|^p<\epsilon$ for any $E$ of sufficiently small measure, so the $f_n$'s are uniformly integrable. By the Vitali convergence theorem we have that $f_n\to f$ in $L^p$ for $p\in [1, p_1)$.
	\end{proof}

	\item Let $(X, \mcal{A}, \mu)$ be a measure space, and let $f: X\to [0, \infty)$ be $\mcal{A}$-measurable. Consider the measure space $(\reals, \mcal{B}_\reals, \mu_L)$, where $\mcal{B}_\reals$ is the Borel $\sigma$-algebra on $\reals$ and $\mu_L$ is the Lebesgue measure, and form the product measure space $(X\times \reals, \sigma(\mcal{A}\times \mcal{B}_\reals), \mu\times \mu_L)$. Define $E\subset X\times R$ by $(x,y)\in E\iff y\in [0, f(x))$. Prove that $E\in \sigma(\mcal{A}\times \mcal{B}_\reals)$ and $(\mu\times \mu_L)(E) = \int_Xf\ d\mu$.
	\begin{proof}
		A function is measurable if it pulls measurable sets back to measurable sets. The plan is then to write $E$ is a union and/or intersection of preimages of measurable sets under measurable functions. The function $F(x,y) = f(x)$ is measurable since
		\[
		F^{-1}[(-\infty, \alpha]) = \{(x,y): f(x)\leq \alpha\} = \{x: f(x)\leq \alpha\} \times \reals \in \sigma(\mcal{A}\times \mcal{B}_\reals),
		\]
		as $f$ is $\mu$-measurable. We also clearly have that the function $G(x,y) = y$ is measurable. Now consider the function $H(x,y) = y-f(x)$. $H$ is measurable as it is the difference of the measurable functions $G$ and $F$. We then have that $E$ is measurable through the following decomposition
		\begin{align*}
			E &= \{(x,y): 0\leq y<f(x)\}\\
			&= \{(x,y): y\geq 0\} \cap \{(x,y): y<f(x)\}\\
			&= G^{-1}[[0, \infty)]\cap H^{-1}[(-\infty, 0)].
		\end{align*}
		If $\{f>0\}$ is $\sigma$-finite we can use Tonelli's theorem to say
		\begin{align*}
			(\mu\times \mu_L)(E) &= \int_{X\times \reals}\chi_E(x,y)\ d(\mu\times \mu_L)\\
			&= \int_X\int_\reals\chi_E(x,y)\ d\mu_Ld\mu\\
			&= \int_X\int_\reals \chi_{[0, f(x))}(y)\ dyd\mu\\
			&= \int_Xf(x)\ d\mu.
		\end{align*}
		On the other hand, suppose that $\{f>0\}$ is note $\sigma$-finite. We claim that $\int_Xf\ d\mu = +\infty$. Indeed, since we can decompose this set into a countable union,
		\begin{equation}\label{s16_5_main}
		\{f>0\} = \bigcup_{m=1}^\infty \{\frac{1}{m+1}<f\leq \frac{1}{m}\} \cup \bigcup_{n=1}^\infty\{n<f\leq n+1\},
		\end{equation}
		we must have that one of these sets has infinite measure. We need to show that $(\mu\times \mu_L)(E) = +\infty$ too. For any $\alpha,\beta>0$ we have that if $\alpha\leq f(x)<\beta$ then the product set
		\[
		\{x: \alpha\leq f(x)<\beta\} \times \{y: 0\leq \alpha\}
		\]
		is contained in $E$. This product set has measure $\alpha\cdot \mu_L\{\alpha\leq f<\beta\}$, so by monotonicity we have that
		\[
		\alpha\cdot \mu_L\{\alpha\leq f<\beta\} \leq (\mu\times \mu_L)(E)
		\]
		for all $\alpha,\beta>0$. But by the decomposition (\ref{s16_5_main}), we have that some set of the form $\{\alpha\leq f(x)<\beta\}$ must have infinite measure, so we must have $(\mu\times \mu_L)(E) = +\infty$.
	\end{proof}

	\item Let $f\in L^1(\reals)$ and let $a_1, \ldots, a_k\in \reals$ and $b_1, \ldots, b_k\in \reals\setminus\{0\}$. Assume that the quotients $\frac{a_j}{b_j}$ are all distinct. Determine
	\[
	\lim_{t\to \infty}\int\left|\sum_{j=1}^kf(b_jx+ta_j)\right| dx.
	\]
	\begin{solution}
		Let $\varphi\in C_0(\reals)$ be such that $\Lp{f-\varphi}{1}<\epsilon$. Our plan is to compute the desired limit with $\varphi$ in place of $f$ and then argue that the difference can be made small. We have
		\[
		\int\left|\sum_{j=1}^k\varphi(b_jx+ta_j)\right| dx = \int\left|\sum_{j=1}^k\varphi\left[b_j\left(x+\frac{a_j}{b_j}t\right)\right]\right|dx
		\]
		Now $\varphi(b_jx+ta_j)$ is $\varphi$ stretched horizontally by a factor of $b_j$ and shifted over $a_j/b_j$. Since the support of $\varphi$ is compact and the $a_j/b_j$ are distinct, the supports of these transformations are disjoint for sufficiently large $t$. When these supports are disjoint we then have
		\begin{align*}
		\int\left|\sum_{j=1}^k\varphi(b_jx+ta_j)\right| dx &= \int\sum_{j=1}^k|\varphi(b_jx+ta_j)|\ dx\\
		&= \Lp{\varphi}{1}\cdot \sum_{j=1}^k\frac{1}{b_j}.
		\end{align*}
		That we can approximate the desired sum for $f\in L^1$ follows from the reverse triangle inequality.
		\begin{align*}
			\left|\int\left|\sum_{j=1}^kf(b_jx+ta_j)\right|dx - \int\left|\sum_{j=1}^k\varphi(b_jx+ta_j)\right|dx\right| &\leq \sum_{j=1}^k\int|f(b_jx+ta_j)-\varphi(b_jx+ta_j)|\ dx\\
			&= \epsilon\cdot \sum_{j=1}^k\frac{1}{b_k}.
		\end{align*}
	\end{solution}
\end{enumerate}

\end{document}